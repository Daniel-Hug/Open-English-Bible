

        \definemarking[RAChapter]
        \definemarking[RABook]
        \definemarking[RASection]

        \setuppapersize [A5][letter]
    %	\setuparranging [2UP,rotated,doublesided]
        \setuppagenumbering [alternative=doublesided]
        \setuplayout [location=middle, marking=on]

    %	\definefontsynonym [HoeflerTextRegular] [file:HoeflerText-Regular] [features=default]
    %	\definefontsynonym [HoeflerTextItalic] [file:HoeflerText-Italic] [features=default]
    %	\definefontsynonym [HoeflerTextBold] [file:HoeflerText-Black] [features=default]
    %	\definefontsynonym [Serif] [HoeflerTextRegular]
    %	\definefontsynonym [SerifBold] [HoeflerTextBold]
    %	\definefontsynonym [SerifItalic] [HoeflerTextItalic]

        \usetypescript[pagella][handling][highquality]
    %	\definefontfeature [hz] [default] [protrusion=pure, mode=node, script=latn]
    %	\definetypeface [biblefont] [rm] [serif] [pagella] [default] [features=hz]
        \setupbodyfont [pagella, 10.5pt]
        \setupalign [handling]

        \setupbodyfontenvironment[default][em=italic]

        \setuppagenumbering[location=]
        \setupheadertexts[{\em \getmarking[RASection]}][{\getmarking[RABook] ~\getmarking[RAChapter]}]
        \setupfootertexts[pagenumber]
        \setuphead[title][header=high,footer=chapter,page=right]

        \setupspacing[packed]   % normal word space at the end of sentences
        \setupwhitespace[none]  % no space between paragraphs
        \setupindenting[yes, small, next]
        \setupinterlinespace[line=13pt] % Line spacing

        \define[1]\V{\setupinmargin[style=small] \inmargin{#1}}
        \define[1]\C{\setupinmargin[style=bold] \inmargin{#1} \marking[RAChapter]{#1}}
        \define[1]\MS{\section{#1} \marking[RASection]{#1}}
        \define[1]\MSS{{\midaligned{\em #1}}}
        \define[1]\MT{{\midaligned{\sc #1}}\blank ~}
        \define[1]\RAHeader{\chapter{#1} \marking[RABook]{#1}}

        \emergencystretch\maxdimen

        \setuphead[chapter][number=no, textstyle=cap, before=\blank, after=\blank, align={middle, nothyphenated, verytolerant}]
        \setuphead[section][number=no, textstyle=em, before=\blank, after=\blank, align={middle, nothyphenated, verytolerant}]

        \setuplist[chapter][alternative=c]

        \def\CapStretchAmount{.08em}
        \def\CapStretch#1{\def\stretchedspaceamount{\CapStretchAmount}\stretchednormalcase{\sc #1}}

        \usemodule[lettrine]

        \starttext

        \title{Open English Bible}

        \title{Table of Contents}
        \placelist[chapter]
        \RAHeader{Preface and Frequently Asked Questions}

\subsubject{What is this?}

This is a preview release of the Gospel According to Mark for the 
new English translation of the Bible, the
\quotation{Open English Bible}.

\subsubject{Why another translation?}

The English language market is currently very well covered by
translations for all seasons, everything from the Message, through
the CEV, NRSV, AV, and so on through the acronym soup. The OEB aims
at the levels of accuracy and natural language achieved by these
bibles, but the purpose of the OEB is not to gift the world with a
more accurate or \quote{better} translation per se. The existing
commercial English translations are, for the large part, of a high
standard of accuracy and are the result of much work by
knowledgeable and well intended translators.

However, with the honourable exception of the World English Bible
(WEB) - and of course the Authorised Version - all of these bibles
are subject to copyright, and owned by a particular organisation or
publisher. Without wishing to criticise these organisations, their
translations naturally have limitations on what end users can do
with them.

The OEB will have no restrictions on what its readers and users can
do with it (for both good and bad). They may quote it, publish it
in part or full, on their blogs, in their churches, remix it,
reword it, correct its egregious translation mistakes or indeed add
their own.

\subsubject{Isn't this dangerous?}

This generation have a greater ability to confirm the accuracy of
their bible translations than any previous generation in history.
The underlying Greek and Hebrew, lexicons, learned discussions and
flamewars are all just a click away - as are a dozen other
translations. Attempting to use copyright law to create authority
is an exercise doomed to failure, as are all attempts to enforce
truths by the power of the state.

\subsubject{Why not just use the World English Bible?}

The
\useURL[1][http://ebible.org][][World English Bible (WEB)]\from[1]
is an impressive achievement, and of great importance. However its
language, based as it is on the ASV, remains bound to the Tyndale
tradition. As a free counterpart to the NRSV and ESV it is
invaluable; however there is still a need for a free bible as a
counterpart to the NEB/REB and TNIV/NIV translations. As well, the
World English BIble is based upon the Byzantine Majority Text, and
the OEB is based on the heavily Alexandrian focused text of Wescott
\& Hort.

\subsubject{What English language texts is the OEB based on?}

The New Testament of the OEB is being formed on the base of the
\quotation{Twentieth Century New Testament}, in particular the
revised edition published in 1904. There is currently no decision
on how to tackle the Old Testament.

\subsubject{Why the Twentieth Century New Testament?}

The TCNT was one of the earliest 20th century attempts at a
translation in clear modern language aimed at the ordinary reader
and based on a modern textual base (ie Westcott and Hort).
Predating the mid--20th century translations such as the New
English Bible and even Moffatt's groundbreaking attempt, it is out
of copyright worldwide. The TCNT also has a particular resonance
with the open source and free content communities of today - it was
created by a loose collaboration of volunteers rather than a
top-down hierachy. A worthwhile article on the making of the TCNT
can be found on Google Books. Given the requirements of (a) modern
language and (b) public domain status the TCNT was the best
contender.

\subsubject{How will the Twentieth Century New Testament be edited?}

The language of the 20CNT will be edited:

\startitemize
\item
  to reflect modern English usage (including the use of \quote{they}
  as a third person single pronoun) at a reading level corresponding
  roughly to the NEB/REB or NRSV
\item
  to reflect modern scholarship, including on the translation of
  terms such as \quote{the Jews} in John and terms referring to
  sexual practices (see TNIV and Dr Ann Nyland's version)
\stopitemize

This editing will be moderate, aiming for a scholarly defensible
mainstream translation usable within a religious community rather
than a translation focused on a readership completely unfamiliar
with the Bible or Christianity (as an example, the OEB will be
comfortable with the words \quote{Christ} and \quote{Messiah} and
will not replace them with \quote{Annointed One} or similar). A
freely licensed translation for the audiences of the Better Life
Bible or the CEV is a project for another day.

\subsubject{How do I know the OEB is accurate and unbiased?}

Although all translators have opinions on the truth and meaning of
the ancient documents which form the Bible, the intent of this
revision of the TCNT is not to push any particular theological line
but to provide a freely usable translation of an Alexandrian
text-type based critical text in modern English. The
\useURL[1][][][GitHub account]\from[1] for the OEB contains the
original text of the Twentieth Century New Testament in
\useURL[2][http://ubs-icap.org/usfm][][USFM]\from[2], (which can be
checked against
\useURL[3][http://www.archive.org/details/twentiethcentury00newyiala][][the pdf of the published edition on Archive.org]\from[3]);
it also contains a list of all of the changes made to that text. As
can be seen from the article on the making of the TCNT mentioned
above, the TCNT itself was the product of a wide range of
translators of differing backgrounds and was not itself intended to
push a particular theological point of view.

\subsubject{What is the underlying textual basis for the OEB?}

The TCNT was based on the Westcott \& Hort critical text and the
OEB will not change that - it will remain based on W\&H and will
not be conformed to the Majority Text or NA27. Since the
\useURL[4][http://groups.google.com/group/open-scriptures/browse_thread/thread/f57701ff851ae905][][NA27 is subject to a claim of copyright]\from[4],
W\&H remains the best available (ie public domain) text.

\subsubject{What licence is the OEB under?}

The OEB will be available under a
\useURL[5][http://creativecommons.org/licenses/by/3.0/us/][][CC Attribution licence]\from[5],
allowing the maximum reuse. There will be a request that altered
versions be distributed under a different name.

\subsubject{What variations will be available?}

The translation will be available in US and Commonwealth/UK
spelling variations.

\subsubject{What formats will the OEB be available in?}

The OEB will be generated in as many formats as possible, including
PDF, HTML etc and a version will be available for purchase from
Lulu or similar print-on-demand provider.

\subsubject{Who is involved in this?}

At the moment, this is a single person project. I am a lawyer with
strong technical skills, but without formal training in either
translation or in Biblical languages. Accordingly, the OEB is not
an exercise in translation so much as editing. I will make all
changes transparently, and will endeavour to ensure that no new
wording is added that is not attested to in at least one modern,
scholarly, mainstream translation.

\subsubject{Where can I find more information?}

http://openenglishbible.org

\marking[RAChapter]{ } \marking[RABook]{ } \marking[RASection]{ }\RAHeader{Mark} \MT{The Good News According to Mark} \MS{The Preparation} \indenting[yes]\par  The beginning of the Good News about Jesus Christ. \indenting[yes]\par 
 \V{2}  It is said in the Prophet Isaiah — ‘Behold! I send my Messenger before your face; he shall prepare your way. 
 \V{3}  The voice of one crying aloud in the Wilderness: “Make ready the way of the Lord, make his paths straight.”’ 
 \V{4}  And in fulfillment of this, John the Baptizer appeared in the Wilderness, proclaiming a baptism upon repentance, for the forgiveness of sins. 
 \V{5}  The whole of Judea, as well as all the inhabitants of Jerusalem, went out to him; and they were baptized by him in the river Jordan, confessing their sins. \indenting[yes]\par 
 \V{6}  John was clad in clothing of camels’ hair, with a belt of leather round his waist, and lived on locusts and wild honey; 
 \V{7}  and he proclaimed — “There is coming after me one more powerful than I, and I am not fit even to stoop down and unfasten his sandals. 
 \V{8}  I have baptized you with water, but he will baptize you with the Holy Spirit.” \indenting[yes]\par 
 \V{9}  Now about that time Jesus came from Nazareth in Galilee, and was baptized by John in the Jordan. 
 \V{10}  Just as he was coming up out of the water, he saw the heavens rent apart, and the Spirit, like a dove, descending upon him, 
 \V{11}  and from the heavens came a voice — “You are my Son, the Beloved; in you I delight.” \indenting[yes]\par 
 \V{12}  Immediately afterwards the Spirit drove Jesus out into the Wilderness; 
 \V{13}  and he was there in the Wilderness forty days, tempted by Satan, and among the wild beasts, while the angels ministered to him. \MS{The Work in Galilee} \indenting[yes]\par 
 \V{14}  After John had been committed to prison, Jesus went to Galilee, proclaiming the Good News of God —  
 \V{15}  “The time has come, and the Kingdom of God is at hand; repent, and believe the Good News.” \indenting[yes]\par 
 \V{16}  As Jesus was going along the shore of the Sea of Galilee, he saw Simon and his brother Andrew casting a net in the Sea, for they were fishermen. 
 \V{17}  “Come and follow me,”  Jesus said,   “and I will set you to fish for people.” 
 \V{18}  They left their nets at once, and followed him. \indenting[yes]\par 
 \V{19}  Going on a little further, he saw James, Zebediah’s son, and his brother John, who also were in their boat mending the nets. 
 \V{20}  Jesus at once called them, and they left their father Zebediah in the boat with the crew, and went after him. \indenting[yes]\par 
 \V{21}  They walked into Capernaum. On the next Sabbath Jesus went into the Synagogue and began to teach. 
 \V{22}  The people were amazed at his teaching, for he taught them like one who had authority, and not like the Teachers of the Law. 
 \V{23}  Now there was in their Synagogue at the time a man under the power of a foul spirit, who called out: 
 \V{24}  “What do you want with us, Jesus of Nazareth? Have you come to destroy us? I know who you are — the Holy One of God!” 
 \V{25}  But Jesus rebuked the spirit:   “Be silent! Come out from him.” 
 \V{26}  The foul spirit threw the man into a fit, and with a loud cry came out from him. 
 \V{27}  They were all so amazed that they kept asking: “What is this? Strange teaching indeed! He gives his commands with authority even to the foul spirits, and they obey him!” 
 \V{28}  And the fame of Jesus spread at once in all directions, through the whole neighborhood of Galilee. \indenting[yes]\par 
 \V{29}  As soon as they had come out from the Synagogue, they went, with James and John, into the house of Simon and Andrew. 
 \V{30}  Now Simon’s mother-in-law was lying ill with fever, and they at once told Jesus about her. 
 \V{31}  Jesus went up to her and, grasping her hand, raised her up; the fever left her, and she began to wait upon them. \indenting[yes]\par 
 \V{32}  In the evening, after sunset, the people brought to Jesus all who were ill or possessed by demons; 
 \V{33}  and the whole city was gathered round the door. 
 \V{34}  Jesus cured many who were ill with various diseases, and drove out many demons, and would not permit them to speak, because they knew him to be the Christ. \indenting[yes]\par 
 \V{35}  In the morning, long before daylight, Jesus rose and went out, and, going to a lonely spot, there began to pray. 
 \V{36}  But Simon and his companions hastened after him; 
 \V{37}  and, when they found him, they exclaimed: “Every one is looking for you!” 
 \V{38}  But Jesus said to them:   “Let us go somewhere else, into the country towns near, that I may make my proclamation in them also; for that was why I came.” 
 \V{39}  And he went about making his proclamation in their Synagogues all through Galilee, and driving out the demons. \indenting[yes]\par 
 \V{40}  One day a leper came to Jesus and, falling on his knees, begged him for help. “If only you are willing,” he said, “you are able to make me clean.” 
 \V{41}  Moved with compassion, Jesus stretched out his hand and touched him, saying as he did so:   “I am willing; become clean.” 
 \V{42}  Instantly the leprosy left the man, and he became clean; 
 \V{43}  and then Jesus, after sternly warning him, immediately sent him away, and said to him: 
 \V{44}  “Be careful not to say anything to any one; but go and show yourself to the Priest, and make the offerings for your cleansing directed by Moses, as evidence of your cure.” 
 \V{45}  The man, however, went away, and began to speak about it publicly, and to spread the story so widely, that Jesus could no longer go openly into a town, but stayed outside in lonely places; and people came to him from every direction. \indenting[yes]\par 
 \C{2}  Some days later, when Jesus came back to Capernaum, the news spread that he was in a house there; 
 \V{2}  and so many people collected together, that after a while there was no room for them even round the door; and he began to tell them his Message. 
 \V{3}  Some people came bringing to him a paralyzed man, who was being carried by four bearers. 
 \V{4}  Being, however, unable to get him near to Jesus, owing to the crowd, they removed the roofing below which Jesus was; and, when they had made an opening, they let down the mat on which the paralyzed man was lying. 
 \V{5}  When Jesus saw their faith, he said to the man:   “Child, your sins are forgiven.” \indenting[yes]\par 
 \V{6}  But some of the Teachers of the Law who were sitting there were debating in their minds: 
 \V{7}  “Why does this man speak like this? He is blaspheming! Who can forgive sins except God?” 
 \V{8}  Jesus, at once intuitively aware that they were debating with themselves in this way, said to them:   “Why are you debating in your minds about this? 
 \V{9}  Which is easier? — to say to the paralyzed man, ‘Your sins are forgiven’? or to say ‘Get up, and take up your mat, and walk about’? 
 \V{10}  But that you may know that the Son of Man has power to forgive sins on earth”  — here he said to the paralyzed man —  
 \V{11}  “To you I say, Get up, take up your mat, and return to your home.” 
 \V{12}  The man got up, and immediately took up his mat, and went out before them all; at which they were amazed, and, as they praised God, they said: “We have never seen anything like this!” \indenting[yes]\par 
 \V{13}  Jesus went out again to the Sea; and all the people came to him, and he taught them. 
 \V{14}  As he went along, he saw Levi, the son of Alphaeus, sitting in the tax-office, and said to him:   “Follow me.”  Levi got up and followed him. \indenting[yes]\par 
 \V{15}  Later on he was in his house at table, and a number of tax-gatherers and outcasts took their places at table with Jesus and his disciples; for many of them were following him. 
 \V{16}  When the Teachers of the Law belonging to the party of the Pharisees saw that he was eating in the company of such people, they said to his disciples: “He is eating in the company of tax- gatherers and outcasts!” 
 \V{17}  Hearing this, Jesus said:   “It is not those who are in health that need a doctor, but those who are ill. I did not come to call the religious, but the outcast.” \indenting[yes]\par 
 \V{18}  Now John’s disciples and the Pharisees were keeping a fast, and people came and asked Jesus: “Why is it that John’s disciples and the disciples of the Pharisees fast, while yours do not?” 
 \V{19}  Jesus answered:   “Can the bridegroom’s friends fast, while the bridegroom is with them? As long as they have the bridegroom with them, they cannot fast. 
 \V{20}  But the days will come, when the bridegroom will be parted from them, and they will fast then — when that day comes. \indenting[yes]\par 
 \V{21}  “No one ever sews a piece of unshrunk cloth on an old garment; if they do, the patch tears away from it — the new from the old — and a worse rent is made. 
 \V{22}  And no one ever puts new wine into old wine-skins; if they do, the wine will burst the skins, and both the wine and the skins are lost. But new wine is put into fresh skins.” \indenting[yes]\par 
 \V{23}  One Sabbath, as Jesus was walking through the cornfields, his disciples began to pick the ears of wheat as they went along. 
 \V{24}  “Look!” the Pharisees said to him, “why are they doing what is not allowed on the Sabbath?” 
 \V{25}  “Have you never read,”  answered Jesus,   “what David did when he was in want and hungry, he and his companions —  
 \V{26}  How he went into the House of God, in the time of Abiathar the High Priest, and ate ‘the consecrated bread,’ which only the priests are allowed to eat, and gave some to his comrades as well?” \indenting[yes]\par 
 \V{27}  Then Jesus added:   “The Sabbath was made for people, and not people for the Sabbath; 
 \V{28}  so the Son of Man is lord even of the Sabbath.” \indenting[yes]\par 
 \C{3}  On another occasion Jesus went in to a Synagogue, where there was a man whose hand was withered. 
 \V{2}  And they watched Jesus closely, to see if he would cure the man on the Sabbath, so that they might have a charge to bring against him. 
 \V{3}  “Stand out in the middle,”  Jesus said to the man with the withered hand; 
 \V{4}  and to the people he said:   “Is it allowable to do good on the Sabbath — or harm? to save a life, or destroy it?” 
 \V{5}  As they remained silent, Jesus looked round at them in anger, grieving at the hardness of their hearts, and said to the man:   “Stretch out your hand.”  The man stretched it out; and his hand had become sound. 
 \V{6}  Immediately on leaving the Synagogue, the Pharisees and the Herodians united in laying a plot against Jesus, to put him to death. \blank\indenting[no]\par 
 \V{7}  \CapStretch{Then Jesus went away with his disciples to} the Sea, followed by a great number of people from Galilee. 
 \V{8}  A great number, hearing of all that he was doing, came to him from Judea, from Jerusalem, from Edom, from beyond the Jordan, and from the country round Tyre and Sidon. 
 \V{9}  So Jesus told his disciples to keep a small boat close by, for fear the crowd should crush him. 
 \V{10}  For he had cured many of them, and so people kept crowding upon him, that all who were afflicted might touch him. 
 \V{11}  The foul spirits, too, whenever they caught sight of him, flung themselves down before him, and screamed out: “You are the Son of God”! 
 \V{12}  But he repeatedly warned them not to make him known. \indenting[yes]\par 
 \V{13}  Jesus made his way up the hill, and called those whom he wished; and they went to him. 
 \V{14}  He appointed twelve — whom he also named ‘Apostles’ — that they might be with him, and that he might send them out as his Messengers, to preach, 
 \V{15}  and with power to drive out demons. 
 \V{16}  So he appointed the Twelve — Peter (which was the name that Jesus gave to Simon), 
 \V{17}  James, the son of Zebediah, and his brother John (to whom he gave the name of Boanerges, which means the Thunderers), 
 \V{18}  Andrew, Philip, Bartholomew, Matthew, Thomas, James the son of Alphaeus, Thaddaeus, Simon the Zealot, 
 \V{19}  and Judas Iscariot, the man that betrayed him. \indenting[yes]\par 
 \V{20}  Jesus went into a house; and again a crowd collected, so that they were not able even to eat their food. 
 \V{21}  When his relations heard of it, they went to take charge of him, for they said that he was out of his mind. \indenting[yes]\par 
 \V{22}  The Teachers of the Law, who had come down from Jerusalem, said: “He has Baal-zebub in him, and he drives the demons out by the help of Baal-zebub, their chief.” 
 \V{23}  So Jesus called them to him, and answered them in parables:   “How can Satan drive out Satan?  When a kingdom is divided against itself, it cannot last; 
 \V{24, 25}  and when a household is divided against itself, it will not be able to last. 
 \V{26}  So, if Satan is in revolt against himself and is divided, he cannot last — his end has come! \indenting[yes]\par 
 \V{27}  “No man who has got into a strong man’s house can carry off his goods, without first securing him; and not till then will he plunder his house. 
 \V{28}  I tell you that people will be forgiven everything — their sins, and all the slanders that they utter; 
 \V{29}  but whoever slanders the Holy Spirit remains unforgiven to the end; he has to answer for an enduring sin.” 
 \V{30}  This was said in reply to the charge that he had a foul spirit in him. 
 \V{31}  His mother and his brothers came, and stood outside, and sent to ask him to come to them. 
 \V{32}  There was a crowd sitting round Jesus, and some of them said to him: “Look, your mother and your brothers are outside, asking for you.” 
 \V{33}  “Who is my mother? and my brothers?”  was his reply. 
 \V{34}  Then he looked around on the people sitting in a circle round him, and said:   “Here are my mother and my brothers! 
 \V{35}  Whoever does the will of God is my brother and sister and mother.” \blank\indenting[no]\par 
 \C{4}  \CapStretch{Jesus again began to teach by the Sea}; and, as an immense crowd was gathering round him, he got into a boat, and sat in it on the Sea, while all the people were on the shore at the water’s edge. \indenting[yes]\par 
 \V{2}  Then he taught them many truths in parables; and in the course of his teaching he said to them: \indenting[yes]\par  “Listen! The sower went out to sow; 
 \V{3, 4}  and presently, as he was sowing, some of the seed fell along the path; and the birds came, and ate it up. 
 \V{5}  Some fell on rocky ground, where it had not much soil, and, having no depth of soil, sprang up at once; 
 \V{6}  but, when the sun rose, it was scorched, and, having no root, withered away. 
 \V{7}  Some of the seed fell among brambles; but the brambles shot up and completely choked it, and it yielded no return. 
 \V{8}  Some fell into good soil, and, shooting up and growing, yielded a return, amounting to thirty, sixty, and even a hundred fold.” 
 \V{9}  And Jesus said:   “Let any one who has ears to hear with hear.” \indenting[yes]\par 
 \V{10}  Afterwards, when he was alone, his followers and the Twelve asked him about his parables; 
 \V{11}  and he said:   “To you the hidden truth of the Kingdom of God has been imparted; but to those who are outside it all teaching takes the form of parables, that —  
 \V{12}  ‘Though they have eyes, they may see without perceiving; and though they have ears, they may hear without understanding; lest some day they should turn and be forgiven.’ \indenting[yes]\par 
 \V{13}  “You do not know the meaning of this parable!”  he went on;   “Then how will you understand all the other parables? 
 \V{14}  The sower sows the Message. 
 \V{15}  The People meant by the seed that falls along the path are these — where the Message is sown, but, as soon as they have heard it, Satan immediately comes and carries away the Message that has been sown in them. 
 \V{16}  So, too, those meant by the seed sown on the rocky places are the people who, when they have heard the Message, at once accept it joyfully; 
 \V{17}  but, as they have no root, they stand only for a short time; and so, when trouble or persecution arises on account of the Message, they fall away at once. 
 \V{18}  Those meant by the seed sown among the brambles are different; they are the people who hear the Message, 
 \V{19}  but the cares of life, and the glamour of wealth, and cravings for many other things come in and completely choke the Message, so that it gives no return. 
 \V{20}  But the people meant by the seed sown on the good ground are those who hear the Message, and welcome it, and yield a return, thirty, sixty, and even a hundred fold.” \indenting[yes]\par 
 \V{21}  Jesus said to them:   “Is a lamp brought to be put under the corn-measure or under the couch, instead of being put on the lampstand? 
 \V{22}  Nothing is hidden unless it is some day to come to light, nor was anything ever kept hidden but that it should some day come into the light of day. 
 \V{23}  Let all who have ears to hear with hear. \indenting[yes]\par 
 \V{24}  Take care what you listen to,”  said Jesus.   “The measure you mete will be meted out to you, and more will be added for you. 
 \V{25}  For, to those who have, more will be given; while, from those who have nothing, even what they have will be taken away.” \indenting[yes]\par 
 \V{26}  Jesus also said:   “This is what the Kingdom of God is like — like a man who has scattered seed on the ground, 
 \V{27}  and then sleeps by night and rises by day, while the seed is shooting up and growing — he knows not how. 
 \V{28}  The ground bears the crop of itself — first the blade, then the ear, and then the full grain in the ear; 
 \V{29}  but, as soon as the crop is ready, immediately he ‘puts in the sickle because harvest has come’.” \indenting[yes]\par 
 \V{30}  Jesus also said:   “To what can we liken the Kingdom of God? 
 \V{31}  By what can we illustrate it? Perhaps by the growth of a mustard-seed. This seed, when sown in the ground, though it is smaller than all other seeds, 
 \V{32}  Yet, when sown, shoots up, and becomes larger than any other herb, and puts out great branches, so that even ‘the wild birds can roost in its shelter.’” \indenting[yes]\par 
 \V{33}  With many such parables Jesus used to speak to the people of his Message, as far as they were able to receive it; 
 \V{34}  and to them he never used to speak except in parables; but in private to his own disciples he explained everything. \blank\indenting[no]\par 
 \V{35}  \CapStretch{In the evening of the same day}, Jesus said to them:   “Let us go across.” 
 \V{36}  So, leaving the crowd behind, they took him with them, just as he was, in the boat; and there were other boats with him. 
 \V{37}  A violent squall came on, and the waves kept dashing into the boat, so that the boat was actually filling. 
 \V{38}  Jesus was in the stern asleep upon the cushion; and the disciples roused him and cried: “Teacher! is it nothing to you that we are lost?” 
 \V{39}  Jesus rose and rebuked the wind, and said to the sea:   “Hush! Be still!”  Then the wind dropped, and a great calm followed. 
 \V{40}  “Why are you so timid?”  he exclaimed.   “Have you no faith yet?” 
 \V{41}  But they were struck with great awe, and said to one another: ”Who can this be that even the wind and the sea obey him?” 
 \C{5} \indenting[yes]\par  They came to the other side of the Sea — the country of the Gerasenes; 
 \V{2}  and, as soon as Jesus had got out of the boat, he met a man coming out of the tombs, who was under the power of a foul spirit, 
 \V{3}  and who made his home in the tombs. No one had ever been able to secure him, even with a chain; 
 \V{4}  for, though he had many times been left secured with fetters and chains, he had snapped the chains and broken the fetters to pieces, and no one could master him. 
 \V{5}  Night and day alike, he was continually shrieking in the tombs and among the hills, and cutting himself with stones. 
 \V{6}  Catching sight of Jesus from a distance, he ran and bowed to the ground before him, 
 \V{7}  shrieking out in a loud voice: “What do you want with me, Jesus, Son of the Most High God? For God’s sake do not torment me!” 
 \V{8}  For Jesus had said:   “Come out from the man, you foul spirit.” 
 \V{9}  And he asked him:   “What is your name?”  ”My name,” he said, ”is Legion, for there are many of us;” 
 \V{10}  and he begged Jesus again and again not to send them away out of that country. \indenting[yes]\par 
 \V{11}  There was a large drove of pigs close by, feeding on the hillside; 
 \V{12}  and the spirits begged Jesus: “Send us into the pigs, that we may take possession of them.” 
 \V{13}  Jesus gave them leave. They came out, and entered into the pigs; and the drove — about two thousand in number — rushed down the steep slope into the Sea and were drowned in the Sea. \indenting[yes]\par 
 \V{14}  On this the men who tended them ran away, and carried the news to the town, and to the country round; and the people went to see what had happened. 
 \V{15}  When they came to Jesus, they found the possessed man sitting there, clothed and in his right mind — the very man who had had the ‘Legion’ in him — and they were awe-struck. 
 \V{16}  Then those who had seen it related to them all that had happened to the possessed man, as well as about the pigs; 
 \V{17}  upon which they began to beg Jesus to leave their neighborhood. \indenting[yes]\par 
 \V{18}  As Jesus was getting into the boat, the possessed man begged him to let him stay with him. 
 \V{19}  But Jesus refused.   “Go back to your home, to your own people,”  he said,   “and tell them of all that the Lord has done for you, and how he took pity on you.” 
 \V{20}  So the man went, and began to proclaim in the district of the Ten Towns all that Jesus had done for him; and every one was amazed. \indenting[yes]\par 
 \V{21}  By the time Jesus had re-crossed in the boat to the opposite shore, a great number of people had gathered to meet him, and were standing by the Sea. 
 \V{22}  One of the Presidents of the Synagogue, whose name was Jaeirus, came and, as soon as he saw Jesus, threw himself at his feet with repeated entreaties. 
 \V{23}  “My little daughter,” he said, “is at the point of death; I beg you to come and place your hands on her, that her life may be spared.” 
 \V{24}  So Jesus went with him. A great number of People followed Jesus, and kept pressing round him. \indenting[yes]\par 
 \V{25}  Meanwhile a woman who for twelve years had suffered from hemorrhage, 
 \V{26}  and undergone much at the hands of many doctors, (spending all she had without obtaining any relief, but, on the contrary, growing worse), 
 \V{27}  heard about Jesus, came behind in the crowd, and touched his cloak. 
 \V{28}  “If I can only touch his clothes,” she said, “I shall get well!” 
 \V{29}  At once the mischief was stopped, and she felt in herself that she was cured of her complaint. 
 \V{30}  Jesus at once became aware of the power that had gone out from him, and, turning round in the crowd, he said:   “Who touched my clothes?” 
 \V{31}  “You see the people pressing round you,” exclaimed his disciples, “and yet you say   ‘Who touched me?’” 
 \V{32}  But Jesus looked about to see who had done it. 
 \V{33}  Then the woman, in fear and trembling, knowing what had happened to her, came and threw herself down before him, and told him the whole truth. 
 \V{34}  “Daughter,”  he said,   “your faith has delivered you. Go, and peace be with you; be free from your complaint.” \indenting[yes]\par 
 \V{35}  Before he had finished speaking, some people from the house of the President of the Synagogue came and said: “Your daughter is dead! Why should you trouble the Teacher further?” 
 \V{36}  But Jesus, overhearing what they were saying, said to the President of the Synagogue:   “Do not be afraid; only have faith.” 
 \V{37}  And he allowed no one to accompany him, except Peter, James, and John, the brother of James. 
 \V{38}  Presently they reached the President’s house, where Jesus saw a scene of confusion — people weeping and wailing incessantly. 
 \V{39}  “Why this confusion and weeping?”  he said on entering.   “The little child is not dead; she is asleep.” 
 \V{40}  They began to laugh at him; but he sent them all out, and then, with the child’s father and mother and his companions, went into the room where she was lying. 
 \V{41}  Taking her hand, Jesus said to her:   “Taleitha, koum!”  — which means   ‘little girl, I am speaking to you — Rise!’ 
 \V{42}  The little girl stood up at once, and began to walk about; for she was twelve years old. And, as soon as they saw it, they were overwhelmed with amazement; 
 \V{43}  but Jesus repeatedly cautioned them not to let any one know of it, and told them to give her something to eat. \indenting[yes]\par 
 \C{6}  On leaving that place, Jesus, followed by his disciples, went to his own part of the country. 
 \V{2}  When the Sabbath came, he began to teach in the Synagogue; and the people, as they listened, were deeply impressed. “Where did he get this?” they said, “and what is this wisdom that has been given him? and these miracles which he is doing? 
 \V{3}  Is not he the carpenter, the son of Mary, and the brother of James, and Joses, and Judas, and Simon? And are not his sisters, too, living here among us?” This proved a hindrance to their believing in him; 
 \V{4}  on which Jesus said:   “A prophet is not without honor, except in his own country, and among his own relations, and in his own home.” 
 \V{5}  And he could not work any miracle there, beyond placing his hands upon a few infirm persons, and curing them; 
 \V{6}  and he wondered at the want of faith shown by the people. Jesus went round the villages, one after another, teaching. \blank\indenting[no]\par 
 \V{7}  \CapStretch{He called the Twelve to him}, and began to send them out as his Messengers, two and two, and gave them authority over foul spirits. 
 \V{8}  He instructed them to take nothing but a staff for the journey- -not even bread, or a bag, or pence in their purse; 
 \V{9}  but they were to wear sandals, and not to put on a second coat. 
 \V{10}  “Whenever you go to stay at a house,”  he said,   “remain there till you leave that place; 
 \V{11}  And if a place does not welcome you, or listen to you, as you go out of it shake off the dust that is on the soles of your feet, as a protest against them.” 
 \V{12}  So they set out, and proclaimed the need of repentance. 
 \V{13}  They drove out many demons, and anointed with oil many who were infirm, and cured them. \indenting[yes]\par 
 \V{14}  Now King Herod heard of Jesus; for his name had become well known. People were saying — ”John the Baptizer must have risen from the dead, and that is why these miraculous powers are active in him.” 
 \V{15}  Others again said — “He is Elijah,” and others — “He is a Prophet, like one of the great Prophets.” 
 \V{16}  But when Herod heard of him, he said — “The man whom I beheaded — John — he must be risen!” \indenting[yes]\par 
 \V{17}  For Herod himself had sent and arrested John, and put him in prison, in chains, to please Herodias, the wife of his brother Philip, because Herod had married her. 
 \V{18}  For John had said to Herod — “You have no right to be living with your brother’s wife.” 
 \V{19}  So Herodias was incensed against John, and wanted to put him to death, but was unable to do so, 
 \V{20}  because Herod stood in fear of John, knowing him to be an upright and holy man, and protected him. He had listened to John, but still remained much perplexed, and yet he found pleasure in listening to him. \indenting[yes]\par 
 \V{21}  A suitable opportunity, however, occurred when Herod, on his birthday, gave a dinner to his high officials, and his generals, and the foremost men in Galilee. 
 \V{22}  When his daughter — that is, the daughter of Herodias — came in and danced, she delighted Herod and those who were dining with him. “Ask me for whatever you like,” the King said to the girl, “and I will give it to you”; 
 \V{23}  and he swore to her that he would give her whatever she asked him — up to half his kingdom. 
 \V{24}  The girl went out, and said to her mother “What must I ask for?” “The head of John the Baptizer,’ answered her mother. 
 \V{25}  So she went in as quickly as possible to the King, and made her request. “I want you,” she said, “to give me at once, on a dish, the head of John the Baptist.” 
 \V{26}  The King was much distressed; yet, on account of his oath and of the guests at his table, he did not like to refuse her. 
 \V{27}  He immediately dispatched one of his bodyguard, with orders to bring John’s head. The man went and beheaded John in the prison, 
 \V{28}  and, bringing his head on a dish, gave it to the girl, and the girl gave it to her mother. \indenting[yes]\par 
 \V{29}  When John’s disciples heard of it, they came and took his body away, and laid it in a tomb. \indenting[yes]\par 
 \V{30}  When the Apostles came back to Jesus, they told him all that they had done and all that they had taught. 
 \V{31}  “Come by yourselves privately to some lonely spot,”  he said,   ”and rest for a while”  — for there were so many people coming and going that they had not time even to eat. 
 \V{32}  So they set off privately in their boat for a lonely spot. 
 \V{33}  Many people saw them going, and recognized them, and from all the towns they flocked together to the place on foot, and got there before them. 
 \V{34}  On getting out of the boat, Jesus saw a great crowd, and his heart was moved at the sight of them, because they were ‘like sheep without a shepherd’; and he began to teach them many things. 
 \V{35}  When it grew late, his disciples came up to him, and said: ”This is a lonely spot, and it is already late. 
 \V{36}  Send the people away, so that they may go to the farms and villages around and buy themselves something to eat.” 
 \V{37}  But Jesus answered:   “It is for you to give them something to eat.”  ”Are we to go and buy twenty pounds’ worth of bread,” they asked, ”to give them to eat?” 
 \V{38}  “How many loaves have you?”  he asked;   ”Go, and see.”  When they had found out, they told him: ”Five, and two fishes.” 
 \V{39}  Jesus directed them to make all the people take their seats on the green grass, in parties; 
 \V{40}  and they sat down in groups — in hundreds, and in fifties. 
 \V{41}  Taking the five loaves and the two fishes, Jesus looked up to Heaven, and said the blessing; he broke the loaves into pieces, and gave them to his disciples for them to serve out to the people, and he divided the two fishes also among them all. 
 \V{42}  Every one had sufficient to eat; 
 \V{43}  and they picked up enough broken pieces to fill twelve baskets, as well as some of the fish. 
 \V{44}  The people who ate the bread were five thousand in number. \indenting[yes]\par 
 \V{45}  Immediately afterwards Jesus made his disciples get into the boat, and cross over in advance, in the direction of Bethsaida, while he himself was dismissing the crowd. 
 \V{46}  After he had taken leave of the people, he went away up the hill to pray. 
 \V{47}  When evening fell, the boat was out in the middle of the Sea, and Jesus on the shore alone. 
 \V{48}  Seeing them laboring at the oars — for the wind was against them — about three hours after midnight Jesus came towards them, walking on the water, intending to join them. 
 \V{49}  But, when they saw him walking on the water, they thought it was a ghost, and cried out; 
 \V{50}  for all of them saw him, and were terrified. But Jesus at once spoke to them.   “Courage!”  he said,   “it is I; do not be afraid!” 
 \V{51}  Then he got into the boat with them, and the wind dropped. The disciples were utterly amazed, 
 \V{52}  for they had not understood about the loaves, their minds being slow to learn. 
 \V{53}  When they had crossed over, they landed at Gennesaret, and moored the boat. 
 \V{54}  But they had no sooner left her than the people, recognizing Jesus, 
 \V{55}  hurried over the whole country-side, and began to carry about upon mats those who were ill, wherever they heard he was. 
 \V{56}  So wherever he went — to villages, or towns, or farms — they would lay their sick in the market-places, begging him to let them touch only the tassel of his cloak; and all who touched were made well. \blank\indenting[no]\par 
 \C{7}  \CapStretch{One day the Pharisees }and some of the Teachers of the Law who had come from Jerusalem gathered round Jesus. 
 \V{2}  They had noticed that some of his disciples ate their food with their hands ‘defiled,’ by which they meant unwashed. 
 \V{3}  (For the Pharisees, and indeed all strict Jews, will not eat without first scrupulously washing their hands, holding in this to the traditions of their ancestors. 
 \V{4}  When they come from market, they will not eat without first sprinkling themselves; and there are many other customs which they have inherited and hold to, such as the ceremonial washing of cups, and jugs, and copper pans). 
 \V{5}  So the Pharisees and the Teachers of the Law asked Jesus this question — “How is it that your disciples do not follow the traditions of our ancestors, but eat their food with defiled hands?” 
 \V{6}  His answer was:   “It was well said by Isaiah when he prophesied about you hypocrites in the words — ‘This is a people that honor me with their lips, while their hearts are far removed from me; 
 \V{7}  but vainly do they worship me, For they teach but human precepts.’ 
 \V{8}  You neglect God’s commandments and hold to human traditions. 
 \V{9}  Wisely do you set aside God’s commandments,”  he exclaimed,   “to keep your own traditions! 
 \V{10}  For while Moses said ‘Honor your father and your mother,’ and ‘Let anyone who abuses their father or mother suffer death,’ 
 \V{11}  you say ‘If a person says to their father or mother “Whatever of mine might have been of service to you is Korban”’  (which means ‘Given to God’) — 
 \V{12}  why, then you do not allow them to do anything further for their father or mother! 
 \V{13}  In this way you nullify the words of God by your traditions, which you hand down; and you do many similar things.” \indenting[yes]\par 
 \V{14}  Then Jesus called the people to him again, and said:   “Listen to me, all of you, and mark my words. 
 \V{15}  There is nothing external to a person, which by going into them can ‘defile’ them; but the things that come out from a person are the things that defile them.” \indenting[yes]\par 
 \V{17}  When Jesus went indoors, away from the crowd, his disciples began questioning him about this saying. 
 \V{18}  “What, do even you understand so little?”  exclaimed Jesus.   “Do not you see that there is nothing external to a person, which by going into a person, can ‘defile’ them, 
 \V{19}  because it does not pass into his heart, but into his stomach, and is afterwards got rid of? — in saying this Jesus pronounced all food ‘clean.’ 
 \V{20}  “It is what comes out from a person,”  he added,   “that defiles them, 
 \V{21}  for it is from within, out of the hearts of people, that there come evil thoughts — unchastity, theft, murder, adultery, 
 \V{22}  greed, wickedness, deceit, indencency, envy, slander, haughtiness, folly; 
 \V{23}  all these wicked things come from within, and do defile a person.” \indenting[yes]\par 
 \V{24}  On leaving that place, Jesus went to the district of Tyre and Sidon. He went into a house, and did not wish anyone to know it, but could not escape notice. 
 \V{25}  For a woman, whose little daughter had a foul spirit in her, heard of him immediately, and came and threw herself at his feet —  
 \V{26}  the woman was a foreigner, a native of Syrian Phoenicia — and she begged him to drive the demon out of her daughter. 
 \V{27}  “Let the children be satisfied first,”  answered Jesus.   “For it is not fair to take the children’s food, and throw it to dogs.” 
 \V{28}  “Yes, Master,” she replied; “even the dogs under the table do feed on the children’s crumbs.” 
 \V{29}  “For saying that,”  he answered,   “you may go. The demon has gone out of your daughter.” 
 \V{30}  The woman went home, and found the child lying on her bed, and the demon gone. \indenting[yes]\par 
 \V{31}  On returning from the district of Tyre, Jesus went, by way of Sidon, to the Sea of Galilee, across the district of the Ten Towns. 
 \V{32}  Some people brought to him a man who was deaf and almost dumb, and they begged Jesus to place his hand on him. 
 \V{33}  Jesus took him aside from the crowd quietly, put his fingers into the man’s ears, and touched his tongue with saliva. 
 \V{34}  Then, looking up to Heaven, he sighed, and said to the man:   “Ephphatha!”  which means   ‘Be opened.’ 
 \V{35}  The man’s ears were opened, the string of his tongue was freed, and he began to talk plainly. 
 \V{36}  Jesus insisted upon their not telling any one; but the more he insisted, the more perseveringly they made it known, 
 \V{37}  and a profound impression was made upon the people. “He has done everything well!” they exclaimed. “He makes even the deaf hear and the dumb speak!” \blank\indenting[no]\par 
 \C{8}  \CapStretch{About that time}, when there was again a great crowd of people who had nothing to eat, Jesus called his disciples to him, and said: 
 \V{2}  “My heart is moved at the sight of all these people, for they have already been with me three days and they have nothing to eat; 
 \V{3}  and if I send them away to their homes hungry, they will break down on the way; and some of them have come a long distance.” 
 \V{4}  “Where will it be possible,” his disciples answered, “to get sufficient bread for these people in this lonely place?” 
 \V{5}  “How many loaves have you?”  he asked. “Seven,” they answered. 
 \V{6}  Jesus told the crowd to sit down upon the ground. Then he took the seven loaves, and, after saying the thanksgiving, broke them, and gave them to his disciples to serve out; and they served them out to the crowd. 
 \V{7}  They had also a few small fish; and, after he had said the blessing, he told the disciples to serve out these as well. 
 \V{8}  The people had sufficient to eat, and they picked up seven baskets full of the broken pieces that were left. 
 \V{9}  There were about four thousand people. Then Jesus dismissed them. 
 \V{10}  Immediately afterwards, getting into the boat with his disciples, Jesus went to the district of Dalmanutha. \indenting[yes]\par 
 \V{11}  Here the Pharisees came out, and began to argue with Jesus, asking him for some sign from the heavens, to test him. 
 \V{12}  Sighing deeply, Jesus said:   “Why does this generation ask for a sign? I tell you, no sign shall be given it.” 
 \V{13}  So he left them to themselves, and, getting into the boat again, went away to the opposite shore. \indenting[yes]\par 
 \V{14}  Now the disciples had forgotten to take any bread with them, one loaf being all that they had in the boat. 
 \V{15}  So Jesus gave them this warning.   “Take care,”  he said,   “beware of the leaven of the Pharisees and the leaven of Herod.” 
 \V{16}  They began talking to one another about their being short of bread; 
 \V{17}  and, noticing this, Jesus said to them:   “Why are you talking about your being short of bread? Do not you yet see or understand? Are your minds still so slow or comprehension? 
 \V{18}  “Though you have eyes, do you not see? and though you have ears, do you not hear?’ Do not you remember, 
 \V{19}  when I broke up the five loaves for the five thousand, how many baskets of broken pieces you picked up?”  “Twelve,” they said. 
 \V{20}  “And when the seven for the four thousand, how many basketfuls of broken pieces did you pick up?”  “Seven,” they said. 
 \V{21}  “Do not you understand now?”  he repeated. \indenting[yes]\par 
 \V{22}  They came to Bethsaida. There some people brought a blind man to Jesus, and begged him to touch him. 
 \V{23}  Taking the blind man’s hand, Jesus led him to the outskirts of the village, and, when he had put saliva on the man’s eyes, he placed his hands on him, and asked him:   “Do you see anything?” 
 \V{24}  The man looked up, and said: “I see the people, for, as they walk about, they look to me like trees.” 
 \V{25}  Then Jesus again placed his hands on the man’s eyes; and the man saw clearly, his sight was restored, and he saw everything with perfect distinctness. 
 \V{26}  Jesus sent him to his home, and said:   “Do not go even into the village.” \blank\indenting[no]\par 
 \V{27}  \CapStretch{Afterwards Jesus }and his disciples went into the villages round Caesarea Philippi; and on the way he asked his disciples this question —   “Who do people say that I am?” 
 \V{28}  “John the Baptist,” they answered, “but others say Elijah, while others say one of the Prophets.” 
 \V{29}  “But you,”  he asked,   “who do you say that I am?”  To this Peter replied: “You are the Christ.” 
 \V{30}  On which Jesus charged them not to say this about him to anyone. 
 \V{31}  Then he began to teach them that the Son of Man must undergo much suffering, and that he must be rejected by the Councillors, and the Chief Priests, and the Teachers of the Law, and be put to death, and rise again after three days. 
 \V{32}  This statement he made openly. But Peter took Jesus aside, and began to rebuke him. 
 \V{33}  Jesus, however, turning round and seeing his disciples, rebuked Peter.   “Out of my sight, Satan!”  he exclaimed.   “For you look at things, not as God does, but as people do.” \indenting[yes]\par 
 \V{34}  Calling the people and his disciples to him, Jesus said:   “If anyone wishes to walk in my steps, let them renounce self, take up their cross, and follow me. 
 \V{35}  For whoever wishes to save their life will lose it, and whoever, for my sake and for the sake of the Good News, will lose their life shall save it. 
 \V{36}  What good is it to a person to gain the whole world and forfeit their life? 
 \V{37}  For what could a person give that is of equal value with their life? 
 \V{38}  Whoever is ashamed of me and of my teaching, in this unfaithful and wicked generation, of them will the Son of Man be ashamed, when he comes in his Father’s Glory with the holy angels.” \indenting[yes]\par 
 \C{9}  “I tell you,”  he added,   “that some of those who are standing here will not know death till they have seen the Kingdom of God come in power.” \indenting[yes]\par 
 \V{2}  Six days later, Jesus took with him Peter, James, and John, and led them up a high mountain alone by themselves. There his appearance was transformed before their eyes, 
 \V{3}  and his clothes became of a more dazzling white than any bleacher in the world could make them. 
 \V{4}  And Elijah appeared to them, in company with Moses; and they were talking with Jesus. 
 \V{5}  “Rabbi,” said Peter, interposing, “it is good to be here; let us make three tents, one for you, one for Moses, and one for Elijah.” 
 \V{6}  For he did not know what to say, because they were much afraid. 
 \V{7}  Then a cloud came down and enveloped them; and from the cloud there came a voice — “This is my Son, the Beloved; him you must hear.” 
 \V{8}  And suddenly, on looking round, they saw that there was now no one with them but Jesus alone. \indenting[yes]\par 
 \V{9}  As they were going down the mountain-side, Jesus cautioned them not to relate what they had seen to any one, till after the Son of Man should have risen again from the dead. 
 \V{10}  They seized upon these words and discussed with one another what this ‘rising from the dead’ meant. 
 \V{11}  “How is it,” they asked Jesus, “that our Teachers of the Law say that Elijah has to come first?” 
 \V{12}  “Elijah does indeed come first,”  answered Jesus,   “and re-establish everything; and does not Scripture speak, with regard to the Son of Man, of his undergoing much suffering and being utterly despised? 
 \V{13}  But I tell you that Elijah has come, and people have treated him just as they pleased, as Scripture says of him.” \indenting[yes]\par 
 \V{14}  When they came to the other disciples, they saw a great crowd round them, and some Teachers of the Law arguing with them. 
 \V{15}  But, as soon as they saw Jesus, all the people, in great astonishment, ran up and greeted him. 
 \V{16}  “What are you arguing about with them?”  Jesus asked. 
 \V{17}  “Teacher,” answered a man in the crowd, “I brought my son to see you, as he has a dumb spirit in him; 
 \V{18}  and, wherever it seizes him, it dashes him down; he foams at the mouth and grinds his teeth, and he is pining away. I asked your disciples to drive the spirit out, but they failed.” 
 \V{19}  “O faithless generation!”  exclaimed Jesus.   “How long must I be with you? how long must I have patience with you? Bring the boy to me.” 
 \V{20}  They brought him to Jesus; but no sooner did the boy see him than the spirit threw him into convulsions; and he fell on the ground, and rolled about, foaming at the mouth. 
 \V{21}  “How long has he been like this?”  Jesus asked the boy’s father. 
 \V{22}  “From his childhood,” he answered; “and it has often thrown him into fire and into water to put an end to his life; but, if you can possibly do anything, take pity on us, and help us!” 
 \V{23}  Why say ‘possibly’?”  Jesus replied.   “Everything is possible for one who has faith.” 
 \V{24}  The boy’s father immediately cried out: “I have faith; help my want of faith!” 
 \V{25}  But, when Jesus saw that a crowd was quickly collecting, he rebuked the foul spirit:   “Deaf and dumb spirit, it is I who command you. Come out from him and never enter him again.” 
 \V{26}  With a loud cry the spirit threw the boy into repeated convulsions, and then came out from him. The boy looked like a corpse, so that most of them said that he was dead. 
 \V{27}  But Jesus took his hand, and lifted him; and he stood up. \indenting[yes]\par 
 \V{28}  When Jesus had gone indoors, his disciples asked him privately: “Why could not we drive it out?” 
 \V{29}  “A spirit of this kind,”  he said,   “can be driven out only by prayer.” \blank\indenting[no]\par 
 \V{30}  \CapStretch{Leaving that place}, Jesus and his disciples went on their way through Galilee; but he did not wish any one to know it, 
 \V{31}  for he was instructing his disciples, and telling them —   “The Son of Man is being betrayed into the hands of his fellow men, and they will put him to death, but, when he has been put to death, he will rise again after three days.” 
 \V{32}  But the disciples did not understand his meaning and were afraid to question him. \indenting[yes]\par 
 \V{33}  They came to Capernaum. When Jesus had gone into the house, he asked them:   “What were you discussing on the way?” 
 \V{34}  But they were silent; for on the way they had been arguing with one another which was the greatest. 
 \V{35}  Sitting down, Jesus called the Twelve and said:   “If any one wishes to be first, he must be last of all, and servant of all.” 
 \V{36}  Then Jesus took a little child, and placed it in the middle of them. Folding it in his arms, he said to them: 
 \V{37}  “Any one who, for the sake of my Name, welcomes even a little child like this is welcoming me, and any one who welcomes me is welcoming not me, but him who sent me as his Messenger.” \indenting[yes]\par 
 \V{38}  “Teacher,” said John, “we saw a man driving out demons by using your name, and we tried to prevent him, because he did not follow us.” 
 \V{39}  “None of you must prevent the man,”  answered Jesus,   “for no one will use my name in working a miracle, and yet find it easy to speak evil of me.  He who is not against us is for us. 
 \V{40, 41}  If any one gives you a cup of water because you belong to Christ, I tell you, he shall assuredly not lose his reward. \indenting[yes]\par 
 \V{42}  ‘And, if any one puts a snare in the way of one of these lowly ones who believe in me, it would be far better for him if he had been thrown into the sea with a great millstone round his neck. 
 \V{43}  If your hand proves a snare to you, cut it off. It would be better for you to enter the Life maimed, than to have both your hands and go into Gehenna, into the inextinguishable fire. 
 \V{45}  If your foot proves a snare to you, cut it off. It would be better for you to enter the Life lame, than to have both your feet and be thrown into Gehenna. 
 \V{47}  If your eye proves a snare to you, tear it out. It would be better for you to enter the Kingdom of God with only one eye, than to have both eyes and be thrown into Gehenna, 
 \V{48}  where ‘their worm does not die, and the fire is not put out.’ \indenting[yes]\par 
 \V{49}  ‘For it is by fire that every one will be salted. \indenting[yes]\par 
 \V{50}  ‘Salt is good, but, if the salt should lose its saltiness, what will you use to season it?  \indenting[yes]\par  ‘You must have salt in yourselves, and live at peace with one another.” \MS{The Journey to Jerusalem} \indenting[yes]\par 
 \C{10}  On leaving that place, Jesus went into the district of Judea on the other side of the Jordan. Crowds gathered about him again; and again, as usual, he began teaching them. 
 \V{2}  Presently some Pharisees came up and, to test him, asked: “Has a husband the right to divorce his wife?” 
 \V{3}  “What direction did Moses give you?”  replied Jesus. 
 \V{4}  “Moses,” they said, “permitted a man to ‘draw up in writing a notice of separation and divorce his wife.’”  “It was owing to the hardness of your hearts,”  said Jesus,   “that Moses gave you this direction; 
 \V{5, 6}  but, at the beginning of the Creation, God ‘made them male and female.’ 
 \V{7}  ‘For this reason a man shall leave his father and mother, 
 \V{8}  and the man and his wife shall become one;’ so that they are no longer two, but one. 
 \V{9}  What God himself, then, has yoked together no one must separate.” \indenting[yes]\par 
 \V{10}  When they were indoors, the disciples asked him again about this, 
 \V{11}  and he said:   “Any one who divorces his wife and marries another woman is guilty of adultery against his wife; 
 \V{12}  and, if the woman divorces her husband and marries another man, she is guilty of adultery.” \indenting[yes]\par 
 \V{13}  Some of the people were bringing little children to Jesus, for him to touch them; but the disciples found fault with those who had brought them. 
 \V{14}  When, however, Jesus saw this, he was indignant.   “Let the little children come to me,”  he said,   “do not hinder them; for it is to the childlike that the Kingdom of God belongs. 
 \V{15}  I tell you, unless a person receives the Kingdom of God like a child, they will not enter it at all.” 
 \V{16}  Then he folded the children in his arms, and, placing his hands on them, gave them his blessing. \indenting[yes]\par 
 \V{17}  As Jesus was resuming his journey, a man came running up to him, and threw himself on his knees before him. “Good Teacher,” he asked, “what must I do to gain eternal life?”  “Why do you call me good?”  answered Jesus.   “No one is good but God. 
 \V{18, 19}  You know the commandments — ‘Do not kill. Do not commit adultery. Do not steal. Do not say what is false about others. Do not cheat. Honor your father and your mother.’” 
 \V{20}  “Teacher,” he replied, “I have observed all these from my childhood.” 
 \V{21}  Jesus looked at the man, and his heart went out to him, and he said:   “There is still one thing wanting in you; go and sell all that you have, and give to the poor, and you shall heave wealth in Heaven; then come and follow me.” 
 \V{22}  But the man’s face clouded at these words, and he went away distressed, for he had great possessions. \indenting[yes]\par 
 \V{23}  Then Jesus looked round, and said to his disciples:   “How hard it will be for people of wealth to enter the Kingdom of God!” 
 \V{24}  The disciples were amazed at his words. But Jesus said again:   “My children, how hard a thing it is to enter the Kingdom of God! 
 \V{25}  It is easier for a camel to get through a needle’s eye, than for a rich person to enter the Kingdom of God.” 
 \V{26}  “Then who can be saved?” they exclaimed in the greatest astonishment. 
 \V{27}  Jesus looked at them, and answered:   “With people it is impossible, but not with God; for everything is possible with God.” \indenting[yes]\par 
 \V{28}  “But we,” began Peter, “we left everything and have followed you.” 
 \V{29}  “I tell you,”  said Jesus,   “there is no one who has left house, or brothers, or sisters, or mother, or father, or children, or land, on my account and on account of the Good News, 
 \V{30}  who will not receive a hundred times as much, even now in the present — houses, and brothers, and sisters, and mothers, and children, and land — though not without persecutions, and, in the age that is coming, eternal life. 
 \V{31}  But many who are first now will then be last, and the last will be first.” \blank\indenting[no]\par 
 \V{32}  \CapStretch{One day}, when they were on their way, going up to Jerusalem, Jesus was walking in front of the Apostles, who were filled with misgivings; while those who were following behind were alarmed. Gathering the Twelve round him once more, Jesus began to tell them what was about to happen to him. 
 \V{33}  “Listen!”  he said.   “We are going up to Jerusalem; and there the Son of Man will be betrayed to the Chief Priests and the Teachers of the Law, and they will condemn him to death, and they will give him up to the Gentiles, 
 \V{34}  who will mock him, spit upon him, and scourge him, and put him to death; and after three days he will rise again.” \indenting[yes]\par 
 \V{35}  James and John, the two sons of Zebediah, went to Jesus, and said: “Teacher, we want you to do for us whatever we ask.” 
 \V{36}  “What do you want me to do for you?”  he asked. 
 \V{37}  “Grant us this,” they answered, “to sit, one on your right, and the other on your left, when you come in glory.” 
 \V{38}  “You do not know what you are asking,”  Jesus said to them.   “Can you drink the cup that I am to drink? or receive the baptism that I am to receive?” 
 \V{39}  “Yes,” they answered, “we can.”   “You shall indeed drink the cup that I am to drink,”  Jesus said,   “and receive the baptism that I am to receive, 
 \V{40}  but as to a seat at my right or at my left — that is not mine to give, but it is for those for whom it has been prepared.” \indenting[yes]\par 
 \V{41}  On hearing of this, the ten others were at first very indignant about James and John. 
 \V{42}  But Jesus called the ten to him, and said:   “Those who are regarded as ruling among the Gentiles lord it over them, as you know, and their great men oppress them. 
 \V{43}  But among you it is not so. No, whoever wants to become great among you must be your servant, 
 \V{44}  and whoever wants to take the first place among you must be the servant of all; 
 \V{45}  for even the Son of Man came, not be served, but to serve, and to give his life as a ransom for many.” \indenting[yes]\par 
 \V{46}  They came to Jericho. When Jesus was going out of the town with his disciples and a large crowd, Bartimaeus, the son of Timaeus, a blind beggar, was sitting by the road-side. 
 \V{47}  Hearing that it was Jesus of Nazareth, he began to call out: “Jesus, Son of David, take pity on me.” 
 \V{48}  Many of the people kept telling him to be quiet; but the man continued to call out all the louder: “Son of David, take pity on me.” 
 \V{49}  Then Jesus stopped.   “Call him,”  he said. So they called the blind man. “Courage!” they exclaimed. “Get up; he is calling you.” 
 \V{50}  The man threw off his cloak, sprang up, and came to Jesus. 
 \V{51}  “What do you want me to do for you?”  said Jesus, addressing him. “Rabboni,” the blind man answered, “I want to recover my sight.” 
 \V{52}  “You may go,”  Jesus said;   “your faith has delivered you.”  Immediately he recovered his sight, and began to follow Jesus along the road. \MS{The Last Days} \indenting[yes]\par 
 \C{11}  When they had almost reached Jerusalem, as far as Bethphage and Bethany, near the Mount of Olives, Jesus sent on two of his disciples. 
 \V{2}  “Go to the village facing you,”  he said;   “and, as soon as you get there, you will find a foal tethered, which no one has ever ridden; untie it, and bring it. 
 \V{3}  And, if any one says to you ‘Why are you doing that?’, say ‘The Master wants it, and will be sure to send it back here at once.’” 
 \V{4}  The two disciples went, and, finding a foal tethered outside a door in the street, they untied it. 
 \V{5}  Some of the by-standers said to them: “What are you doing, untying the foal?” 
 \V{6}  And the two disciples answered as Jesus had told them; and they allowed them to go. 
 \V{7}  Then they brought the foal to Jesus, and, when they had laid their cloaks on it, he seated himself upon it. 
 \V{8}  Many of the people spread their cloaks on the road, while some strewed boughs which they had cut from the fields; 
 \V{9}  and those who led the way, as well as those who followed, kept shouting: “‘God save him! Blessed is He who comes in the name of the Lord!’ 
 \V{10}  Blessed is the coming Kingdom of our father David! ‘God save him from on high!’” \indenting[yes]\par 
 \V{11}  Jesus entered Jerusalem, and went into the Temple Courts; and, after looking round at everything, as it was already late, he went out to Bethany with the Twelve. \indenting[yes]\par 
 \V{12}  The next day, after they had left Bethany, Jesus became hungry; 
 \V{13}  and, noticing a fig-tree at a distance in leaf, he went to it to see if by any chance he could find something on it; but, on coming up to it, he found nothing but leaves, for it was not the season for figs. 
 \V{14}  So, addressing the tree, he exclaimed:   “May no one ever again eat of your fruit!”  And his disciples heard what he said. \indenting[yes]\par 
 \V{15}  They came to Jerusalem. Jesus went into the Temple Courts, and began to drive out those who were buying and selling there. He overturned the tables of the money-changers, and the seats of the pigeon-dealers, 
 \V{16}  and would not allow any one to carry anything across the Temple Courts. 
 \V{17}  Then he began to teach.   “Does not Scripture say,”  he asked,   “‘My House shall be called a House of Prayer for all the nations’? But you have made it a den of robbers.’” 
 \V{18}  Now the Chief Priests and the Teachers of the Law heard this and began to look for some way of putting Jesus to death; for they were afraid of him, since all the people were greatly impressed by his teaching. 
 \V{19}  As soon as evening fell, Jesus and his disciples went out of the city. \indenting[yes]\par 
 \V{20}  As they passed by early in the morning, they noticed that the fig-tree was withered up from the very roots. 
 \V{21}  Then Peter remembered what had occurred. “Look, Rabbi,” he exclaimed, “the fig-tree which you doomed is withered up!” 
 \V{22}  “Have faith in God!”  replied Jesus. 
 \V{23}  “I tell you that if any one should say to this hill ‘Be lifted up and hurled into the sea!’, without ever a doubt in his mind, but in the faith that what he says will be done, he would find that it would be. 
 \V{24}  And therefore I say to you ‘Have faith that whatever you ask for in prayer is already granted you, and you will find that it will be.’ \indenting[yes]\par 
 \V{25}  ‘And, whenever you stand up to pray, forgive any grievance that you have against any one, that your Father who is in Heaven also may forgive you your offenses.” \blank\indenting[no]\par 
 \V{27}  \CapStretch{They came to Jerusalem again. While Jesus} was walking about in the Temple Courts, the Chief Priests, the Teachers of the Law, and the Councillors came up to him. 
 \V{28}  “What authority have you to do these things?” they said. “Who gave you the authority to do them?” 
 \V{29}  “I will put one question to you,”  said Jesus.   “Answer me that, and then I will tell you what authority I have to act as I do. 
 \V{30}  It is about John’s baptism. Was it of divine or human origin? Answer me that.” 
 \V{31}  They began arguing together; “If we say ‘divine,’ he will say ‘Why then did not you believe him?’ 
 \V{32}  Yet can we say ‘human’?” — They were afraid of the people, for everyone regarded John as undoubtedly a Prophet. 
 \V{33}  So their answer to Jesus was — “We do not know.”   “Then I,”  replied Jesus,   “refuse to tell you what authority I have to do these things.” \indenting[yes]\par 
 \C{12}  Jesus began to speak to them in parables:   “A man once planted a vineyard, put a fence round it, dug a wine-press, built a tower, and then let it out to tenants and went abroad. 
 \V{2}  At the proper time he sent a servant to the tenants, to receive from them a share of the produce of the vintage; 
 \V{3}  but they seized him, and beat him, and sent him away empty- handed. 
 \V{4}  A second time the owner sent a servant to them; this man, too, the tenants struck on the head, and insulted. 
 \V{5}  He sent another, but him they killed; and so with many others- -some they beat and some they killed. 
 \V{6}  He had still one son, who was very dear to him; and him he sent to them last of all. ‘They will respect my son,’ he said. 
 \V{7}  But those tenants said to one another ‘Here is the heir! Come, let us kill him, and his inheritance will be ours.’ 
 \V{8}  So they seized him, and killed him, and threw his body outside the vineyard. 
 \V{9}  What will the owner of the vineyard do? He will come and put the tenants to death, and he will let the vineyard to others. \indenting[yes]\par 
 \V{10}  ‘Have you never read this passage of Scripture? — ‘The very stone which the builders despised has now itself become the corner-stone; 
 \V{11}  this corner-stone has come from the Lord, and is marvelous in our eyes.’” \indenting[yes]\par 
 \V{12}  After this his enemies were eager to arrest him, but they were afraid of the crowd; for they saw that it was at them that he had aimed the parable. So they let him alone, and went away. \blank\indenting[no]\par 
 \V{13}  \CapStretch{Afterwards they sent to Jesus some of the} Pharisees and of the Herodians, to set a trap for him in the course of conversation. 
 \V{14}  These men came to him and said: “Teacher, we know that you are an honest man, and are not afraid of any one, for you pay no regard to a person’s position, but teach the Way of God honestly; are we right in paying taxes to the Emperor, or not? 
 \V{15}  Should we pay, or should we not pay?” Knowing their hypocrisy, Jesus said to them:   “Why are you testing me? Bring me a florin to look at.” 
 \V{16}  And, when they had brought it, he asked:   “Whose head and title are these?”  “The Emperor’s,” they said; 
 \V{17}  and Jesus replied:   “Pay to the Emperor what belongs to the Emperor, and to God what belongs to God.”  And they wondered at him. \indenting[yes]\par 
 \V{18}  Next came some Sadducees — the men who maintain that there is no resurrection. Their question was this —  
 \V{19}  “Teacher, in our Scriptures Moses decreed that, should a man’s brother die, leaving a widow but no child, the man should take the widow as his wife, and raise up a family for his brother. 
 \V{20}  There were once seven brothers; of whom the eldest took a wife, but died and left no family; 
 \V{21}  and the second took her, and died without family; and so did the third. 
 \V{22}  All the seven died and left no family. The woman herself died last of all. 
 \V{23}  At the resurrection whose wife will she be, all seven brothers having had her as their wife?” 
 \V{24}  “Is not the reason of your mistake,”  answered Jesus,   “your ignorance of the Scriptures and of the power of God? 
 \V{25}  When people rise from the dead, there is no marrying or being married; but they are as angels in Heaven. \indenting[yes]\par 
 \V{26}  “As to the dead, and the fact that they rise, have you never read in the Book of Moses, in the passage about the Bush, how God said to him — ‘I am the God of Abraham, and the God of Isaac, and the God of Jacob’? 
 \V{27}  He is not God of dead people, but of living. You are greatly mistaken.” \indenting[yes]\par 
 \V{28}  Then came up one of the Teachers of the Law who had heard their discussions. Knowing that Jesus had answered them wisely, he asked him this question: “What is the first of all the commandments?” 
 \V{29}  “The first,”  answered Jesus,   “is — ‘Hear, O Israel; the Lord our God is the one Lord; 
 \V{30}  and you must love the Lord your God with all your heart, and with all your soul, and with all your mind, and with all your strength.’ 
 \V{31}  The second is this — ‘You must love your neighbor as you love yourself.’ There is no commandment greater than these.” 
 \V{32}  “Wisely answered, Teacher!” exclaimed the Teacher of the Law. “It is true, as you say, that ‘there is one God,’ and that ‘there is no other besides him’; 
 \V{33}  and to ‘love him with all one’s heart, and with all one’s understanding, and with all one’s strength,’ and to ‘love one’s neighbor as one loves oneself’ is far beyond all ‘burnt-offerings and sacrifices.’” 
 \V{34}  Seeing that he had answered with discernment, Jesus said to him:   “You are not far from the Kingdom of God.” \indenting[yes]\par  After that no one ventured to question him further. \indenting[yes]\par 
 \V{35}  While Jesus was teaching in the Temple Courts, he asked:   “How is it that the Teachers of the Law say that the Christ is to be David’s son? 
 \V{36}  David said himself, speaking under the inspiration of the Holy Spirit —  ‘The Lord said to my lord: “Sit at my right hand, until I put your enemies beneath your feet.’” 
 \V{37}  David himself calls him ‘lord,’ how comes it, then, that he is to be his son?” \indenting[yes]\par  The mass of the people listened to Jesus with delight. 
 \V{38}  In the course of his teaching, Jesus said:   “See that you are on your guard against the Teachers of the Law, who delight to walk about in long robes, and to be greeted in the streets with respect, 
 \V{39}  and to have the best seats in the Synagogues, and places of honor at dinner. 
 \V{40}  They are the men that rob widows of their homes, and make a pretense of saying long prayers. Their sentence will be all the heavier.” \indenting[yes]\par 
 \V{41}  Then Jesus sat down opposite the chests for the Temple offerings, and watched how the people put money into them. Many rich people were putting in large sums; 
 \V{42}  but one poor widow came and put in two small coins, worth very little. 
 \V{43}  On this, calling his disciples to him, Jesus said:   “I tell you that this poor widow has put in more than all the others who were putting money into the chests; 
 \V{44}  for every one else put in something from what he had to spare, while she, in her need, put in all she had — everything that she had to live on.” \blank\indenting[no]\par 
 \C{13}  \CapStretch{As Jesus was walking out of the Temple Courts,} one of his disciples said to him: “Teacher, look what fine stones and buildings these are!” 
 \V{2}  “Do you see these great buildings?”  asked Jesus.   “Not a single stone will be left here upon another, which shall not be thrown down.” \indenting[yes]\par 
 \V{3}  When Jesus had sat down on the Mount of Olives, facing the Temple, Peter, James, John and Andrew questioned him privately: 
 \V{4}  “Tell us when this will be, and what will be the sign when all this is drawing to its close.” \indenting[yes]\par 
 \V{5}  Then Jesus began:   “See that no one leads you astray. 
 \V{6}  Many will take my name, and come saying ‘I am He’, and will lead many astray. \indenting[yes]\par 
 \V{7}  “And, when you hear of wars and rumors of wars, do not be alarmed; such things must occur; but the end is not yet. 
 \V{8}  For ‘nation will rise against nation, and kingdom against kingdom’; there will be earthquakes in various places; there will be famines. This will be but the beginning of the birth-pangs. \indenting[yes]\par 
 \V{9}  “See to yourselves! They will betray you to courts of law; and you will be taken to Synagogues and beaten; and you will be brought up before governors and kings for my sake, that you may bear witness before them. 
 \V{10}  But the Good News must first be proclaimed to every nation. 
 \V{11}  Whenever they betray you and hand you over for trial, do not be anxious beforehand as to what you shall say, but say whatever is given you at the moment; for it will not be you who speak, but the Holy Spirit. 
 \V{12}  Brother will betray brother to death, and the father his child; and children will turn against their parents, and cause them to be put to death; 
 \V{13}  and you will be hated by every one on account of my Name. Yet the person that endures to the end shall be saved. \indenting[yes]\par 
 \V{14}  “As soon, however, as you see ‘the Foul Desecration’ standing where he ought not”  (the reader must consider what this means)   “then those of you who are in Judea must take refuge in the mountains; 
 \V{15}  and a person on the house-top must not go down, or go in to get anything out of their house: 
 \V{16}  nor must one who is on their farm turn back to get their cloak. 
 \V{17}  And alas for the women that are with child, and for those that are nursing infants in those days! 
 \V{18}  Pray, too, that this may not occur in winter. 
 \V{19}  For those days will be a time of distress, the like of which has not occurred from the beginning of God’s creation until now — and never will again. 
 \V{20}  And, had not the Lord put a limit to those days, not a single soul would escape; but, for the sake of God’s own chosen People, he did limit them. \indenting[yes]\par 
 \V{21}  “And at that time if any one should say to you ‘Look, here is the Christ!’ ‘Look, there he is!’, do not believe it; 
 \V{22}  for false Christs and false Prophets will arise, and display signs and marvels, to lead astray, were it possible, even God’s People. 
 \V{23}  But see that you are on your guard! I have told you all this beforehand. \indenting[yes]\par 
 \V{24}  “In those days, after that time of distress, ‘the sun will be darkened, the moon will not give her light, 
 \V{25}  the stars will be falling from the heavens,’ and ‘the forces that are in the heavens will be convulsed.’ 
 \V{26}  Then will be seen the ‘Son of Man coming in clouds’ with great power and glory; 
 \V{27}  and then he will send the angels, and gather his People from the four winds, from one end of the world to the other. \indenting[yes]\par 
 \V{28}  “Learn the lesson taught by the fig-tree. As soon as its branches are full of sap, and it is bursting into leaf, you know that summer is near. 
 \V{29}  And so may you, as soon as you see these things happening, know that he is at your doors. 
 \V{30}  I tell you that even the present generation will not pass away, until all these things have taken place. 
 \V{31}  The heavens and the earth will pass away, but my words will not pass away. \indenting[yes]\par 
 \V{32}  “But about ‘That Day,’ or ‘The Hour,’ no one knows — not even the angels in Heaven, nor yet the Son — but only the Father. \indenting[yes]\par 
 \V{33}  “See that you are on the watch; for you do not know when the time will be. 
 \V{34}  It is like a man going on a journey, who leaves his home, puts his servants in charge — each having their special duty — and orders the porter to watch. 
 \V{35}  Therefore watch, for you cannot be sure when the Master of the house is coming — whether in the evening, at midnight, at daybreak, or in the morning —  
 \V{36}  lest he should come suddenly and find you asleep. 
 \V{37}  And what I say to you I say to all — Watch!” \blank\indenting[no]\par 
 \C{14}  \CapStretch{It was now two days before the Festival of} the Passover and the Unleavened bread. The Chief Priests and the Teachers of the Law were looking for an opportunity to arrest Jesus by stealth, and to put him to death; 
 \V{2}  for they said: “Not during the Festival, for fear of a riot.” \indenting[yes]\par 
 \V{3}  When Jesus was still at Bethany, in the house of Simon the leper, while he was at table, a woman came with an alabaster jar of choice spikenard perfume of great value. She broke the jar, and poured the perfume on his head. 
 \V{4}  Some of those who were present said to one another indignantly: “Why has the perfume been wasted like this? 
 \V{5}  This perfume could have been sold for more than thirty pounds, and the money given to the poor.” 
 \V{6}  “Let her alone,”  said Jesus, as they began to find fault with her,   “why are you troubling her? This is a beautiful deed that she has done for me. 
 \V{7}  You always have the poor with you, and whenever you wish you can do good to them; but you will not always have me. 
 \V{8}  She has done what she could; she has perfumed my body beforehand for my burial. 
 \V{9}  And I tell you, wherever, in the whole world, the Good News is proclaimed, what this woman has done will be told in memory of her.” \indenting[yes]\par 
 \V{10}  After this, Judas Iscariot, one of the Twelve, went to the Chief Priests, to betray Jesus to them. 
 \V{11}  They were glad to hear what he said, and promised to pay him. So he began looking for a good opportunity to betray Jesus. \blank\indenting[no]\par 
 \V{12}  \CapStretch{On the first day of the Festival of the Unleavened} bread, when it was customary to kill the Passover lambs, his disciples said to Jesus: “Where do you wish us to go and make preparations for your eating the Passover?” 
 \V{13}  Jesus sent forward two of his disciples and said to them:   “Go into the city, and there a man carrying a pitcher of water will meet you; follow him; 
 \V{14}  and, wherever he goes in, say to the owner of the house ‘The Teacher says — Where is my room where I am to eat the Passover with my disciples?’ 
 \V{15}  He will himself show you a large upstairs room, set out ready; and there make preparations for us.” 
 \V{16}  So the disciples set out and went into the city, and found everything just as Jesus had told them; and they prepared the Passover. \indenting[yes]\par  In the evening he went there with the Twelve, 
 \V{17, 18}  and when they had taken their places and were eating, Jesus said:   “I tell you that one of you is going to betray me — one who is eating with me.” 
 \V{19}  They were grieved at this, and began to say to him, one after another: “Can it be I?” 
 \V{20}  “It is one of you Twelve,”  said Jesus,   “the one who is dipping his bread beside me into the dish. 
 \V{21}  True, the Son of Man must go, as Scripture says of him, yet alas for that man by whom the Son of Man is being betrayed! For that man ‘it would be better never to have been born!’” \indenting[yes]\par 
 \V{22}  While they were eating, Jesus took some bread, and, after saying the blessing, broke it, and gave it to them, and said:   “Take it; this is my body.” 
 \V{23}  Then he took a cup, and, after saying the thanksgiving, gave it to them, and they all drank from it. 
 \V{24}  “This is my Covenant-blood,”  he said,   “which is poured out on behalf of many. 
 \V{25}  I tell you that I shall never again drink of the juice of the grape, until that day when I shall drink it new in the Kingdom of God.” \indenting[yes]\par 
 \V{26}  They then sang a hymn, and went out up the Mount of Olives, 
 \V{27}  presently Jesus said to them:   “All of you will fall away; for Scripture says — ‘I will strike down the Shepherd, and the sheep will be scattered.’ 
 \V{28}  Yet, after I have risen, I shall go before you into Galilee.” 
 \V{29}  “Even if every one else falls away,” said Peter, “yet I shall not.” 
 \V{30}  “I tell you,”  answered Jesus,   “that you yourself today — yes, this very night — before the cock crows twice, will disown me three times.” 
 \V{31}  But Peter vehemently protested: “Even if I must die with you, I shall never disown you!” And they all said the same. \blank\indenting[no]\par 
 \V{32}  \CapStretch{Presently they came to a garden known as} Gethsemane, and Jesus said to his disciples   “Sit down here while I pray.” 
 \V{33}  He took with him Peter, James, and John; and began to show signs of great dismay and deep distress of mind. 
 \V{34}  “I am sad at heart,”  he said,   “sad even to death; wait here, and watch.” 
 \V{35}  Going on a little further, he threw himself on the ground, and began to pray that, if it were possible, he might be spared that hour. 
 \V{36}  “Abba, Father,”  he said,   “all things are possible to you; take away this cup from me; yet, not what I will, but what you will.” \indenting[yes]\par 
 \V{37}  Then he came and found the three Apostles asleep.   “Simon,”  he said to Peter,   “are you asleep? Could not you watch for one hour? 
 \V{38}  Watch and pray,” he said to them all, “that you may not fall into temptation. True, the spirit is eager, but human nature is weak.”  Again he went away, and prayed in the same words; 
 \V{39, 40}  and coming back again he found them asleep, for their eyes were heavy; and they did not know what to say to him. \indenting[yes]\par 
 \V{41}  A third time he came, and said to them:   “Sleep on now, and rest yourselves. Enough! My time has come. Hark! the Son of Man is being betrayed into the hands of wicked people. 
 \V{42}  Up, and let us be going. Look! my betrayer is close at hand.” \indenting[yes]\par 
 \V{43}  And just then, while he was still speaking, Judas, who was one of the Twelve, came up; and with him a crowd of people, with swords and clubs, sent by the Chief Priests, the Teachers of the Law, and the Councillors. 
 \V{44}  Now the betrayer had arranged a signal with them. “The man whom I kiss,” he had said, “will be the one; arrest him and take him away safely.” 
 \V{45}  As soon as Judas came, he went up to Jesus at once, and said: “Rabbi!” and kissed him. 
 \V{46}  Then the men seized Jesus, and arrested him. \indenting[yes]\par 
 \V{47}  One of those who were standing by drew his sword, and struck at the High Priest’s servant, and cut off his ear. 
 \V{48}  But Jesus interposed, and said to the men:   “Have you come out, as if after a robber, with swords and clubs, to take me?  
 \V{49}  I have been among you day after day in the Temple Courts teaching, and yet you did not arrest me; but this is in fulfillment of the Scriptures.”  
 \V{50}  And all the Apostles forsook him, and fled. 
 \V{51}  One young man did indeed follow him, wrapped only in a linen sheet. They tried to arrest him; 
 \V{52}  But he left the sheet in their hands, and fled naked. \blank\indenting[no]\par 
 \V{53}  \CapStretch{Then they took Jesus to the High Priest}; and all the Chief Priests, the Councillors, and the Teachers of the Law assembled. 
 \V{54}  Peter, who had followed Jesus at a distance into the court- yard of the High Priest, was sitting there among the police- officers, warming himself at the blaze of the fire. \indenting[yes]\par 
 \V{55}  Meanwhile the Chief Priest and the whole of the High Council were trying to get such evidence against Jesus as would warrant his being put to death, but they could not find any; 
 \V{56}  for, though there were many who gave false evidence against him, yet their evidence did not agree. 
 \V{57}  Presently some men stood up, and gave this false evidence against him —  
 \V{58}  “We ourselves heard him say ‘I will destroy this Temple made with hands, and in three days build another made without hands.’” 
 \V{59}  Yet not even on that point did their evidence agree. \indenting[yes]\par 
 \V{60}  Then the High Priest stood forward, and questioned Jesus. “Have you no answer to make?” he asked. “What is this evidence which these men are giving against you?” 
 \V{61}  But Jesus remained silent, and made no answer. \indenting[yes]\par  A second time the High Priest questioned him. “Are you,” he asked, “the Christ, the Son of the Blessed One?” 
 \V{62}  “I am,”  replied Jesus,   “and you shall all see the Son of Man sitting on the right hand of the Almighty; and ‘coming in the clouds of heaven’.” 
 \V{63}  At this the High Priest tore his vestments. “Why do we want any more witnesses?” he exclaimed. 
 \V{64}  “You heard his blasphemy? What is your verdict?” They all condemned him, declaring that he deserved death. \indenting[yes]\par 
 \V{65}  Some of those present began to spit at him, and to blindfold his eyes, and strike him, saying, as they did so, “Now play the Prophet!” and even the police-officers received him with blows. \indenting[yes]\par 
 \V{66}  While Peter was in the court-yard down below, one of the High Priest’s maidservants came up; 
 \V{67}  and, seeing Peter warming himself, she looked closely at him, and exclaimed: “Why, you were with Jesus, the Nazarene!” 
 \V{68}  But Peter denied it. “I do not know or understand what you mean,” he replied. Then he went out into the porch; 
 \V{69}  and there the maidservant, on seeing him, began to say again to the by-standers: “This is one of them!” 
 \V{70}  But Peter again denied it. \indenting[yes]\par  Soon afterwards the bystanders again said to him: “You certainly are one of them; why you are a Galilean!” 
 \V{71}  But he began to swear with the most solemn imprecations: “I do not know the man you are speaking about.” 
 \V{72}  At that moment, for the second time, a cock crowed; and Peter remembered the words that Jesus had said to him —   ‘Before a cock has crowed twice, you will disown me three times’;  and, as he thought of it, he began to weep. \blank\indenting[no]\par 
 \C{15}  \CapStretch{As soon as it was daylight}, the Chief Priests, after holding a consultation with the Councillors and Teachers of the Law — that is to say, the whole High Council — put Jesus in chains, and took him away, and gave him up to Pilate. 
 \V{2}  “Are you the King of the Jews?” asked Pilate.   “It is true,”  replied Jesus. 
 \V{3}  Then the Chief Priests brought a number of charges against him; 
 \V{4}  upon which Pilate questioned Jesus again. “Have you no reply to make?” he asked. “Listen, how many charges they are bringing against you.” 
 \V{5}  But Jesus still made no reply whatever; at which Pilate was astonished. \indenting[yes]\par 
 \V{6}  Now, at the Feast, Pilate used to grant the people the release of any one prisoner whom they might ask for. 
 \V{7}  A man called Barabbas was in prison, with the rioters who had committed murder during a riot. 
 \V{8}  So, when the crowd went up and began to ask Pilate to follow his usual custom, 
 \V{9}  he answered: “Do you want me to release the ‘King of the Jews’ for you?” 
 \V{10}  For he was aware that it was out of jealousy that the Chief Priests had given Jesus up to him. 
 \V{11}  But the Chief Priests incited the crowd to get Barabbas released instead. 
 \V{12}  Pilate, however, spoke to them again: “What shall I do then with the man whom you call the ‘King of the Jews’?” 
 \V{13}  Again they shouted: “Crucify him!” 
 \V{14}  “Why, what harm has he done?” Pilate kept saying to them. But they shouted furiously: “Crucify him!” 
 \V{15}  And Pilate, wishing to satisfy the crowd, released Barabbas to them, and, after scourging Jesus, gave him up to be crucified. \indenting[yes]\par 
 \V{16}  The soldiers then took Jesus away into the court-yard — that is the Government House — and they called the whole garrison together. 
 \V{17}  They dressed him in a purple robe, and, having twisted a crown of thorns, put it on him, 
 \V{18}  and then began to salute him. “Long life to you, King of the Jews!” they said. 
 \V{19}  And they kept striking him on the head with a rod, spitting at him, and bowing to the ground before him — going down on their knees; 
 \V{20}  and, when they had left off mocking him, they took off the purple robe, and put his own clothes on him. \blank\indenting[no]\par 
 \V{21}  \CapStretch{They led Jesus out to crucify him}; and they compelled a passer-by, Simon from Cyrene, who was on his way in from the country, the father of Alexander and Rufus, to go with them to carry his cross. \indenting[yes]\par 
 \V{22}  They brought Jesus to the place which was known as Golgotha — a name which means ‘Place of a Skull.’ 
 \V{23}  There they offered him drugged wine; but Jesus refused it. 
 \V{24}  Then they crucified him, and divided his clothes among them, casting lots for them, to settle what each should take. \indenting[yes]\par  It was nine in the morning when they crucified him. 
 \V{25, 26}  The words of the charge against him, written up over his head, read — ‘THE KING OF THE JEWS.’ 
 \V{27}  And with him they crucified two robbers, one on the right, and the other on the left. \indenting[yes]\par 
 \V{29}  The passers-by railed at him, shaking their heads, as they said: “Ah! you who would destroy the Temple and build one in three days,  come down from the cross and save yourself!” 
 \V{30, 31}  In the same way the Chief Priests, with the Teachers of the Law, said to one another in mockery: 
 \V{32}  “He saved others, but he cannot save himself! Let the Christ, the ‘King of Israel,’ come down from the cross now, that we may see it and believe.” Even the men who had been crucified with Jesus reviled him. \indenting[yes]\par 
 \V{33}  At midday, a darkness came over the whole country, lasting till three in the afternoon. 
 \V{34}  And, at three, Jesus called out loudly:   ‘Eloi, Eloi, lama sabacthani?’”  which means   ‘My God, my God, why have you forsaken me?’ 
 \V{35}  Some of those standing round heard this, and said: “Listen! He is calling for Elijah!” 
 \V{36}  And a man ran, and, soaking a sponge in common wine, put it on the end of a rod, and offered it to him to drink, saying as he did so: “Wait and let us see if Elijah is coming to take him down.” 
 \V{37}  But Jesus, giving a loud cry, expired. 
 \V{38}  The Temple curtain was torn in two from top to bottom. 
 \V{39}  The Roman Officer, who was standing facing Jesus, on seeing the way in which he expired, exclaimed: “This man must indeed have been ‘God’s Son’!” \blank\indenting[no]\par 
 \V{40}  \CapStretch{There were some women also watching from} a distance, among them being Mary of Magdala, Mary the mother of James the Little and of Joseph, and Salome —  
 \V{41}  all of whom used to accompany Jesus when he was in Galilee, and attend on him — besides many other women who had come up with him to Jerusalem. \indenting[yes]\par 
 \V{42}  The evening had already fallen, when, as it was the Preparation Day — the day before the Sabbath —  
 \V{43}  Joseph from Ramah, a Councillor of good position, who was himself living in expectation of the Kingdom of God, came and ventured to go in to see Pilate, and to ask for the body of Jesus. 
 \V{44}  But Pilate was surprised to hear that he had already died. So he sent for the Officer, and asked if he were already dead; 
 \V{45}  and, on learning from the Officer that it was so, he gave the corpse to Joseph. 
 \V{46}  Joseph, having bought a linen sheet, took Jesus down, and wound the sheet round him, and laid him in a tomb which had been cut out of the rock; and then rolled a stone up against the entrance of the tomb. 
 \V{47}  Mary of Magdala and Mary, the mother of Joseph, were watching to see where he was laid. \MS{The Risen Life Announced} \indenting[yes]\par 
 \C{16}  When the Sabbath was over, Mary of Magdala, Mary the mother of James, and Salome bought some spices, so that they might go and anoint the body of Jesus. 
 \V{2}  And very early on the first day of the week they went to the tomb, after sunrise. 
 \V{3}  They were saying to one another: “Who will roll away the stone for us from the entrance of the tomb?” 
 \V{4}  But, on looking up, they saw that the stone had already been rolled back; it was a very large one. 
 \V{5}  Going into the tomb, they saw a young man sitting on their right, in a white robe, and they were dismayed; But he said to them: 
 \V{6}  “Do not be dismayed; you are looking for Jesus, the Nazarene, who has been crucified; he has risen, he is not here! Look! Here is the place where they laid him. 
 \V{7}  But go, and say to his disciples and to Peter ‘He is going before you into Galilee; there you will see him, as he told you.’” 
 \V{8}  They went out, and fled from the tomb, for they were trembling and bewildered; and they did not say a word to any one, for they were frightened; \MS{A late appendix} \MSS{(Inserted in some manuscripts from an ancient source)} \indenting[yes]\par 
 \V{9}  After his rising again, early on the first day of the week, Jesus appeared first of all to Mary of Magdala, from whom he had driven out seven demons. 
 \V{10}  She went and told the news to those who had been with him and who were now in sorrow and tears; 
 \V{11}  yet even they, when they heard that he was alive and had been seen by her, did not believe it. 
 \V{12}  Afterwards, altered in appearance, he made himself known to two of them, as they were walking, on their way into the country. 
 \V{13}  They also went and told the rest, but they did not believe even them. 
 \V{14}  Later on, he made himself known to the Eleven themselves as they were at a meal, and reproached them with their want of faith and their stubbornness, because they did not believe those who had seen him after he had risen from the dead. 
 \V{15}  Then he said to them:   “Go into all the world, and proclaim the Good News to all creation. 
 \V{16}  He who believes and is baptized shall be saved; but he who refuses to believe will be condemned. 
 \V{17}  Moreover these signs shall attend those who believe. In my Name they shall drive out demons; they shall speak with ‘tongues’; 
 \V{18}  they shall take up serpents in their hands; and, if they drink any poison, it shall not hurt them; they will place their hands on sick people and they shall recover.” 
 \V{19}  So the Lord Jesus, after he had spoken to them, was taken up into Heaven, and sat at the right hand of God. 
 \V{20}  But they set out, and made the proclamation everywhere, the Lord working with them, and confirming the Message by the signs which attended it. \MS{Another appendix} \indenting[yes]\par  But all that had been enjoined on them they reported briefly to Peter and his companions. Afterwards Jesus himself sent them out, from east to west, the sacred and imperishable proclamation of eternal Salvation. \marking[RAChapter]{ } \marking[RABook]{ } \marking[RASection]{ }\RAHeader{Matthew} \MT{The Good News According to Matthew} \indenting[yes]\par  A genealogy of Jesus Christ, a descendant of David and Abraham. 
 \V{2}  Abraham was the father of Isaac, Isaac of Jacob, Jacob of Judah and his brothers, 
 \V{3}  Judah of Perez and Zerah, whose mother was Tamar, Perez of Hezron, Hezron of Ram, 
 \V{4}  Ram of Amminadab, Amminadab of Nashon, Nashon of Salmon, 
 \V{5}  Salmon of Boaz, whose mother was Rahab, Boaz of Obed, whose mother was Ruth, Obed of Jesse, 
 \V{6}  Jesse of David the King. David was the father of Solomon, whose mother was Uriah’s widow, 
 \V{7}  Solomon of Rehoboam, Rehoboam of Abijah, Abijah of Asa, 
 \V{8}  Asa of Jehoshaphat, Jehoshaphat of Jehoram, Jehoram of Uzziah, 
 \V{9}  Uzziah of Jotham, Jotham of Ahaz, Ahaz of Hezekiah, 
 \V{10}  Hezekiah of Manasseh, Manasseh of Ammon, Ammon of Josiah, 
 \V{11}  Josiah of Jeconiah and his brothers, at the time of the Exile to Babylon. 
 \V{12}  After the Exile to Babylon — Jeconiah was the father of Shealtiel, Shealtiel of Zerubbabel, 
 \V{13}  Zerubbabel of Abiud, Abiud of Eliakim, Eliakim of Azor, 
 \V{14}  Azor of Zadok, Zadok of Achim, Achim of Eliud, 
 \V{15}  Eliud of Eleazar, Eleazar of Matthan, Matthan of Jacob, 
 \V{16}  Jacob of Joseph, the husband of Mary, who was the mother of Jesus, who is called ‘Christ’. 
 \V{17}  So the whole number of generations from Abraham to David is fourteen; from David to the Exile to Babylon fourteen; and from the Exile to Babylon to the Christ fourteen. \blank\indenting[no]\par 
 \V{18}  \CapStretch{The birth of Jesus Christ took place as follows:}  \indenting[yes]\par  His mother Mary was betrothed to Joseph, but, before the marriage took place, she found herself to be with child by the power of the Holy Spirit. 
 \V{19}  Her husband, Joseph, was a religious man and, being unwilling to expose her to contempt, resolved to put an end to their betrothal privately. 
 \V{20}  He had been dwelling upon this, when an angel of the Lord appeared to him in a dream. \indenting[yes]\par  “Joseph, son of David,” the angel said, “do not be afraid to take Mary for your wife, for her child has been conceived by the power of the Holy Spirit. 
 \V{21}  She shall give birth to a son; and you shall give him the name Jesus, for it is he who shall save his people from their sins.” \indenting[yes]\par 
 \V{22}  All this happened in fulfillment of these words of the Lord in the Prophet, where he says —  \startnarrower[1*left,1*right]\indenting[no]
 \V{23}  ‘Behold! the virgin shall be with child and shall give birth to a son, And they will give him the name Immanuel’ \stopnarrower\indenting[no]\par  — a word which means ‘God is with us.’ 
 \V{24}  When Joseph awoke from his sleep, he did as the angel of the Lord had directed him. 
 \V{25}  He made Mary his wife, but did not live with her as her husband until after the birth of her son; and to this son he gave the name Jesus. \blank\indenting[no]\par 
 \C{2}  \CapStretch{After the birth of Jesus at Bethlehem in} Judea, in the reign of King Herod, some Astrologers from the East arrived in Jerusalem, asking: 
 \V{2}  “Where is the new-born King of the Jews? for we saw his star in the east, and have come to do homage to him.” 
 \V{3}  When King Herod heard of this, he was much troubled, and so, too, was all Jerusalem. 
 \V{4}  He called together all the Chief Priests and Teachers of the Law in the nation, and questioned them as to where the Christ was to be born. \indenting[yes]\par 
 \V{5}  “At Bethlehem in Judea,” was their answer; “for it is said in the Prophet —  \startnarrower[1*left,1*right]\indenting[no]
 \V{6}  ‘And you, Bethlehem in Judah’s land, \stopnarrower\startnarrower[2*left,1*right]\indenting[no] Art in no way least among the chief cities of Judah; \stopnarrower\startnarrower[1*left,1*right]\indenting[no] For out of thee will come a Chieftain —  \stopnarrower\startnarrower[2*left,1*right]\indenting[no] One who will shepherd my people Israel.’” \stopnarrower\indenting[yes]\par 
 \V{7}  Then Herod secretly sent for the Astrologers, and ascertained from them the date of the appearance of the star; 
 \V{8}  And, sending them to Bethlehem, he said: “Go and make careful inquiries about the child, and, as soon as you have found him, bring me word, that I, too, may go and do homage to him.” 
 \V{9}  The Astrologers heard what the King had to say, and then continued their journey. And the star which they had seen in the east led them on, until it reached, and stood over, the place where the child was. 
 \V{10}  At the sight of the star they were filled with joy. 
 \V{11}  Entering the house, they saw the child with his mother, Mary, and fell at his feet and did homage to him. Then they unpacked their treasures, and offered to the child presents of gold, frankincense, and myrrh. 
 \V{12}  But afterwards, having been warned in a dream not to go back to Herod, they returned to their own country by another road. \indenting[yes]\par 
 \V{13}  After they had left, an angel of the Lord appeared to Joseph in a dream, and said: \indenting[yes]\par  “Awake, take the child and his mother, and seek refuge in Egypt; and stay there until I bid you return, for Herod is about to search for the child, to put him to death.” 
 \V{14}  Joseph awoke, and taking the child and his mother by night, went into Egypt, 
 \V{15}  And there he stayed until Herod’s death; in fulfillment of these words of the Lord in the Prophet, where he says —  \startnarrower[1*left,1*right]\indenting[no] ‘Out of Egypt I called my Son.’ \stopnarrower\indenting[yes]\par 
 \V{16}  When Herod found that he had been trifled with by the Astrologers, he was very angry. He sent and put to death all the boys in Bethlehem and the whole of that neighborhood, who were two years old or under, guided by the date which he had ascertained from the Astrologers. 
 \V{17}  Then were fulfilled these words spoken in the Prophet Jeremiah, where he says —  \startnarrower[1*left,1*right]\indenting[no]
 \V{18}  ‘A voice was heard in Ramah, \stopnarrower\startnarrower[2*left,1*right]\indenting[no] Weeping and much lamentation; \stopnarrower\startnarrower[1*left,1*right]\indenting[no] Rachel, weeping for her children, \stopnarrower\startnarrower[2*left,1*right]\indenting[no] Refused all comfort because they were not.’ \stopnarrower\indenting[yes]\par 
 \V{19}  But, on the death of Herod, an angel of the Lord appeared in a dream to Joseph in Egypt, and said: 
 \V{20}  “Awake, take the child and his mother, and go into the Land of Israel, for those who sought to take the child’s life are dead.” 
 \V{21}  And he awoke, and taking the child and his mother, went into the Land of Israel. 
 \V{22}  But, hearing that Archelaus had succeeded his father Herod as King of Judea, he was afraid to go back there; and having been warned in a dream, he went into the part of the country called Galilee. 
 \V{23}  And there he settled in the town of Nazareth, in fulfillment of these words in the Prophets — ‘He will be called a Nazarene.’ \blank\indenting[no]\par 
 \C{3}  \CapStretch{About that time John the Baptist first appeared,} proclaiming in the Wilderness of Judea: 
 \V{2}  “Repent, for the Kingdom of Heaven is at hand.” 
 \V{3}  This is he who was spoken of in the Prophet Isaiah, where he says —  \startnarrower[1*left,1*right]\indenting[no] ‘The voice of one crying aloud in the Wilderness: \stopnarrower\startnarrower[1*left,1*right]\indenting[no] “Make ready the way of the Lord, \stopnarrower\startnarrower[1*left,1*right]\indenting[no] Make his paths straight.”’ \stopnarrower\indenting[yes]\par 
 \V{4}  John wore clothing made of camels’ hair, with a belt of leather round his waist, and his food was locusts and wild honey. 
 \V{5}  At that time Jerusalem, and all Judea, as well as the whole district of the Jordan, went out to him 
 \V{6}  And were baptized by him in the river Jordan, confessing their sins. \indenting[yes]\par 
 \V{7}  When, however, John saw many of the Pharisees and Sadducees coming to receive his baptism, he said to them: \indenting[yes]\par  “You brood of vipers! Who has prompted you to seek refuge from the coming judgment? 
 \V{8}  Let your life, then, prove your repentance; 
 \V{9}  And do not think that you can say among yourselves ‘Abraham is our ancestor,’ for I tell you that out of these very stones God is able to raise descendants for Abraham! 
 \V{10}  Already the axe is lying at the root of the trees. Therefore every tree that fails to bear good fruit will be cut down and thrown into the fire. 
 \V{11}  I, indeed, baptize you with water to teach repentance; but He who is Coming after me is more powerful than I, and I am not fit even to carry his sandals. He will baptize you with the Holy Spirit and with fire. 
 \V{12}  His winnowing-fan is in his hand, and he will clear his threshing-floor, and store his grain in the barn, but the chaff he will burn with inextinguishable fire.” \indenting[yes]\par 
 \V{13}  Then Jesus came from Galilee to the Jordan, to John, to be baptized by him. 
 \V{14}  But John tried to prevent him. \indenting[yes]\par  “It is I,” he said, “who need to be baptized by you; why then do you come to me?” \indenting[yes]\par 
 \V{15}  “Let it be so for the present,”   Jesus answered,   “since it is fitting for us thus to satisfy every claim of religion.”   Upon this, John consented. \indenting[yes]\par 
 \V{16}  After the baptism of Jesus, and just as he came up from the water, the heavens opened, and he saw the Spirit of God descending, like a dove, and alighting upon him, 
 \V{17}  And from the heavens there came a voice which said: “This is my son, the Beloved, in whom I delight.” \blank\indenting[no]\par 
 \C{4}  \CapStretch{Then Jesus was led up into the Wilderness} by the Spirit to be tempted by the Devil. 
 \V{2}  And, after he had fasted for forty days and forty nights, he became hungry. 
 \V{3}  And the Tempter came to him, and said: \startnarrower[1*left,1*right]\indenting[no] “If you are God’s Son, tell these stones to become loaves of bread.” \stopnarrower\indenting[no]\par 
 \V{4}  But Jesus answered:   “Scripture says —  \startnarrower[1*left,1*right]\indenting[no] ‘It is not on bread alone that man is to live, but on every word that comes from the mouth of God.’”  \stopnarrower\indenting[no]\par 
 \V{5}  Then the Devil took him to the Holy City, and, placing him on the parapet of the temple, said to him: 
 \V{6}  “If you are God’s Son, throw yourself down, for Scripture says —  \startnarrower[1*left,1*right]\indenting[no] ‘He will give his angels commands about thee, \stopnarrower\startnarrower[1*left,1*right]\indenting[no] And on their hands they will upbear thee, \stopnarrower\startnarrower[1*left,1*right]\indenting[no] Lest ever thou shouldst strike thy foot against a stone.’” \stopnarrower\indenting[no]\par 
 \V{7}  “Scripture also says,”   answered Jesus,  \startnarrower[1*left,1*right]\indenting[no] “Thou shalt not tempt the Lord thy God.’”  \stopnarrower\indenting[no]\par 
 \V{8}  The third time, the Devil took Jesus to a very high mountain, and, showing him all the kingdoms of the world and their splendor, said to him: \startnarrower[1*left,1*right]\indenting[no]
 \V{9}  “All these I will give you, if you will fall at my feet and do homage to me.” 
 \V{10}  Then Jesus said to him:   “Begone, Satan! for Scripture says —  \stopnarrower\startnarrower[1*left,1*right]\indenting[no] ‘Thou shalt do homage to the Lord thy God, and worship him only.’”  
 \V{11}  Then the Devil left him alone, and angels came and ministered to him. \stopnarrower\blank\indenting[no]\par 
 \V{12}  \CapStretch{When Jesus heard that John had been committed} to prison, he retired to Galilee. 
 \V{13}  Afterwards, leaving Nazareth, he went and settled at Capernaum, which is by the side of the Sea, within the borders of Zebulun and Naphtali; 
 \V{14}  In fulfillment of these words in the Prophet Isaiah —  \startnarrower[1*left,1*right]\indenting[no]
 \V{15}  ‘The land of Zebulun and the land of Naphtali, \stopnarrower\startnarrower[1*left,1*right]\indenting[no] The land of the Road by the Sea, and beyond the Jordan, \stopnarrower\startnarrower[2*left,1*right]\indenting[no] With Galilee of the Gentiles —  \stopnarrower\startnarrower[1*left,1*right]\indenting[no]
 \V{16}  The people who were dwelling in darkness \stopnarrower\startnarrower[2*left,1*right]\indenting[no] Have seen a great Light, \stopnarrower\startnarrower[1*left,1*right]\indenting[no] And, for those who were dwelling in the shadow-land of Death, \stopnarrower\startnarrower[2*left,1*right]\indenting[no] A Light has risen!’ \stopnarrower\blank\indenting[no]\par 
 \V{17}  \CapStretch{At that time Jesus began to proclaim —}  \startnarrower[1*left,1*right]\indenting[no] “Repent, for the Kingdom of Heaven is at hand.”  \stopnarrower\indenting[yes]\par 
 \V{18}  As Jesus was walking along the shore of the Sea of Galilee, he saw two brothers — Simon, also known as Peter, and his brother Andrew — casting a net into the Sea; for they were fishermen. \indenting[yes]\par 
 \V{19}  “Come and follow me,”   Jesus said,   “and I will set you to fish for men.”  
 \V{20}  The two men left their nets at once and followed him. 
 \V{21}  Going further on, he saw two other men who were also brothers, James, Zebediah’s son, and his brother John, in their boat with their father, mending their nets. Jesus called them, 
 \V{22}  And they at once left their boat and their father, and followed him. \indenting[yes]\par 
 \V{23}  And Jesus went all through Galilee, teaching in their Synagogues, proclaiming the Good News of the Kingdom, and curing every kind of disease and every kind of sickness among the people; 
 \V{24}  And his fame spread all through Syria. They brought to him all who were ill with any form of disease, or who were suffering pain — any who were either possessed by demons, or were lunatic, or paralyzed; and he cured them. 
 \V{25}  And he was followed by large crowds from Galilee, the district of the Ten Towns, Jerusalem, Judea, and from beyond the Jordan. \indenting[yes]\par 
 \C{5}  On seeing the crowds of People, Jesus went up the hill; and, when he had taken his seat, his disciples came up to him; 
 \V{2}  And he began to teach them as follows: \startnarrower[1*left,1*right]\indenting[no]
 \V{3}  “Blessed are the poor in spirit,  \stopnarrower\startnarrower[2*left,1*right]\indenting[no] for theirs is the Kingdom of Heaven. \stopnarrower\startnarrower[1*left,1*right]\indenting[no]
 \V{4}  Blessed are the mourners,  \stopnarrower\startnarrower[2*left,1*right]\indenting[no] for they shall be comforted. \stopnarrower\startnarrower[1*left,1*right]\indenting[no]
 \V{5}  Blessed are the gentle,  \stopnarrower\startnarrower[2*left,1*right]\indenting[no] for they shall inherit the earth. \stopnarrower\startnarrower[1*left,1*right]\indenting[no]
 \V{6}  Blessed are those who hunger and thirst for righteousness,  \stopnarrower\startnarrower[2*left,1*right]\indenting[no] for they shall be satisfied. \stopnarrower\startnarrower[1*left,1*right]\indenting[no]
 \V{7}  Blessed are the merciful,  \stopnarrower\startnarrower[2*left,1*right]\indenting[no] for they shall find mercy. \stopnarrower\startnarrower[1*left,1*right]\indenting[no]
 \V{8}  Blessed are the pure in heart,  \stopnarrower\startnarrower[2*left,1*right]\indenting[no] for they shall see God. \stopnarrower\startnarrower[1*left,1*right]\indenting[no]
 \V{9}  Blessed are the peacemakers,  \stopnarrower\startnarrower[2*left,1*right]\indenting[no] for they shall be called Sons of God. \stopnarrower\startnarrower[1*left,1*right]\indenting[no]
 \V{10}  Blessed are those who have been persecuted in the cause of righteousness,  \stopnarrower\startnarrower[2*left,1*right]\indenting[no] for theirs is the Kingdom of Heaven. \stopnarrower\startnarrower[1*left,1*right]\indenting[no]
 \V{11}  Blessed are you when people taunt you, and persecute you, and say everything evil about you — untruly, and on my account. 
 \V{12}  Be glad and rejoice, because your reward in Heaven will be great; for so men persecuted the Prophets who lived before you. \stopnarrower\indenting[yes]\par 
 \V{13}  It is you who are the Salt of the earth; but, if the salt should lose its strength, what will you use to restore its saltiness? It is no longer good for anything, but is thrown away, and trampled underfoot. 
 \V{14}  It is you who are the Light of the world. A town that stands on a hill cannot be hidden. 
 \V{15}  Men do not light a lamp and put it under the corn-measure, but on the lamp-stand, where it gives light to every one in the house. 
 \V{16}  Let your light so shine before the eyes of your fellow men, that, seeing your good actions, they may praise your Father who is in Heaven. \indenting[yes]\par 
 \V{17}  Do not think that I have come to do away with the Law or the Prophets; I have not come to do away with them, but to complete them. 
 \V{18}  For I tell you, until the heavens and the earth disappear, not even the smallest letter, nor one stroke of a letter, shall disappear from the Law until all is done. 
 \V{19}  Whoever, therefore, breaks one of these commandments, even the least of them, and teaches others to do so, will be the least- esteemed in the Kingdom of Heaven; but whoever keeps them, and teaches others to do so, will be esteemed great in the Kingdom of Heaven. 
 \V{20}  Indeed I tell you that, unless your religion is above that of the Teachers of the Law, and Pharisees, you will never enter the Kingdom of Heaven. \indenting[yes]\par 
 \V{21}  You have heard that to our ancestors it was said — ‘Thou shalt not commit murder,’ and ‘Whoever commits murder shall be liable to answer for it to the Court.’ 
 \V{22}  I, however, say to you that any one who cherishes anger against his brother shall be liable to answer for it to the Court; and whoever pours contempt upon his brother shall be liable to answer for it to the High Council, while whoever calls down curses upon him shall be liable to answer for it in the fiery Pit. 
 \V{23}  Therefore, when presenting your gift at the altar, if even there you remember that your brother has some grievance against you, 
 \V{24}  Leave your gift there, before the altar, go and be reconciled to your brother, first, then come and present your gift. 
 \V{25}  Be ready to make friends with your opponent, even when you meet him on your way to the court; for fear that he should hand you over to the judge, and the judge to his officer, and you should be thrown into prison. 
 \V{26}  I tell you, you will not come out until you have paid the last penny. \indenting[yes]\par 
 \V{27}  You have heard that it was said — ‘Thou shalt not commit adultery.’ 
 \V{28}  I, however, say to you that any one who looks at a woman with an impure intention has already committed adultery with her in his heart. 
 \V{29}  If your right eye is a snare to you, take it out and throw it away. It would be best for you to lose one part of your body, and not to have the whole of it thrown into the Pit. 
 \V{30}  And, if your right hand is a snare to you, cut it off and throw it away. It would be best for you to lose one part of your body, and not to have the whole of it go down to the Pit. \indenting[yes]\par 
 \V{31}  It was also said — ‘Let any one who divorces his wife serve her with a notice of separation.’ 
 \V{32}  I, however, say to you that any one who divorces his wife, except on the ground of her unchastity, leads to her committing adultery; while any one who marries her after her divorce is guilty of adultery. 
 \V{33}  Again, you have heard that to our ancestors it was said — ‘Thou shalt not break an oath, but thou shall keep thine oaths as a debt due to the Lord.’ 
 \V{34}  I, however, say to you that you must not swear at all, either by Heaven, since that is God’s throne, 
 \V{35}  Or by the earth, since that is his footstool, or by Jerusalem, since that is the city of the Great King. 
 \V{36}  Nor should you swear by your head, since you cannot make a single hair either white or black. 
 \V{37}  Let your words be simply ‘Yes’ or ‘No’; anything beyond this comes from what is wrong. \indenting[yes]\par 
 \V{38}  You have heard that it was said — ‘An eye for an eye and a tooth for a tooth.’ 
 \V{39}  I, however, say to you that you must not resist wrong; but, if any one strikes you on the right cheek, turn the other to him also; 
 \V{40}  And, when any one wants to go to law with you, to take your coat, let him have your cloak as well; 
 \V{41}  And, if any one compels you to go one mile, go two miles with him. 
 \V{42}  Give to him who asks of you; and, from him who wants to borrow from you, do not turn away. \indenting[yes]\par 
 \V{43}  You have heard that it was said — ‘Thou shalt love thy neighbor and hate thy enemy.’ 
 \V{44}  I, however, say to you — Love your enemies, and pray for those who persecute you, 
 \V{45}  That you may become Sons of your Father who is in Heaven; for he causes his sun to rise upon bad and good alike, and sends rain upon the righteous and upon the unrighteous. 
 \V{46}  For, if you love only those who love you, what reward will you have? Even the tax-gatherers do this! 
 \V{47}  And, if you show courtesy to your brothers only, what are you doing more than others? Even the Gentiles do this! 
 \V{48}  You, then, must become perfect — as your heavenly Father is perfect. \indenting[yes]\par 
 \C{6}  Take care not to perform your religious duties in public in order to be seen by others; if you do, your Father who is in Heaven has no reward for you. 
 \V{2}  Therefore, when you do acts of charity, do not have a trumpet blown in front of you, as hypocrites do in the Synagogues and in the streets, that they may be praised by others. There, I tell you, is their reward! 
 \V{3}  But, when you do acts of charity, do not let your left hand know what your right hand is doing, 
 \V{4}  So that your charity may be secret; and your Father, who sees what is in secret, will recompense you. \indenting[yes]\par 
 \V{5}  And, when you pray, you are not to behave as hypocrites do. They like to pray standing in the Synagogues and at the corners of the streets, that they may be seen by men. There, I tell you, is their reward! 
 \V{6}  But, when one of you prays, let him go into his own room, shut the door, and pray to his Father who dwells in secret; and his Father, who sees what is secret, will recompense him. 
 \V{7}  When praying, do not repeat the same words over and over again, as is done by the Gentiles, who think that by using many words they will obtain a hearing. 
 \V{8}  Do not imitate them; for God, your Father, knows what you need before you ask him. 
 \V{9}  You, therefore, should pray thus —  \startnarrower[1*left,1*right]\indenting[no] ‘Our Father, who art in Heaven, \stopnarrower\startnarrower[1*left,1*right]\indenting[no] May thy name be held holy, \stopnarrower\startnarrower[1*left,1*right]\indenting[no]
 \V{10}  Thy Kingdom come, thy will be done —   \stopnarrower\startnarrower[2*left,1*right]\indenting[no] on earth, as in Heaven. \stopnarrower\startnarrower[1*left,1*right]\indenting[no]
 \V{11}  Give us to-day  \stopnarrower\startnarrower[2*left,1*right]\indenting[no] the bread that we shall need; \stopnarrower\startnarrower[1*left,1*right]\indenting[no]
 \V{12}  And forgive us our wrong-doings,  \stopnarrower\startnarrower[2*left,1*right]\indenting[no] as we have forgiven those who have wronged us; \stopnarrower\startnarrower[1*left,1*right]\indenting[no]
 \V{13}  And take us not into temptation,  \stopnarrower\startnarrower[2*left,1*right]\indenting[no] but deliver us from Evil.’ 
 \V{14}  For, if you forgive others their offences, your heavenly Father will forgive you also; 
 \V{15}  But, if you do not forgive others their offences, not even your Father will forgive your offences. \stopnarrower\indenting[yes]\par 
 \V{16}  And, when you fast, do not put on gloomy looks, as hypocrites do who disfigure their faces that they may be seen by men to be fasting. That, I tell you, is their reward! 
 \V{17}  But, when one of your fasts, let him anoint his head and wash his face, 
 \V{18}  That he may not be seen by men to be fasting, but by his Father who dwells in secret; and his Father, who sees what is secret, will recompense him. \indenting[yes]\par 
 \V{19}  Do not store up treasures for yourselves on earth, where moth and rust destroy, and where thieves break in and steal. 
 \V{20}  But store up treasures for yourselves in Heaven, where neither moth nor rust destroys, and where thieves do not break in or steal. 
 \V{21}  For where your treasure is, there will your heart be also. 
 \V{22}  The lamp of the body is the eye. If your eye is unclouded, your whole body will be lit up; 
 \V{23}  But, if your eye is diseased, your whole body will be darkened. And, if the inner light is darkness, how intense must that darkness be! 
 \V{24}  No one can serve two masters, for either he will hate one and love the other, or else he will attach himself to one and despise the other. You cannot serve both God and Money. \indenting[yes]\par 
 \V{25}  That is why I say to you, Do not be anxious about your life here — what you can get to eat or drink; nor yet about your body —  what you can get to wear. Is not life more than food, and the body than its clothing? 
 \V{26}  Look at the wild birds — they neither sow, nor reap, nor gather into barns; and yet your heavenly Father feeds them! And are not you more precious than they? 
 \V{27}  But which of you, by being anxious, can prolong his life a single moment? 
 \V{28}  And why be anxious about clothing? Study the wild lilies, and how they grow. They neither toil nor spin; 
 \V{29}  Yet I tell you that even Solomon in all his splendor was not robed like one of these. 
 \V{30}  If God so clothes even the grass of the field, which is living to-day and to-morrow will be thrown into the oven, will not he much more clothe you, O men of little faith? 
 \V{31}  Do not then ask anxiously ‘What can we get to eat?’ or ‘What can we get to drink?’ or ‘What can we get to wear?’ 
 \V{32}  All these are the things for which the nations are seeking, and your heavenly Father knows that you need them all. 
 \V{33}  But first seek his Kingdom and the righteousness that he requires, and then all these things shall be added for you. 
 \V{34}  Therefore do not be anxious about to-morrow, for to-morrow will bring its own anxieties. Every day has trouble enough of its own. \indenting[yes]\par 
 \C{7}  Do not judge, that you may not be judged. 
 \V{2}  For, just as you judge others, you will yourselves be judged, and the measure that you mete will be meted out to you. 
 \V{3}  And why do you look at the straw in your brother’s eye, while you pay no attention at all to the beam in yours? 
 \V{4}  How will you say to your brother ‘Let me take out the straw from your eye,’ when all the time there is a beam in your own? 
 \V{5}  Hypocrite! Take out the beam from your own eye first, and then you will see clearly how to take out the straw from your brother’s. \indenting[yes]\par 
 \V{6}  Do not give what is sacred to dogs; nor yet throw your pearls before pigs, lest they should trample them under their feet, and then turn and attack you. 
 \V{7}  Ask, and your prayer shall be granted; search, and you shall find; knock, and the door shall be opened to you. 
 \V{8}  For he that asks receives, he that searches finds, and to him that knocks the door shall be opened. 
 \V{9}  Who among you, when his son asks him for a loaf, will give him a stone, 
 \V{10}  Or when he asks for a fish, will give him a snake? 
 \V{11}  If you, then, wicked though you are, know how to give good gifts to your children, how much more will your Father who is in Heaven give what is good to those that ask him! \indenting[yes]\par 
 \V{12}  Do to others whatever you would wish them to do to you; for that is the teaching of both the Law and the Prophets. 
 \V{13}  Go in by the small gate. Broad and spacious is the road that leads to destruction, and those that go in by it are many; 
 \V{14}  For small is the gate, and narrow the road, that leads to Life, and those that find it are few. \indenting[yes]\par 
 \V{15}  Beware of false Teachers — men who come to you in the guise of sheep, but at heart they are ravenous wolves. 
 \V{16}  By the fruit of their lives you will know them. Do people gather grapes from thorn bushes, or figs from thistles? 
 \V{17}  So, too, every sound tree bears good fruit, while a worthless tree bears bad fruit. 
 \V{18}  A sound tree cannot produce bad fruit, nor can a worthless tree bear good fruit. 
 \V{19}  Every tree that fails to bear good fruit is cut down and thrown into the fire. 
 \V{20}  Hence it is by the fruit of their lives that you will know such men. 
 \V{21}  Not every one who says to me ‘Master! Master!’ will enter the Kingdom of Heaven, but only he who does the will of my Father who is in Heaven. 
 \V{22}  On ‘That Day’ many will say to me ‘Master, Master, was not it in your name that we taught, and in your name that we drove out demons, and in your name that we did many miracles?’ 
 \V{23}  And then I shall say to them plainly ‘I never knew you. Go from my presence, you who live in sin.’ \indenting[yes]\par 
 \V{24}  Everyone, therefore, that listens to this teaching of mine and acts upon it may be compared to a prudent man, who built his house upon the rock. 
 \V{25}  The rain poured down, the rivers rose, the winds blew and beat upon that house, but it did not fall, for its foundations were upon the rock. 
 \V{26}  And every one that listens to this teaching of mine and does not act upon it may be compared to a foolish man, who built his house on the sand. 
 \V{27}  The rain poured down, the rivers rose, the winds blew and struck against that house, and it fell; and great was its downfall.”  \indenting[yes]\par 
 \V{28}  By the time that Jesus had finished speaking, the crowd was filled with amazement at his teaching. 
 \V{29}  For he taught them like one who had authority, and not like their Teachers of the Law. \blank\indenting[no]\par 
 \C{8}  \CapStretch{When Jesus had come down from the hill}, great crowds followed him. 
 \V{2}  And he saw a leper who came up, and bowed to the ground before him, and said: “Master, if only you are willing, you are able to make me clean.” 
 \V{3}  Stretching out his hand, Jesus touched him, saying as he did so:   “I am willing; become clean.”   Instantly he was made clean from his leprosy; 
 \V{4}  And then Jesus said to him:   “Be careful not to say a word to any one, but go and show yourself to the Priest, and offer the gift directed by Moses, as evidence of your cure.”  
 \V{5}  After Jesus had entered Capernaum, a Captain in the Roman army came up to him, entreating his help. 
 \V{6}  “Sir,” he said, “my manservant is lying ill at my house with a stroke of paralysis, and is suffering terribly.” 
 \V{7}  “I will come and cure him,”   answered Jesus. 
 \V{8}  “Sir,” the Captain went on, “I am unworthy to receive you under my roof; but only speak, and my manservant will be cured. 
 \V{9}  For I myself am a man under the orders of others, with soldiers under me; and, if I say to one of them ‘Go,’ he goes, and to another ‘Come,’ he comes, and to my slave ‘Do this,’ he does it.” 
 \V{10}  Jesus was surprised to hear this, and said to those who were following him:   “Never I tell you, in any Israelite have I met with such faith as this! 
 \V{11}  Yes, and many will come in from East and West and take their places beside Abraham, Isaac, and Jacob, in the Kingdom of Heaven; 
 \V{12}  While the heirs to the Kingdom will be ‘banished into the darkness’ outside; there, there will be weeping and grinding of teeth.”  
 \V{13}  Then Jesus said to the Captain:   “Go now, and it shall be according to your faith.”   And the man was cured that very hour. \indenting[yes]\par 
 \V{14}  When Jesus went into Peter’s house, he saw Peter’s mother-in- law prostrated with fever. 
 \V{15}  On his taking her hand, the fever left her, and she rose and began to wait upon him. 
 \V{16}  In the evening the people brought to Jesus many who were possessed by demons; and he drove out the spirits with a word, And cured all who were ill, 
 \V{17}  In fulfillment of these words in the Prophet Isaiah — ‘He took our infirmities on himself, and bore the burden of our diseases.’ \indenting[yes]\par 
 \V{18}  Seeing a crowd round him, Jesus gave orders to go across. 
 \V{19}  And a Teacher of the Law came up to him, and said: “Teacher, I will follow you wherever you go.” 
 \V{20}  “Foxes have holes,”   answered Jesus,   “and wild birds their roosting-places, but the Son of Man has nowhere to lay his head.”  
 \V{21}  “Master,” said another, who was a disciple, “let me first go and bury my father.” 
 \V{22}  But Jesus answered:   “Follow me, and leave the dead to bury their dead.”  
 \V{23}  Then he got into the boat, followed by his disciples. 
 \V{24}  Suddenly so great a storm came on upon the Sea, that the waves broke right over the boat. But Jesus was asleep; 
 \V{25}  And the disciples came and roused him. “Master,” they cried, “save us; we are lost!” 
 \V{26}  “Why are you so timid?”   he said.   “O men of little faith!”   Then Jesus rose and rebuked the winds and the sea, and a great calm followed. 
 \V{27}  The men were amazed, and exclaimed: “What kind of man is this, that even the winds and the sea obey him!” \indenting[yes]\par 
 \V{28}  And on getting to the other side — the country of the Gadarenes — Jesus met two men who were possessed by demons, coming out of the tombs. They were so violent that no one was able to pass that way. 
 \V{29}  Suddenly they shrieked out: “What do you want with us, Son of God? Have you come here to torment us before our time?” 
 \V{30}  A long way off, there was a drove of many pigs, feeding; 
 \V{31}  And the foul spirits began begging Jesus: “If you drive us out, send us into the drove of pigs.” 
 \V{32}  “Go,”   he said. The spirits came out, and entered the pigs; and the whole drove rushed down the steep slope into the Sea, and died in the water. 
 \V{33}  At this the men who tended them ran away and went to the town, carrying the news of all that had occurred, and of what had happened to the possessed men. 
 \V{34}  At the news the whole town went out to meet Jesus, and, when they saw him, they entreated him to go away from their neighborhood. \indenting[yes]\par 
 \C{9}  Afterwards Jesus got into a boat, and, crossing over, came to his own city. 
 \V{2}  And there some people brought to him a paralyzed man on a bed. When Jesus saw their faith, he said to the man:   “Courage, Child! Your sins are forgiven.”  
 \V{3}  Then some of the teachers of the Law said to themselves: “This man is blaspheming!” 
 \V{4}  Knowing their thoughts, Jesus exclaimed:   “Why do your cherish such wicked thoughts? 
 \V{5}  Which, I ask, is the easier? — to say ‘Your sins are forgiven’? or to say ‘Get up, and walk about’? 
 \V{6}  But, that you may know that the Son of Man has power on earth to forgive sins”   — then he said to the paralyzed man —   “Get up, take up your bed, and return to your home.”  
 \V{7}  The man got up and went to his home. 
 \V{8}  When the crowd saw this, they were awe-struck, and praised God for giving such power to men. \indenting[yes]\par 
 \V{9}  As Jesus went along, he saw a man, called Matthew, sitting in the tax-office, and said to him:   “Follow me.”   Matthew got up and followed him. \indenting[yes]\par 
 \V{10}  And, later on, when he was at table in the house, a number of tax-gatherers and outcasts came in and took their places at table with Jesus and his disciples. 
 \V{11}  When the Pharisees saw this, they said to his disciples: “Why does your Teacher eat in the company of tax-gatherers and outcasts?” 
 \V{12}  On hearing this, Jesus said:   “It is not those who are in health that need a doctor, but those who are ill. 
 \V{13}  Go and learn what this means — ‘I desire mercy, and not sacrifice’; for I did not come to call the religious, but the outcast.”  
 \V{14}  Then John’s disciples came to Jesus, and asked: “Why do we and the Pharisees fast while your disciples do not?” 
 \V{15}  Jesus answered:   “Can the bridegroom’s friends mourn as long as the bridegroom is with them? But the days will come, when the bridegroom will be parted from them, and they will fast then. 
 \V{16}  No man ever puts a piece of unshrunk cloth on an old garment; for such a patch tears away from the garment, and a worse rent is made. 
 \V{17}  Nor do people put new wine into old wine-skins; for, if they do, the skins burst, and the wine runs out, and the skins are lost; but they put new wine into fresh skins, and so both are preserved.”  \indenting[yes]\par 
 \V{18}  While Jesus was saying this, a President of a Synagogue came up and bowed to the ground before him. “My daughter,” he said, “Has just died; but come and place your hand on her, and she will be restored to life.” 
 \V{19}  So Jesus rose and followed him, and his disciples went also. 
 \V{20}  But meanwhile a woman, who had been suffering from hemorrhage for twelve years, came up behind and touched the tassel of his cloak. 
 \V{21}  “If I only touch his cloak,” she said to herself, “I shall get well.” 
 \V{22}  Turning and seeing her, Jesus said:   “Courage, Daughter! your faith has delivered you.”   And the woman was delivered from her malady from that very hour. 
 \V{23}  When Jesus reached the President’s house, seeing the flute- players, and a number of people all in confusion, 
 \V{24}  He said:   “Go away, the little girl is not dead; she is asleep.”   They began to laugh at him; 
 \V{25}  But, when the people had been sent out, Jesus went in, and took the little girl’s hand, and she rose. 
 \V{26}  The report of this spread through all that part of the country. \indenting[yes]\par 
 \V{27}  As Jesus was passing on from there, he was followed by two blind men, who kept calling out: “Take pity on us, Son of David!” 
 \V{28}  When he had gone indoors, the blind men came up to him; and Jesus asked them:   “Do you believe that I am able to do this?”   “Yes, Master!” they answered. 
 \V{29}  Upon that he touched their eyes, and said:   “It shall be according to your faith.”  
 \V{30}  Then their eyes were opened. Jesus sternly cautioned them.   “See that no one knows of it,”   he said. 
 \V{31}  But the men went out, and spread the news about him through all that part of the country. 
 \V{32}  Just as they were going out, some people brought up to Jesus a dumb man who was possessed by a demon; 
 \V{33}  And, as soon as the demon had been driven out, the dumb man spoke. The people were astonished at this, and exclaimed: “Nothing like this has ever been seen in Israel!” 
 \V{34}  But the Pharisees said: “He drives out the demons by the help of the chief of the demons.” \indenting[yes]\par 
 \V{35}  Jesus went round all the towns and the villages, teaching in their Synagogues, proclaiming the Good News of the Kingdom, and curing every kind of disease and every kind of sickness. 
 \V{36}  But, when he saw the crowds, his heart was moved with compassion for them, because they were distressed and harassed, ‘like sheep without a shepherd’; 
 \V{37}  And he said to his disciples:   “The harvest is abundant, but the laborers are few. 
 \V{38}  Therefore pray to the Owner of the harvest to send laborers to gather in his harvest.”  \indenting[yes]\par 
 \C{10}  Calling his twelve Disciples to him, Jesus gave them authority over foul spirits, so that they could drive them out, as well as the power of curing every kind of disease and every kind of sickness. 
 \V{2}  The names of the twelve Apostles are these: First Simon, also known as Peter, and his brother Andrew; James the son of Zebediah, and his brother John; 
 \V{3}  Philip and Bartholomew; Thomas and Matthew the tax-gather; James the son of Alphaeus, and Thaddaeus; 
 \V{4}  Simon the Zealot, and Judas Iscariot — the Apostle who betrayed him. \indenting[yes]\par 
 \V{5}  These twelve Jesus sent out as his Messengers, after giving them these instructions —   “Do not go to the Gentiles, nor enter any Samaritan town, 
 \V{6}  But make your way rather to the lost sheep of Israel. 
 \V{7}  And on your way proclaim that the Kingdom of Heaven is at hand. 
 \V{8}  Cure the sick, raise the dead, make the lepers clean, drive out demons. You have received free of cost, give free of cost. 
 \V{9}  Do not provide yourselves with gold, or silver, or pence in your purses; 
 \V{10}  Not even with a bag for the journey, or a change of clothes, or sandals, or even a staff; for the worker is worth his food. 
 \V{11}  Whatever town or village you visit, find out who is worthy in that place, and remain there till you leave. 
 \V{12}  As you enter the house, greet it. 
 \V{13}  Then, if the house is worthy, let your blessing rest upon it, but, if it is unworthy, let your blessing return upon yourselves. 
 \V{14}  If no one welcomes you, or listens to what you say, as you leave that house or that town, shake off its dust from your feet. 
 \V{15}  I tell you, the doom of the land of Sodom and Gomorrah will be more bearable in the ‘Day of Judgment’ than the doom of that town. \indenting[yes]\par 
 \V{16}  Remember, I am sending you out as my Messengers like sheep among wolves. So be as wise as serpents, and as blameless as doves. 
 \V{17}  Be on your guard against your fellow men, for they will betray you to courts of law, and scourge you in their Synagogues; 
 \V{18}  And you will be brought before governors and kings for my sake, that you may witness for me before them and the nations. 
 \V{19}  Whenever they betray you, do not be anxious as to how you shall speak or what you shall say, for what you shall say will be given you at the moment; 
 \V{20}  For it will not be you who speak, but the Spirit of your Father that speaks within you. 
 \V{21}  Brother will betray brother to death, and the father his child; and children will turn against their parents, and cause them to be put to death; 
 \V{22}  And you will be hated by every one on account of my Name. Yet the man that endures to the end shall be saved. 
 \V{23}  But, when they persecute you in one town, escape to the next; for, I tell you, you will not have come to the end of the towns of Israel before the Son of Man comes. 
 \V{24}  A scholar is not above his teacher, nor a servant above his master. 
 \V{25}  It is enough for a scholar to be treated like his teacher, and a servant like his master. If the head of the house has been called Baal-zebub, how much more the members of his household! 
 \V{26}  Do not, therefore, be afraid of them. There is nothing concealed which will not be revealed, nor anything hidden which will not become known. 
 \V{27}  What I tell you in the dark, say again in the light; and what is whispered in your ear, proclaim upon the housetops. 
 \V{28}  And do not be afraid of those who kill the body, but are unable to kill the soul; rather be afraid of him who is able to destroy both soul and body in the Pit. 
 \V{29}  Are not two sparrows sold for a half-penny? Yet not one of them will fall to the ground without your Father’s knowledge. 
 \V{30}  While as for you, the very hairs of your head are numbered. 
 \V{31}  Do not, therefore, be afraid; you are of more value than many sparrows. 
 \V{32}  Every one, therefore, who shall acknowledge me before his fellow men, I, too, will acknowledge before my Father who is in Heaven; 
 \V{33}  But, if any one disowns me before his fellow men, I, too, will disown him before my Father who is in Heaven. \indenting[yes]\par 
 \V{34}  Do not imagine that I have come to bring peace upon the earth. I have come to bring, not peace, but the sword. 
 \V{35}  For I have come to set — ‘a man against his father, and a daughter against her mother, and a daughter-in-law against her mother-in-law. 
 \V{36}  A man’s enemies will be the members of his own household.’ 
 \V{37}  He who loves father or mother more than me is not worthy of me; and he who loves son or daughter more than me is not worthy of me. 
 \V{38}  And the man who does not take his cross and follow in my steps is not worthy of me. 
 \V{39}  He who has found his life will lose it, while he who, for my sake, has lost his life shall find it. \indenting[yes]\par 
 \V{40}  He who welcomes you is welcoming me; and he who welcomes me is welcoming him who sent me as his Messenger. 
 \V{41}  He who welcomes a Prophet, because he is a Prophet, shall receive a Prophet’s reward; and he who welcomes a good man, because he is a good man, shall receive a good man’s reward. 
 \V{42}  And, if any one gives but a cup of cold water to one of these lowly ones because he is a disciple, I tell you that he shall assuredly not lose his reward.”  \indenting[yes]\par 
 \C{11}  After Jesus had finished giving directions to his twelve Disciples, he left that place in order to teach and preach in their towns. \indenting[yes]\par 
 \V{2}  Now John had heard in prison what the Christ was doing, and he sent a message by his disciples, 
 \V{3}  And asked — ” Are you ‘The Coming One,’ or are we to look for someone else?” 
 \V{4}  The answer of Jesus to the question was —   “Go and report to John what you hear and see —  
 \V{5}  The blind recover their sight and the lame walk, the lepers are made clean and the deaf hear, the dead, too, are raised to life, and the good news is told to the poor. 
 \V{6}  And blessed is the man who finds no hindrance in me.”  \indenting[yes]\par 
 \V{7}  While these men were going back, Jesus began to say to the crowds with reference to John: 
 \V{8}  “What did you go out into the Wilderness to look at? A reed waving in the wind? If not, what did you go out to see? A man richly dressed? Why, those who wear rich things are to be found in the courts of kings! 
 \V{9}  What, then, did you go for? To see a Prophet? Yes, I tell you, and far more than a Prophet. 
 \V{10}  This is the man of whom Scripture says — ‘Behold, I am myself sending my Messenger before thy face, And he shall prepare thy way before thee.’ 
 \V{11}  I tell you, no one born of a woman has yet appeared who is greater than John the Baptist; and yet the lowliest in the Kingdom of Heaven is greater than he. 
 \V{12}  From the time of John the Baptist to this very hour, the Kingdom of Heaven has been taken by force, and men using force have been seizing it. 
 \V{13}  For the teaching of all the Prophets and of the Law continued till the time of John; 
 \V{14}  And — if you are ready to accept it — John is himself the Elijah who was destined to come. 
 \V{15}  Let him who has ears hear. 
 \V{16}  But to what shall I compare the present generation? It is like little children sitting in the market-places and calling out to their playmates —  
 \V{17}  ‘We have played the flute for you, but you have not danced; We have wailed, but you have not mourned.’ 
 \V{18}  For, when John came, neither eating nor drinking, men said ‘He has a demon in him’; 
 \V{19}  And now that the Son of Man has come, eating and drinking, they are saying ‘Here is a glutton and a wine-drinker, a friend of tax-gatherers and outcasts!’ And yet Wisdom is vindicated by her actions.”  \indenting[yes]\par 
 \V{20}  Then Jesus began to reproach the towns in which most of his miracles had been done, because they had not repented: 
 \V{21}  “Alas for you, Chorazin! Alas for you, Bethsaida! For, if the miracles which were done in you had been done in Tyre and Sidon, they would have repented long ago in sackcloth and ashes. 
 \V{22}  Yet, I tell you, the doom of Tyre and Sidon will be more bearable in the ‘Day of Judgment’ than yours. 
 \V{23}  And you, Capernaum! Will you ‘exalt yourself to Heaven’? ‘You shall go down to the Place of Death.’ For, if the miracles which have been done in you had been done in Sodom, it would have been standing to this day. 
 \V{24}  Yet, I tell you, the doom of Sodom will be more bearable in the ‘Day of Judgment’ than yours.”  
 \V{25}  At that same time Jesus uttered the words:   “I thank thee, Father, Lord of Heaven and earth, that, though thou has hidden these things from the wise and learned, thou hast revealed them to the child-like! 
 \V{26}  Yes, Father, I thank thee that this has seemed good to thee. 
 \V{27}  Everything has been committed to me by my Father; nor does any one fully know the Son, except the Father, or fully know the Father, except the Son and those to whom the Son may choose to reveal him. 
 \V{28}  Come to me, all you who toil and are burdened, and I will give you rest! 
 \V{29}  Take my yoke upon you, and learn from me, for I am gentle and lowly-minded, and ‘you shall find rest for your souls’; 
 \V{30}  For my yoke is easy, and my burden is light.”  
 \C{12} \indenting[yes]\par  About the same time Jesus walked through the corn-fields one Sabbath. His disciples were hungry, and began to pick some ears of wheat and eat them. 
 \V{2}  But, when the Pharisees saw this, they said: “Look! your disciples are doing what it is not allowable to do on a Sabbath!” 
 \V{3}  “Have not you read,”   replied Jesus,   “what David did, when he and his companions were hungry —  
 \V{4}  How he went into the House of God, and how they ate the consecrated bread, through it was not allowable for him or his companions to eat it, but only for the priests? 
 \V{5}  And have not you read in the law that, on the Sabbath, the priest in the Temple break the Sabbath and yet are not guilty? 
 \V{6}  Here, however, I tell you, there is something greater than the Temple! 
 \V{7}  And had you learned the meaning of the words — ‘I desire mercy, and not sacrifice,’ you would not have condemned those who are not guilty. 
 \V{8}  For the Son of Man is lord of the Sabbath.”  \indenting[yes]\par 
 \V{9}  Passing on, Jesus went into their Synagogue, 
 \V{10}  And there he saw a man with a withered hand. Some people asked Jesus whether it was allowable to work a cure on the Sabbath- -so that they might have a charge to bring against him. 
 \V{11}  But Jesus said to them:   “Which of you, if he had only one sheep, and that sheep fell into a pit on the Sabbath, would not lay hold of it and pull it out? 
 \V{12}  And how much more precious a man is than a sheep! Therefore it is allowable to do good on the Sabbath.”  
 \V{13}  Then he said to the man.   “Stretch out your hand.”   The man stretched it out; and it had become as sound as the other. 
 \V{14}  On coming out, the Pharisees plotted against Jesus, to put him to death. \indenting[yes]\par 
 \V{15}  Jesus, however, became aware of it, and went away from that place. A number of people followed him, and he cured them all; 
 \V{16}  but he warned them not to make him known, 
 \V{17}  in fulfillment of these words in the Prophet Isaiah —  
 \V{18}  ‘Behold! the Servant of my Choice, My Beloved, in whom my heart delights! I will breathe my spirit upon him, And he shall announce a time of judgment to the Gentiles. 
 \V{19}  He shall not contend, nor cry aloud, Neither shall any one hear his voice in the streets; 
 \V{20}  A bruised reed he will not break, And a smoldering wick he will not quench, Till he has brought the judgment to a victorious issue, 
 \V{21}  And on his name shall the Gentiles rest their hopes.” \indenting[yes]\par 
 \V{22}  Then some people brought to Jesus a possessed man, who was blind and dumb; and he cured him, so that the man who had been dumb both talked and saw. 
 \V{23}  At this all the people were astounded. “Is it possible that this is the son of David?” they exclaimed. 
 \V{24}  But the Pharisees heard of it and said: “He drives out demons only by the help of Baal-zebub the chief of the demons.” 
 \V{25}  Jesus, however, was aware of what was passing in their minds, and said to them:   “Any kingdom divided against itself becomes a desolation, and any town or household divided against itself will not last. 
 \V{26}  So, if Satan drives Satan out, he must be divided against himself; and how, then, can his kingdom last? 
 \V{27}  And, if it is by Baal-zebub’s help that I drive out demons, by whose help is it that your own sons drive them out? Therefore they shall themselves be your judges. 
 \V{28}  But, if it is by the help of the Spirit of God that I drive out demons, then the Kingdom of God must already be upon you. 
 \V{29}  How, again, can any one get into a strong man’s house and carry off his goods, without first securing him? And not till then will he plunder his house. 
 \V{30}  He who is not with me is against me, and he who does not help me to gather is scattering. 
 \V{31}  Therefore, I tell you, men will be forgiven every sin and slander; but slander against the Holy Spirit will not be forgiven. 
 \V{32}  Whoever speaks against the Son of Man will be forgiven, but whoever speaks against the Holy Spirit will not be forgiven, either in the present age, or in the age to come. \indenting[yes]\par 
 \V{33}  You must assume either that both tree and fruit are good, or that both tree and fruit are worthless; since it is by it’s fruits that a tree is known. 
 \V{34}  You brood of vipers! how can you, evil as you are, say anything good? For what fills the heart will rise to the lips. 
 \V{35}  A good man, from his good stores, produces good things; while an evil man, from his evil stores, produces evil things. 
 \V{36}  I tell you that for every careless thing that men say, they must answer on the ‘Day of Judgment.’ 
 \V{37}  For it is by your words that you will be acquitted, and by your words that you will be condemned.”  \indenting[yes]\par 
 \V{38}  At this point, some Teachers of the Law and Pharisees interposed. “Teacher,” they said, “ we want to see some sign from you.” 
 \V{39}  “It is a wicked and unfaithful generation,”   answered Jesus,   “that is asking for a sign, and no sign shall be given it except the sign of the Prophet Jonah. 
 \V{40}  For, just as ‘Jonah was inside the sea-monster three days and three nights,’ so shall the Son of Man be three days and three nights in the heart of the earth. 
 \V{41}  At the Judgment, the men of Nineveh will stand up with this generation, and will condemn it, because they repented at Jonah’s proclamation; and here is more than a Jonah! 
 \V{42}  At the Judgment the Queen of the South will rise up with the present generation, and will condemn it, because she came from the very ends of the earth to listen to the wisdom of Solomon; and here is more than a Solomon! 
 \V{43}  No sooner does a foul spirit leave a man, than it passes through places where there is no water, in search of rest, and does not find it. 
 \V{44}  Then it says ‘I will go back to the home which I left’; but, on coming there, it finds it unoccupied, and swept, and put in order. 
 \V{45}  Then it goes and brings with it seven other spirits more wicked than itself, and they go in, and make their home there; and the last state of that man proves to be worse than the first. So, too, will it be with this wicked generation.”  \indenting[yes]\par 
 \V{46}  While he was still speaking to the crowds, his mother and brothers were standing outside, asking to speak to him. 
 \V{47}  Someone told him this, and Jesus replied: 
 \V{48}  “Who is my mother? and who are my brothers?”  
 \V{49}  Then, stretching out his hands towards his disciples, he said:   “Here are my mother and my brothers! 
 \V{50}  For any one who does the will of my Father who is in Heaven is my brother and sister and mother.”  \indenting[yes]\par 
 \C{13}  That same day, when Jesus had left the house and was sitting by the Sea, 
 \V{2}  such great crowds gathered round him, that he got into a boat, and sat in it, while all the people stood upon the beach. 
 \V{3}  Then he told them many truths in parables.   “The sower,”   he began,   “went out to sow; and, 
 \V{4}  As he was sowing, some seed fell along the path, and the birds came and ate it up. 
 \V{5}  Some fell on rocky places, where it had not much soil, and, having no depth of soil, sprang up at once. 
 \V{6}  As soon as the sun had risen, it was scorched, and, having no root, withered away. 
 \V{7}  Some, again, fell into the brambles; but the brambles shot up and choked it. 
 \V{8}  Some, however, fell on good soil, and yielded a return, sometimes one hundred, sometimes sixty, sometimes thirty fold. 
 \V{9}  Let him who has ears hear.”  \indenting[yes]\par 
 \V{10}  Afterwards his disciples came to him, and said: “Why do you speak to them in parables?” 
 \V{11}  “To you,”   answered Jesus,   “the knowledge of the hidden truths of the Kingdom of Heaven has been imparted, but not to those. 
 \V{12}  For, to all who have, more will be given, and they shall have abundance; but, from all who have nothing, even what they have will be taken away. 
 \V{13}  That is why I speak to them in parables, because, though they have eyes, they do not see, and though they have ears, they do not hear or understand. 
 \V{14}  And in them is being fulfilled that prophecy of Isaiah which says — ‘You will hear with your ears without ever understanding, And, though you have eyes, you will see without ever perceiving, 
 \V{15}  For the mind of this nation has grown dense, And their ears are dull of hearing, Their eyes also have they closed; Lest some day they should perceive with their eyes, And with their ears they should hear, And in their mind they should understand, and should turn — And I should heal them.’ 
 \V{16}  But blessed are your eyes, for they see, and your ears, for they hear; 
 \V{17}  For I tell you that many Prophets and good men have longed for the sight of the things which you are seeing, yet never saw them, and to hear the things which you are hearing, yet never heard them. \indenting[yes]\par 
 \V{18}  Listen, then, yourselves to the parable of the Sower. 
 \V{19}  When any one hears the Message of the Kingdom without understanding it, the Evil One comes and snatches away what has been sown in his mind. This is the man meant by the seed which was sown along the path. 
 \V{20}  By the seed which was sown on rocky places is meant the man who hears the Message, and at once accepts it joyfully; 
 \V{21}  But, as he has no root, he stands for only a short time; and, when trouble or persecution arises on account of the Message, he falls away at once. 
 \V{22}  By the seed which was sown among the brambles is meant the man who hears the Message, but the cares of life and the glamour of wealth completely choke the Message, so that it gives no return. 
 \V{23}  But by the seed which was sown on the good ground is meant the man who hears the Message and understands it, and really yields a return, sometimes one hundred, sometimes sixty, sometimes thirty fold.”  \indenting[yes]\par 
 \V{24}  Another parable which Jesus told them was this —   “The Kingdom of Heaven is compared to a man who sowed good seed in his field. 
 \V{25}  But, while every one was asleep, his enemy came and sowed tares among the wheat, and then went away. 
 \V{26}  So, when the blades of corn shot up, and came into ear, the tares made their appearance also. 
 \V{27}  On this the owner’s servants came to him, and said ‘Was not it good seed that you sowed in your field? Where, then, do the tares in it come from?’ 
 \V{28}  ‘An enemy has done this,’ was his answer. ‘Do you wish us, then,’ they asked,’ to go and gather them together?’ 
 \V{29}  ‘No,’ said he, ‘for fear that, while you are gathering the tares, you should root up the wheat as well. 
 \V{30}  Let both grow side by side till harvest; and then I shall say to the reapers, Gather the tares together first, and tie them in bundles for burning; but bring all the wheat into my barn.’”  \indenting[yes]\par 
 \V{31}  Another parable which he told them was this —   “The Kingdom of Heaven is like a mustard-seed, which a man took and sowed in his field. 
 \V{32}  This seed is smaller than all other seeds, but, when it has grown up, it is larger than the herbs and becomes a tree, so that ‘the wild birds come and roost in its branches.’”  \indenting[yes]\par 
 \V{33}  This was another parable which Jesus related —   “The Kingdom of Heaven is like some yeast which a woman took and covered up in three pecks of flour, until the whole had risen.”  
 \V{34}  Of all this Jesus spoke to the crowd in parables; indeed to them he used never to speak at all except in parables, 
 \V{35}  in fulfillment of these words in the Prophet — ‘I will speak to them in parables; I will utter things kept secret since the foundation of the world.’ \indenting[yes]\par 
 \V{36}  Then Jesus left the crowd, and went into the house. Presently his disciples came to him, and said: “Explain to us the parable of the tares in the field.” 
 \V{37}  And he answered:   “The sower of the good seed is the Son of Man. 
 \V{38}  The field is the world. By the good seed is meant the People of the Kingdom. The tares are the wicked, 
 \V{39}  And the enemy who sowed them is the Devil. The harvest-time is the close of the age, and the reapers are angels. 
 \V{40}  And, just as the tares are gathered and burnt, so it will be at the close of the age. 
 \V{41}  The Son of Man will send his angels, and they will gather from his kingdom all that hinders and those who live in sin, 
 \V{42}  And ‘will throw them into the fiery furnace,’ where there will be weeping and grinding of teeth. 
 \V{43}  Then shall the righteous shine, like the sun, in the Kingdom of their Father. Let him who has ears hear. \indenting[yes]\par 
 \V{44}  The Kingdom of Heaven is like a treasure hidden in a field, which a man found and hid again, and then, in his delight, went and sold everything that he had, and bought that field. \indenting[yes]\par 
 \V{45}  Again, the Kingdom of Heaven is like a merchant in search of choice pearls. 
 \V{46}  Finding one of great value, he went and sold everything that he had, and bought it. 
 \V{47}  Or again, the Kingdom of Heaven is like a net which was cast into the sea, and caught fish of all kinds. 
 \V{48}  When it was full, they hauled it up on the beach, and sat down and sorted the good fish into baskets, but threw the worthless ones away. 
 \V{49}  So will it be at the close of the age. The angels will go out and separate the wicked from the righteous, 
 \V{50}  And ‘will throw them into the fiery furnace,’ where there will be weeping and grinding of teeth. \indenting[yes]\par 
 \V{51}  Have you understood all this?”   Jesus asked. “Yes,” they answered. 
 \V{52}  Then he added:   “So every Teacher of the Law, who has received instruction about the Kingdom of Heaven, is like a householder who produces from his stores things both new and old.”  \indenting[yes]\par 
 \V{53}  When Jesus had finished these parables, he withdrew from that place. 
 \V{54}  Going to his own part of the country, he taught the people in their Synagogue in such a manner that they were deeply impressed. “Where did he get this wisdom?” they said, “and the miracles? 
 \V{55}  Is not he the carpenter’s son? Is not his mother called Mary, and his brothers James, and Joseph, and Simon, and Judas? 
 \V{56}  And his sisters, too — are not they all living among us? Where, then did he get all this?” 
 \V{57}  These things proved a hindrance to their believing in him; whereupon Jesus said:   “A prophet is not without honor, except in his own country and in his own house.”  
 \V{58}  And he did not work many miracles there, because of their want of faith. \indenting[yes]\par 
 \C{14}  At that time Prince Herod heard of the fame of Jesus, 
 \V{2}  And said to his attendants: “ This must be John the Baptist; he must be risen from the dead, and that is why these miraculous powers are active in him.” 
 \V{3}  For Herod had arrested John, put him in chains, and shut him up in prison, to please Herodias, the wife of Herod’s brother Philip. 
 \V{4}  For John had said to him ‘You have no right to be living with her.’ 
 \V{5}  Yet, though Herod wanted to put him to death, he was afraid of the people, because they looked on John as a Prophet. 
 \V{6}  But, when Herod’s birthday came, the daughter of Herodias danced before his guests, and so pleased Herod, 
 \V{7}  That he promised with an oath to give her whatever she asked. 
 \V{8}  Prompted by her mother, the girl said ‘Give me here, on a dish, the head of John the Baptist.’ 
 \V{9}  The king was distressed at this; yet, on account of his oath and of the guests at his table, he ordered it to be given her. 
 \V{10}  He sent and beheaded John in the prison; 
 \V{11}  And his head was brought on a dish and given to the girl, and she took it to her mother. 
 \V{12}  Then John’s disciples came, and took the body away, and buried it; and went and told Jesus. \indenting[yes]\par 
 \V{13}  When Jesus heard of it, he retired privately in a boat to a lonely spot. The people, however, heard of his going, and followed him in crowds from the towns on foot. 
 \V{14}  On getting out of the boat, Jesus saw a great crowd, and his heart was moved at the sight of them; and he cured all the sick among them. 
 \V{15}  In the evening the disciples came up to him, and said: “ This is a lonely spot, and the day is now far advanced; send the crowds away, that they may go to the villages, and buy themselves food.” 
 \V{16}  But Jesus said:   “They need not go away, it is for you to give them something to eat.”  
 \V{17}  “We have nothing here,” they said, “except five loaves and two fishes.” 
 \V{18}  “Bring them here to me,”   was his reply. 
 \V{19}  Jesus ordered the people to take their seats on the grass; and, taking the five loaves and the two fishes, he looked up to Heaven, and said the blessing, and, after he had broken the loaves, gave them to his disciples; and they gave them to the crowds. 
 \V{20}  Every one had sufficient to eat, and they picked up enough of the broken pieces that were left to fill twelve baskets. 
 \V{21}  The men who ate were about five thousand in number, without counting women and children. 
 \V{22}  Immediately afterwards Jesus made the disciples get into a boat and cross over in advance of him, while he dismissed the crowds. 
 \V{23}  After dismissing the crowds, he went up the hill by himself to pray; and, when evening fell, he was there alone. 
 \V{24}  The boat was by this time some miles from shore, laboring in the waves, for the wind was against her. 
 \V{25}  Three hours after midnight, however, Jesus came towards the disciples, walking on the water. 
 \V{26}  But, when they saw him walking on the water, they were terrified. “It is a ghost,” they exclaimed, and cried out for fear. 
 \V{27}  But Jesus at once spoke to them.   “Courage!”   he said,   “It is I; do not be afraid!”  
 \V{28}  “Master,” Peter exclaimed, “if it is you, tell me to come to you on the water.” 
 \V{29}  And Jesus said:   “Come.”   So Peter got down from the boat, and walked on the water, and went towards Jesus; 
 \V{30}  But, when he felt the wind, he was frightened, and, beginning to sink, cried out: “Master! Save me!” 
 \V{31}  Instantly Jesus stretched out his hand, and caught hold of him.   “O man of little faith!”   he said,   “Why did you falter?”  
 \V{32}  When they had got into the boat, the wind dropped. 
 \V{33}  But the men in the boat threw themselves on their faces before him, and said: “You are indeed God’s Son.” \indenting[yes]\par 
 \V{34}  When they had crossed over, they landed at Gennesaret. 
 \V{35}  But the people of that place, recognizing Jesus, sent out to the whole country round, and brought to him all who were ill, 
 \V{36}  Begging him merely to let them touch the tassel of his cloak; and all who touched were made perfectly well. \indenting[yes]\par 
 \C{15}  Then some Pharisees and Teachers of the Law came to Jesus, and said: 
 \V{2}  “How is it that your disciples break the traditions of our ancestors? For they do not wash their hands when they eat food.” 
 \V{3}  His reply was:   “How is it that you on your side break God’s commandments out of respect for your own traditions? 
 \V{4}  For God said — ‘Honor thy father and mother,’ and ‘Let him who reviles his father or mother suffer death,’ 
 \V{5}  But you say ‘Whenever any one says to his father or mother “Whatever of mine might have been of service to you is ‘Given to God,’”  
 \V{6}  He is in no way bound to honor his father.’ In this way you have nullified the words of God on account of your traditions. 
 \V{7}  Hypocrites! It was well said by Isaiah when he prophesied about you —  
 \V{8}  ‘This is a people that honor me with their lips, While their hearts are far removed from me; 
 \V{9}  But vainly do they worship me, For they teach but the precepts of men.’”  
 \V{10}  Then Jesus called the people to him, and said:   “Listen, and mark my words. 
 \V{11}  It is not what enters a man’s mouth that ‘defiles’ him, but what comes out from his mouth — that does defile him!”  
 \V{12}  On this his disciples came up to him, and said: “Do you know that the Pharisees were shocked on hearing what you said?” 
 \V{13}  “Every plant,”   Jesus replied,   “that my heavenly Father has not planted will be rooted up. 
 \V{14}  Let them be; they are but blind guides; and, if one blind man guides another, both of them will fall into a ditch.”  
 \V{15}  Upon this, Peter said to Jesus: “Explain this saying to us.” 
 \V{16}  “What, do even you understand nothing yet?”   Jesus exclaimed. 
 \V{17}  “Do not you see that whatever goes into the mouth passes into the stomach, and is afterwards expelled? 
 \V{18}  But the things that come out from the mouth proceed from the heart, and it is these that defile a man; 
 \V{19}  For out of the heart proceed evil thoughts — murder, adultery, unchastity, theft, perjury, slander. 
 \V{20}  These are the things that defile a man; but eating with unwashed hands does not defile a man.”  \indenting[yes]\par 
 \V{21}  On going away from that place, Jesus retired to the country round Tyre and Sidon. 
 \V{22}  There, a Canaanite woman of that district came out and began calling to Jesus: “Take pity on me, Master, Son of David; my daughter is grievously possessed by a demon.” 
 \V{23}  But Jesus did not answer her a word; and his disciples came up and begged him to send her away. “She keeps calling out after us,” they said. 
 \V{24}  “I was not sent,”   replied Jesus,   “to any one except the lost sheep of Israel.”  
 \V{25}  But the woman came, and, bowing to the ground before him, said: “Master, help me.” 
 \V{26}  “It is not fair,”   replied Jesus,   “to take the children’s food and throw it to dogs.”  
 \V{27}  “Yes, Master,” she said, “for even dogs do feed on the scraps that fall from their owners’ table.” 
 \V{28}  “Your faith is great,”   was his reply to the woman;   “it shall be as you wish!”   And her daughter was cured that very hour. \indenting[yes]\par 
 \V{29}  On leaving that place, Jesus went to the shore of the Sea of Galilee; and then went up the hill, and sat down., you will do what not only what has been done to the fig tree, but, even if you should say to this hill ‘Be lifted up and hurled into the sea!’ it would be done. 
 \V{30}  Great crowds of people came to him, bringing with them those who were lame, crippled, blind, or dumb, and many others. They put them down at his feet, and he cured them; 
 \V{31}  And the crowds were astonished, when they saw the dumb talking, the cripples made sound, the lame walking about, and the blind with their sight restored; and they praised the God of Israel. 
 \V{32}  Afterwards Jesus called his disciples to him, and said:   “My heart is moved at the sight of all these people, for they have already been with me three days and they have nothing to eat; and I am unwilling to send them away hungry, for fear that they should break down on the way.”  
 \V{33}  “Where can we,” his disciples asked, “in a lonely place find enough bread for such a crowd as this?” 
 \V{34}  “How many loaves have you?”   said Jesus. “Seven,” they answered, “and a few small fish.” 
 \V{35}  Telling the crowd to sit down on the ground, 
 \V{36}  Jesus took the seven loaves and the fish, and, after saying the thanksgiving, broke them, and gave them to the disciples; and the disciples gave them to the crowds. 
 \V{37}  Every one had sufficient to eat, and they picked up seven baskets full of the broken pieces left. 
 \V{38}  The men who ate were four thousand in number without counting women and children. 
 \V{39}  Then, after dismissing the crowds, Jesus got into the boat, and went to the neighborhood of Magadan. \indenting[yes]\par 
 \C{16}  Here the Pharisees and Sadducees came up, and, to test Jesus, requested him to show them some sign from the heavens. 
 \V{2}  But Jesus answered:   “In the evening you say ‘It will be fine weather, for the sky is as red as fire.’ 
 \V{3}  But in the morning you say ‘To-day it will be stormy, for the sky is as red as fire and threatening.’ You learn to read the sky; yet you are unable to read the signs of the times! 
 \V{4}  A wicked and unfaithful generation is asking for a sign, but no sign shall be given it except the sign of Jonah.”   So he left them and went away. \indenting[yes]\par 
 \V{5}  Now the disciples had crossed to the opposite shore, and had forgotten to take any bread. 
 \V{6}  Presently Jesus said to them: “Take care and be on your guard against the leaven of the Pharisees and Sadducees.” 
 \V{7}  But the disciples began talking among themselves about their having brought no bread. 
 \V{8}  On noticing this, Jesus said:   “Why are you talking among yourselves about your being short of bread, O men of little faith? 
 \V{9}  Do not you yet see, nor remember the five loaves for the five thousand, and how many baskets you took away? 
 \V{10}  Nor yet the seven loaves for the four thousand, and how many basketfuls you took away? 
 \V{11}  How is it that you do not see that I was not speaking about bread? Be on your guard against the leaven of the Pharisees and Sadducees.”  
 \V{12}  Then they understood that he had told them to be on their guard, not against the leaven of bread, but against the teaching of the Pharisees and Sadducees. \indenting[yes]\par 
 \V{13}  On coming into the neighborhood of Caesarea Philippi, Jesus asked his disciples this question —   “Who do people say that the Son of Man is?”  
 \V{14}  “Some say John the Baptist,” they answered, “Others, however, say that he is Elijah, while others again say Jeremiah, or one of the Prophets.” 
 \V{15}  “But you,”   he said,   “who do you say that I am?”  
 \V{16}  And to this Simon Peter answered: “You are the Christ, the Son of the Living God.” 
 \V{17}  “Blessed are you, Simon, Son of Jonah,”   Jesus replied.   “For no human being has revealed this to you, but my Father who is in Heaven. 
 \V{18}  Yes, and I say to you, Your name is ‘Peter — a Rock, and on this rock I will build my Church, and the Powers of the Place of Death shall not prevail over it. 
 \V{19}  I will give you the keys of the Kingdom of Heaven. Whatever you forbid on earth will be held in Heaven to be forbidden, and whatever you allow on earth will be held in Heaven to be allowed.”  
 \V{20}  Then he charged his disciples not to tell any one that he was the Christ. \indenting[yes]\par 
 \V{21}  At that time Jesus Christ began to explain to his disciples that he must go to Jerusalem, and undergo much suffering at the hands of the Councillors, and Chief Priests, and Teachers of the Law, and be put to death, and rise on the third day. 
 \V{22}  But Peter took Jesus aside, and began to rebuke him. “Master,” he said, “please God that shall never be your fate!” 
 \V{23}  Jesus, however, turning to Peter, said:   “Out of my way, Satan! You are a hindrance to me; for you look at things, not as God does, but as man does.”  
 \V{24}  Then Jesus said to his disciples:   “If any man wishes to walk in my steps, let him renounce self, and take up his cross, and follow me. 
 \V{25}  For whoever wishes to save his life will lose it, and whoever, for my sake, loses his life shall find it. 
 \V{26}  What good will it do a man to gain the whole world, if he forfeits his life? or what will a man give that is of equal value with his life? 
 \V{27}  For the Son of Man is to come in his Father’s Glory, with his angels, and then he ‘will give to every man what his actions deserve.’ 
 \V{28}  I tell you, some of those who are standing here will not know death till they have seen the Son of Man coming into his Kingdom.”  \indenting[yes]\par 
 \C{17}  Six days later, Jesus took with him Peter, and the brothers James and John, and led them up a high mountain alone. 
 \V{2}  There his appearance was transformed before their eyes; his face shown like the sun, and his clothes became as white as the light. 
 \V{3}  And all at once Moses and Elijah appeared to them, talking with Jesus. 
 \V{4}  “Master,” exclaimed Peter, interposing, “it is good to be here; if you wish, I will make three tents here, one for you, one for Moses, and one for Elijah.” 
 \V{5}  While he was still speaking, a bright cloud enveloped them, and there was a voice from the cloud that said —  “This is my Son, the Beloved, in whom I delight; him you must hear.” 
 \V{6}  The disciples, on hearing this, fell on their faces, greatly afraid. 
 \V{7}  But Jesus came and touched them, saying as he did so:   “Rise up, and do not be afraid.”  
 \V{8}  When they raised their eyes, they saw no one but Jesus himself alone. 
 \V{9}  As they were going down the mountain side, Jesus gave them this warning —   “Do not speak of this vision to any one, until the Son of Man has risen from the dead.”  
 \V{10}  “How is it,” his disciples asked, “that our Teachers of the Law say that Elijah has to come first?” 
 \V{11}  “Elijah indeed does come,”   Jesus replied,   “and will restore everything; 
 \V{12}  And I tell you that Elijah has already come, and people have not recognized him, but have treated him just as they pleased. In the same way, too, the Son of Man is destined to undergo suffering at men’s hands.”  
 \V{13}  Then the disciples understood that it was of John the Baptist that he had spoken to them. \indenting[yes]\par 
 \V{14}  When they came to the crowd, a man came up to Jesus, and, kneeling down before him, said: 
 \V{15}  “Master, take pity on my son, for he is epileptic and suffers terribly; indeed, he often falls into the fire and into the water; 
 \V{16}  I brought him to your disciples, but they could not cure him.” 
 \V{17}  “O faithless and perverse generation!”   Jesus exclaimed,   “how long must I be among you? how long must I have patience with you? Bring the boy here to me.”  
 \V{18}  Then Jesus rebuked the demon, and it came out of the boy; and he was cured from that very hour. 
 \V{19}  Afterwards the disciples came up to Jesus, and asked him privately: “Why was it that we could not drive it out?” 
 \V{20}  “Because you have so little faith,”   he answered;   “for, I tell you, if your faith were only like a mustard-seed, you could say to this mountain ‘Move from this place to that!’ and it would be moved; and nothing would be impossible to you.”  \indenting[yes]\par 
 \V{22}  While Jesus and his disciples were together in Galilee, he said to them:   “The Son of Man is destined to be betrayed into the hands of his fellow-men, 
 \V{23}  And they will put him to death, but on the third day he will rise.”   And the disciples were greatly distressed. \indenting[yes]\par 
 \V{24}  After they had reached Capernaum, the collectors of the Temple-rate came up to Peter, and said: “Does not your Master pay the Temple-rate?” 
 \V{25}  “Yes,” answered Peter. But, on going into the house, before he could speak, Jesus said:   “What do you think, Simon? From whom do earthly kings take taxes or tribute? From their sons, or from others?”  
 \V{26}  “From others,” answered Peter.   “Well then,”   continued Jesus,   “their sons go free. 
 \V{27}  Still, that we may not shock them, go and throw a line into the Sea; take the first fish that rises, open its mouth, and you will find in it a piece of money. Take that, and give it to the collectors for both of us.”  \indenting[yes]\par 
 \C{18}  On the same occasion the disciples came to Jesus, and asked him: “Who is really the greatest in the Kingdom of Heaven?” 
 \V{2}  Jesus called a little child to him, and placed it in the middle of them, and then said: 
 \V{3}  “I tell you, unless you change and become like little children, you will not enter the Kingdom of Heaven at all. 
 \V{4}  Therefore, any one who will humble himself like this child —  that man shall be the greatest in the Kingdom of Heaven. 
 \V{5}  And any one who, for the sake of my Name, welcomes even one little child like this, is welcoming me. 
 \V{6}  But, if any one puts a snare in the way of one of these lowly ones who believe in me, it would be best for him to be sunk in the depths of the sea with a great millstone hung round his neck. 
 \V{7}  Alas for the world because of such snares! There cannot but be snares; yet alas for the man who is answerable for the snare! \indenting[yes]\par 
 \V{8}  If your hand or your foot is a snare to you, cut it off, and throw it away. It would be better for you to enter the Life maimed or lame, than to have both hands, or both feet, and be thrown into the aeonian fire. 
 \V{9}  If your eye is a snare to you, take it out, and throw it away. It would be better for you to enter the Life with only one eye, than to have both eyes and be thrown into the fiery Pit. 
 \V{10}  Beware of despising one of these lowly ones, for in Heaven, I tell you, their angels always see the face of my Father who is in Heaven. \indenting[yes]\par 
 \V{12}  What think you? If a man owns a hundred sheep, and one of them strays, will he not leave the ninety-nine on the hills, and go and search for the one that is straying? 
 \V{13}  And, if he succeeds in finding it, I tell you that he rejoices more over that one sheep than over the ninety-nine which did not stray. 
 \V{14}  So, too, it is the will of my Father who is in Heaven that not one of these lowly ones should be lost. \indenting[yes]\par 
 \V{15}  If your Brother does wrong, go to him and convince him of his fault when you and he are alone. If he listens to you, you have won your Brother. 
 \V{16}  But, if he does not listen to you, take with you one or two others, so that ‘on the evidence of two or three witnesses, every word may be put beyond dispute.’ 
 \V{17}  If he refuses to listen to them, speak to the Church; and, if he also refuses to listen to the Church, treat him as you would a Gentile or a tax-gatherer. \indenting[yes]\par 
 \V{18}  I tell you, all that you forbid on earth will be held in Heaven to be forbidden, and all that you allow on earth will be held in Heaven to be allowed. 
 \V{19}  Again, I tell you that, if but two of you on earth agree as to what they shall pray for, whatever it be, it will be granted them by my Father who is in Heaven. 
 \V{20}  For where two or three have come together in my Name, I am present with them.”  \indenting[yes]\par 
 \V{21}  Then Peter came up, and said to Jesus: “Master, how often am I to forgive my Brother when he wrongs me? As many as seven times?” 
 \V{22}  But Jesus answered:   “Not seven times, but ‘seventy times seven.’ 
 \V{23}  And therefore the Kingdom of Heaven may be compared to a king who wished to settle accounts with his servants. 
 \V{24}  When he had begun to do so, one of them was brought to him who owed him six million pounds; 
 \V{25}  And, as he could not pay, his master ordered him to be sold towards the payment of the debt, together with his wife, and his children, and everything that he had. 
 \V{26}  Thereupon the servant threw himself down on the ground before him and said ‘Have patience with me, and I will pay you all.’ 
 \V{27}  The master was moved with compassion; and he let him go, and forgave him the debt. 
 \V{28}  But, on going out, that same servant came upon one of his fellow-servants who owed him ten pounds. Seizing him by the throat, he said ‘Pay what you owe me.’ 
 \V{29}  Thereupon his fellow-servant threw himself on the ground and begged for mercy. ‘Have patience with me,’ he said, ‘and I will pay you.’ 
 \V{30}  But the other would not, but went and put him in prison till he should pay his debt. 
 \V{31}  When his fellow-servants saw what had happened, they were greatly distressed, and went to their master and laid the whole matter before him. 
 \V{32}  Upon that the master sent for the servant, and said to him ‘You wicked servant! When you begged me for mercy, I forgave you the whole of that debt. 
 \V{33}  Ought not you, also, to have shown mercy to your fellow- servant, just as I showed mercy to you?’ 
 \V{34}  Then his master, in anger, handed him over to the jailers, until he should pay the whole of his debt. 
 \V{35}  So, also, will my heavenly Father do to you, unless each one of you forgives his Brother from his heart.”  \indenting[yes]\par 
 \C{19}  At the conclusion of this teaching, Jesus withdrew from Galilee, and went into that district of Judea which is on the other side of the Jordan. 
 \V{2}  Great crowds followed him, and he cured them there. 
 \V{3}  Presently some Pharisees came up to him, and, to test him, said: “Has a man the right to divorce his wife for every cause?” 
 \V{4}  “Have not you read,”   replied Jesus,   “that at the beginning the Creator ‘made them male and female,’ 
 \V{5}  And said — ‘For this reason a man shall leave his father and mother, and be united to his wife, and the man and his wife shall become one’? 
 \V{6}  So that they are no longer two, but one. What God himself, then, has yoked together man must not separate.”  
 \V{7}  “Why, then,” they said, “did Moses direct that a man should ‘serve his wife with a notice of separation and divorce her’?” 
 \V{8}  “Moses, owing to the hardness of your hearts,”   answered Jesus,   “permitted you to divorce your wives, but that was not so at the beginning. 
 \V{9}  But I tell you that any one who divorces his wife, except on the ground of her unchastity, and marries another woman, is guilty of adultery.”  
 \V{10}  “If that,” said the disciples, “is the position of a man with regard to his wife, it is better not to marry.” 
 \V{11}  “It is not every one,”   replied Jesus,   “who can accept this teaching, but only those who have been enabled to do so. 
 \V{12}  Some men, it is true, have from birth been disabled for marriage, while others have been disabled by their fellow men, and others again have disabled themselves for the sake of the Kingdom of Heaven. Let him accept it who can.”  \indenting[yes]\par 
 \V{13}  Then some little children were brought to Jesus, for him to place his hands on them, and pray; but the disciples found fault with those who had brought them. 
 \V{14}  Jesus, however, said:   “Let the little children come to me, and do not hinder them, for it is to the childlike that the Kingdom of Heaven belongs.”  
 \V{15}  So he placed his hands on them, and then went on his way. \indenting[yes]\par 
 \V{16}  And a man came up to Jesus, and said: “Teacher, what good thing must I do to obtain Immortal life?” 
 \V{17}  “Why ask me about goodness?”   answered Jesus.   “There is but One who is good. If you want to enter the Life, keep the commandments.”  
 \V{18}  “What commandments?” asked the man.   “These,”   answered Jesus: —   “‘Thou shalt not kill. Thou shalt not commit adultery. Thou shalt not steal. Thou shalt not say what is false about others. 
 \V{19}  Honor thy father and thy mother.’ And ‘Thou shalt love thy neighbor as thou dost thyself.”  
 \V{20}  “I have observed all these,” said the young man. “What is still wanting in me?” 
 \V{21}  “If you wish to be perfect,”   answered Jesus,   “go and sell your property, and give to the poor, and you shall have wealth in Heaven; then come and follow me.”  
 \V{22}  On hearing these words, the young man went away distressed, for he had great possessions. 
 \V{23}  At this, Jesus said to his disciples:   “I tell you that a rich man will find it hard to enter the Kingdom of Heaven! 
 \V{24}  I say again, it is easier for a camel to get through a needle’s eye than for a rich man to enter the Kingdom of Heaven!”  
 \V{25}  On hearing this, the disciples exclaimed in great astonishment: “Who then can possibly be saved?” 
 \V{26}  But Jesus looked at them, and said:   “With men this is impossible, but with God everything is possible.”   Then Peter turned and said to Jesus: 
 \V{27}  “But we — we left everything, and followed you; what, then, shall we have?” 
 \V{28}  “I tell you,”   answered Jesus,   “that at the New Creation, ‘when the Son of Man takes his seat on his throne of glory,’ you who followed me shall be seated upon twelve thrones, as judges of the twelve tribes of Israel. 
 \V{29}  Every one who has left houses, or brothers, or sisters, or father, or mother, or children, or land, on account of my Name, will receive many times as much, and will ‘gain Immortal Life.’ 
 \V{30}  But many who are first now will then be last, and those who are last will be first. 
 \C{20}  For the Kingdom of Heaven is like an employer who went out in the early morning to hire laborers for his vineyards. 
 \V{2}  He agreed with the laborers to pay them two shillings a day, and sent them into his vineyard. 
 \V{3}  On going out again, about nine o’clock, he saw some others standing in the market-place, doing nothing. 
 \V{4}  ‘You also may go into my vineyard,’ he said, ‘and I will pay you what is fair.’ 
 \V{5}  So the men went. Going out again about mid-day and about three o’clock, he did as before. 
 \V{6}  When he went out about five, he found some other men standing there, and said to them ‘Why have you been standing here all day long, doing nothing?’ 
 \V{7}  ‘Because no one has hired us,’ they answered. ‘You also may go into my vineyard,’ he said. 
 \V{8}  In the evening the owner of the vineyard said to his steward ‘Call the laborers, and pay them their wages, beginning with the last, and ending with the first. 
 \V{9}  Now when those who had been hired about five o’clock went up, they received two shillings each. 
 \V{10}  So, when the first went up, they thought that they would receive more, but they also received two shillings each; 
 \V{11}  On which they began to grumble at their employer. 
 \V{12}  ‘These last,’ they said, ‘have done only one hour’s work, and yet you have put them on the same footing with us, who have borne the brunt of the day’s work, and the heat.’ 
 \V{13}  ‘My friend,’ was his reply to one of them, ‘I am not treating you unfairly. Did not you agree with me for two shillings? 
 \V{14}  Take what belongs to you, and go. I choose to give to this last man the same as to you. 
 \V{15}  Have not I the right to do as I choose with what is mine? Are you envious because I am liberal?’ 
 \V{16}  So those who are last will be first, and the first last.”  \indenting[yes]\par 
 \V{17}  When Jesus was on the point of going up to Jerusalem, he gathered the twelve disciples round him by themselves, and said to them as they were on their way: 
 \V{18}  “Listen! We are going up to Jerusalem; and there the Son of Man will be betrayed to the Chief Priests and Teachers of the Law, and they will condemn him to death, 
 \V{19}  And give him up to the Gentiles for them to mock, and to scourge, and to crucify; and on the third day he will rise.”  \indenting[yes]\par 
 \V{20}  Then the mother of Zebediah’s sons came to him with her sons, bowing to the ground, and begging a favor. 
 \V{21}  “What is it that you want?”   he asked. “I want you to say,” she replied, “that in your Kingdom these two sons of mine may sit, one on your right, and the other on your left.” 
 \V{22}  “You do not know what you are asking,”   was Jesus’ answer.   “Can you drink the cup that I am to drink?”   “Yes,” they exclaimed, “we can.” 
 \V{23}  “You shall indeed drink my cup,”   he said,   “but as to a seat at my right and at my left — that is not mine to give, but it is for those for whom it has been prepared by my Father.”  
 \V{24}  On hearing of this, the ten others were very indignant about the two brothers. 
 \V{25}  Jesus, however, called the ten to him, and said:   “The rulers of the Gentiles lord it over them as you know, and their great men oppress them. 
 \V{26}  Among you it is not so. 
 \V{27}  No, whoever wants to become great among you must be your servant, and whoever wants to take the first place among you, must be your slave; 
 \V{28}  Just as the Son of Man came, not to be served, but to serve, and to give his life as a ransom for many.”  \indenting[yes]\par 
 \V{29}  As they were going out of Jericho, a great crowd followed him. 
 \V{30}  Two blind men who were sitting by the road-side, hearing that Jesus was passing, called out: “Take pity on us, Master, Son of David!” 
 \V{31}  The crowd told them to be quiet; but the men only called out the louder: “Take pity on us, Master, Son of David!” 
 \V{32}  Then Jesus stopped and called them.   “What do you want me to do for you?”   he said. 
 \V{33}  “Master,” they replied, “we want our eyes to be opened.” 
 \V{34}  So Jesus, moved with compassion, touched their eyes, and immediately they recovered their sight, and followed him. \indenting[yes]\par 
 \C{21}  When they had almost reached Jerusalem, having come as far as Bethphage, on the Mount of Olives, Jesus sent on two disciples. 
 \V{2}  “Go to the village facing you,”   he said,   “and you will immediately find an ass tethered, with a foal by her side; untie her, and lead her here for me. 
 \V{3}  And, if any one says anything to you, you are to say this —  ‘The Master wants them’; and he will send them at once.”  
 \V{4}  This happened in fulfillment of these words in the Prophet —  
 \V{5}  ‘Say to the daughter of Zion — “Behold, thy King is coming to thee, Gentle, and riding on an ass, And on the foal of a beast of burden.”’ \indenting[yes]\par 
 \V{6}  So the disciples went and did as Jesus had directed them. 
 \V{7}  They led the ass and the foal back, and, when they had put their cloaks on them, he seated himself upon them. 
 \V{8}  The immense crowd of people spread their cloaks in the road, while some cut branches off the trees, and spread them on the road. 
 \V{9}  The crowds that led the way, as well as those that followed behind, kept shouting: “God save the Son of David! Blessed is he who comes in the name of the Lord! God save him from on high!” 
 \V{10}  When he had entered Jerusalem, the whole city was stirred, and asked —  
 \V{11}  “Who is this?”, to which the crowd replied — “This is the Prophet Jesus from Nazareth in Galilee.” \indenting[yes]\par 
 \V{12}  Jesus went into the Temple Courts, and drove out all those who were buying and selling there. He overturned the tables of the money-changers, and the seats of the pigeon-dealers, 
 \V{13}  And said to them:   “Scripture says ‘My House shall be called a house of prayer’; but you are making it ‘a den of robbers.’”  
 \V{14}  While he was still in the Temple Courts, some blind and some lame people came up to him, and he cured them. 
 \V{15}  But, when the Chief Priests and the Teachers of the Law saw the wonderful things that Jesus did, and the boys who were calling out in the Temple Courts “God save the Son of David!”, they were indignant, 
 \V{16}  And said to him: “Do you hear what these boys are saying?”   “Yes,”   answered Jesus;   “but did you never read the words — ‘Out of the mouths of babes and sucklings thou hast called forth perfect praise’?”  \indenting[yes]\par 
 \V{17}  Then he left them, and went out of the city to Bethany, and spent the night there. \indenting[yes]\par 
 \V{18}  The next morning, in returning to the city, Jesus became hungry; 
 \V{19}  And, noticing a solitary fig tree by the road-side, he went up to it, but found nothing on it but leaves. So he said to it:   “Never again shall fruit be gathered off you.”   And suddenly the fruit tree withered up. 
 \V{20}  When the disciples saw this, they exclaimed in astonishment: “How suddenly the fig tree withered up!” 
 \V{21}  “I tell you,”   replied Jesus,   “if you have faith, without ever a doubt, you will do what not only what has been done to the fig tree, but, even if you should say to this hill ‘Be lifted up and hurled into the sea!’ it would be done. 
 \V{22}  And whatever you ask for in your prayers will, if you have faith, be granted you.”  \indenting[yes]\par 
 \V{23}  After Jesus had come into the Temple Courts, the Chief Priests and the Councillors of the Nation came up to him as he was teaching, and said: “ What authority have you to do these things? Who gave you this authority?” 
 \V{24}  “I, too,”   said Jesus in reply,   “will ask you one question; if you will give me an answer to it, then I, also, will tell you what authority I have to act as I do. 
 \V{25}  It is about John’s baptism. What was its origin? divine or human?” But they began arguing among themselves: “If we say ‘divine,’ he will say to us ‘Why then did not you believe him?’ 
 \V{26}  But if we say ‘human,’ we are afraid of the people, for every one regards John as a Prophet.” 
 \V{27}  So the answer they gave Jesus was — “We do not know.”   “Then I,”   he said,   “refuse to tell you what authority I have to do these things. 
 \V{28}  What do you think of this? There was a man who had two sons. He went to the elder and said ‘Go and work in the vineyard to-day my son.’ 
 \V{29}  ‘Yes, sir,’ he answered; but he did not go. 
 \V{30}  Then the father went to the second son, and said the same. ‘I will not,’ he answered; but afterwards he was sorry and went. 
 \V{31}  Which of the two sons did as his father wished?”   “ The second,” they said.   “I tell you,”   added Jesus,   “that tax-gatherers and prostitutes are going into the Kingdom of God before you. 
 \V{32}  For when John came to you, walking in the path of righteousness, you did not believe him, but tax-gatherers and prostitutes did; and yet you, though you saw this, even then were not sorry, nor did you believe him. \indenting[yes]\par 
 \V{33}  Listen to another parable. A man, who was an employer, once planted a vineyard, put a fence round it, dug a winepress in it, built a tower, and then let it out to tenants and went abroad. 
 \V{34}  When the time for the vintage drew near, he sent his servants to the tenants, to receive his share of the produce. 
 \V{35}  But the tenants seized his servants, beat one, killed another, and stoned a third. 
 \V{36}  A second time the owner sent some servants, a larger number than before, and the tenants treated them in the same way. 
 \V{37}  As a last resource he sent his son to them. ‘They will respect my son,’ he said. 
 \V{38}  But the tenants, on seeing his son, said to each other ‘Here is the heir! Come, let us kill him, and get his inheritance.’ 
 \V{39}  So they seized him, and threw him outside the vineyard, and killed him. 
 \V{40}  Now, when the owner of the vineyard comes, what will he do to those tenants?”  
 \V{41}  “Miserable wretches!” they exclaimed, “he will put them to a miserable death, and he will let out the vineyard to other tenants, who will pay him his share of the produce at the proper times.” 
 \V{42}  Then Jesus added:   “Have you never read in the Scriptures? —  ‘The very stone which the builders despised — Has now itself become the corner-stone; This corner-stone has come from the Lord, And is marvelous in our eyes.’ 
 \V{43}  And that, I tell you, is why the Kingdom of God will be taken from you, and given to a nation that does produce the fruit of the Kingdom. 
 \V{44}  Yes, and he who falls on this stone will be dashed to pieces, while any one on whom it falls — it will scatter him as dust.”  \indenting[yes]\par 
 \V{45}  After listening to these parables, the Chief Priests and the Pharisees saw that it was about them that he was speaking; 
 \V{46}  Yet, although eager to arrest him, they were afraid of the crowds, who regarded him as a Prophet. \indenting[yes]\par 
 \C{22}  Once more Jesus spoke to them in parables. 
 \V{2}  “The Kingdom of Heaven,” he said, “may be compared to a king who gave a banquet in honor of his son’s wedding. 
 \V{3}  He sent his servants to call those who had been invited to the banquet, but they were unwilling to come. 
 \V{4}  A second time he sent some servants, with orders to say to those who had been invited ‘I have prepared my breakfast, my cattle and fat beasts are killed and everything is ready; come to the banquet.’ 
 \V{5}  They, however, took no notice, but went off, one to his farm, another to his business; 
 \V{6}  While the rest, seizing his servants, ill-treated them and killed them. 
 \V{7}  The king, in anger, sent his troops, put those murderers to death, and set their city on fire. 
 \V{8}  Then he said to his servants ‘The banquet is prepared, but those who were invited were not worthy. 
 \V{9}  So go to the cross-roads, and invite everyone you find to the banquet.’ 
 \V{10}  The servants went out into the roads and collected all the people whom they found, whether bad or good; and the bridal-hall was filled with guests. 
 \V{11}  But, when the king went in to see his guests, he noticed there a man who had not put on a wedding-robe. 
 \V{12}  So he said to him ‘My friend, how is it that you came in here without a wedding-robe?’ The man was speechless. 
 \V{13}  Then the king said to the attendants ‘Tie him hand and foot, and ‘put him out into the darkness’ outside, where there will be weeping and grinding of teeth.’ 
 \V{14}  For many are called, but few chosen.”  \indenting[yes]\par 
 \V{15}  Then the Pharisees went away and conferred together as to how they might lay a snare for Jesus in the course of conversation. 
 \V{16}  They sent their disciples, with the Herodians, to say to him: “Teacher, we know that you are an honest man, and that you teach the way of God honestly, and are not afraid of any one; for you pay no regard to a man’s position. 
 \V{17}  Tell us, then, what you think. Are we right in paying taxes to the Emperor, or not?” 
 \V{18}  Perceiving their malice, Jesus answered:   “Why are you testing me, you hypocrites? 
 \V{19}  Show me the coin with which the tax is paid.”   And, when they had brought him a florin, 
 \V{20}  He asked:   “Whose head and title are these?”  
 \V{21}  “The Emperor’s,” they answered: on which he said to them:   “Then pay to the Emperor what belongs to the Emperor, and to God what belongs to God.”  
 \V{22}  They wondered at his answer, and left him alone and went away. \indenting[yes]\par 
 \V{23}  That same day some Sadducees came up to Jesus, maintaining that there is no resurrection. Their question was this: —  
 \V{24}  “Teacher, Moses said — ’should a man die without children, the man’s brother shall become the husband of the widow, and raise a family for his brother.’ 
 \V{25}  Now we had living among us seven brothers; of whom the eldest married and died, and, as he had no family, left his wife for his brother. 
 \V{26}  The same thing happened to the second and the third brothers, and indeed to all the seven. 
 \V{27}  The woman herself died last of all. 
 \V{28}  At the resurrection, then, whose wife will she be out of the seven, all of them having had her?” 
 \V{29}  “Your mistake,”   replied Jesus,   “is due to your ignorance of the Scriptures, and of the power of God. 
 \V{30}  For at the resurrection there is no marrying or being married, but all who rise are as angels in Heaven. 
 \V{31}  As to the resurrection of the dead, have you not read these words of God —  
 \V{32}  ‘I am the God of Abraham, and the God of Isaac, and the God of Jacob’? He is not the God of dead men, but of living.”  
 \V{33}  The crowds, who had been listening to him, were greatly struck with his teaching. \indenting[yes]\par 
 \V{34}  When the Pharisees heard that Jesus had silenced the Sadducees, they collected together. 
 \V{35}  Then one of them, a Student of the Law, to test him, asked this question —  
 \V{36}  “Teacher, what is the great commandment in the Law?” 
 \V{37}  His answer was:   “‘Thou shalt love the Lord thy God with all thy Heart, and with all thy soul, and with all thy mind.’ 
 \V{38}  This is the great first commandment. 
 \V{39}  The second, which is like it, is this — ‘Thou shalt love thy neighbor as thou dost thyself.’ 
 \V{40}  On these two commandments hang all the Law and the Prophets.”  
 \V{41}  Before the Pharisees separated, Jesus put this question to them —  
 \V{42}  “What do you think about the Christ? Whose son is he?”   “David’s,” they said. 
 \V{43}  “How is it, then,”   Jesus replied,   “that David, speaking under inspiration, calls him ‘lord,’ in the passage- 
 \V{44}  ‘The Lord said to my Lord: “Sit at my right hand, Until I put thy enemies beneath thy feet”’? 
 \V{45}  Since, then, David calls him ‘lord,’ how is he David’s son?”  
 \V{46}  No one could say a word in answer; nor did any one after that day venture to question him further. \indenting[yes]\par 
 \C{23}  Then Jesus speaking to the crowds and to his disciples, said: 
 \V{2}  “The teachers of the Law and the Pharisees now occupy the chair of Moses. 
 \V{3}  Therefore practice and lay to heart everything that they tell preach but do not practice. 
 \V{4}  While they make up heavy loads and pile them on other men’s shoulder’s they decline, themselves, to lift a finger to move them. 
 \V{5}  All their actions are done to attract attention. They widen their phylacteries, and increase the size of their tassels, 
 \V{6}  and like to have the place of honor at dinner, and the best seats in the Synagogues, 
 \V{7}  and to be greeted in the markets with respect, and to be called ‘Rabbi’ for everybody. 
 \V{8}  But do not allow yourselves to be called ‘Rabbi,’ for you have only one Father, the heavenly Father. 
 \V{9}  And do not call any one Father, the heavenly Father. 
 \V{10}  Nor must you allow yourselves to be called ‘Leaders,’ for you have only one Leader, the Christ. 
 \V{11}  The man who would be the greatest among you must be your servant. 
 \V{12}  Whoever shall exalt himself will be humbled, and whoever shall humble himself will be exalted. 
 \V{13}  But alas for you, Teachers of the Law and Pharisees, hypocrites that you are! You turn the key of the Kingdom of Heaven in men’s faces. For you do not go in yourselves, nor yet allow those who try to go in to do so. 
 \V{14}  Alas for you, Teachers of the Law and Pharisees, hypocrites that you are! You destroy widow’s houses, even while pretending to make long prayers; therefore you shall receive greater condemnation. 
 \V{15}  Alas for you, teachers of the law and Pharisees, hypocrites that you are! You scour land and sea to make a single convert, and, when he is gained, you make him twice as deserving of the Pit as you are yourselves. 
 \V{16}  Alas for you, you blind guides! You say ‘if any answer by the Temple, his oath counts for nothing; but, if any one swears by the gold of the Temple, his oath is binding him’! 
 \V{17}  Fools that you are and blind! Which is the more important? The gold? Or the Temple which has given sacredness to the gold? 
 \V{18}  You say, too, ‘If any one swears by the altar, his oath counts for nothing, but, if any one swears by the offering placed on it, his oath is binding on him’! 
 \V{19}  Blind indeed! Which is the more important? The offering? or the altar which gives sacredness to the offering? 
 \V{20}  Therefore a man, swearing by the altar, swears by it and by all that is on it, 
 \V{21}  and a man, swearing by the Temple, swears by it and by him who dwells in it, 
 \V{22}  while a man, swearing by Heaven, swears by the throne of God, and by him who sits upon it. 
 \V{23}  Alas for you, Teachers of the Law and Pharisees, hypocrites that you are! You pay tithes on mint, fennel, and caraway seed, and have neglected the weightier matters of the Law — justice, mercy, and good faith. These last you ought to have put into practice, without neglecting the first. 
 \V{24}  You blind guides, to strain out a gnat and to swallow a camel! 
 \V{25}  Alas for you, Teachers of the Law and Pharisees, hypocrites that you are! You clean the outside of the cup and the dish, but inside they are filled with the results of greed and self-indulgence. 
 \V{26}  You blind Pharisee! First clean the inside of the cup and the dish, so that the outside may become clean as well. 
 \V{27}  Alas for you, Teachers of the Law and Pharisees, hypocrites that you are! You are like whitewashed tombs, which indeed look fair outside, while inside they are filled with dead men’s bones and all kinds of filth. 
 \V{28}  It is the same with you. Outwardly, and to others, you have the look of religious men, but inwardly you are full of hypocrisy and sin. 
 \V{29}  Alas for you, Teachers of the Law and Pharisees, hypocrites that you are! You build the tombs of the Prophets, and decorate the monuments of religious men, 
 \V{30}  and say ‘Had we been living in the days of our ancestors, we should have taken part in their murder of the Prophets! 
 \V{31}  By doing this you are furnishing evidence against yourselves that you are true children of the men who murdered the Prophets. 
 \V{32}  Fill up the measure of your ancestor’s guilt. 
 \V{33}  You serpents and brood of vipers! How can you escape being sentenced to the Pit? 
 \V{34}  That is why I send you Prophets, wise men, and Teachers of the Law, some of whom you will crucify and kill, and some of whom you will scourge in your Synagogues, and persecute from town to town; 
 \V{35}  in order that upon your heads may fall every drop of innocent ‘blood split on earth,’ from the blood of innocent Abel down to that of Zechariah, Barachiah’s son, whom you murdered between the Temple and the altar. 
 \V{36}  All this, I tell you, will come home to the present generation. 
 \V{37}  Jerusalem! Jerusalem! She who slays the Prophets and stones the messengers sent to her — Oh, how often have I wished to gather your children round me, as a hen gathers her brood under her wings, and you would not came! 
 \V{38}  Verily, your house is left to you desolate! 
 \V{39}  For nevermore, I tell you, shall you see me, until you say —  ‘Blessed is He who comes in the Name of the Lord!’”  \indenting[yes]\par 
 \C{24}  Leaving the Temple Courts, Jesus was walking away, when his disciples came up to draw his attention to the Temple buildings. 
 \V{2}  “Do you see all these things?”   was his answer.   “I tell you, not a single stone will be left here upon another, which will not be throne down,”  
 \V{3}  So, while Jesus was sitting on the Mount of Olives, his disciples came up to him privately and said: “Tell us when this will be, and what will be the sign of your Coming, and of the close of the age.” 
 \V{4}  Jesus replied to them as follows:   “See that no one leads you astray; 
 \V{5}  for, many will take my name, and come saying ‘I am the Christ,’ and will lead many astray. 
 \V{6}  And you will hear of wars and rumors of wars; take care not to be alarmed, for such things must occur; but the end is not yet here. 
 \V{7}  For ‘nation will rise against nation and kingdom against kingdom,’ and there will be famines and earthquakes in various places. 
 \V{8}  All this, however, will be but the beginning of the birth pangs! 
 \V{9}  When that time comes, they will give you up to persecution, and will put you to death, and you will be hated by all nations on account of my Name. 
 \V{10}  And then many will fall away, and will betray one another, and hate one another. 
 \V{11}  Many false Prophets, also, will appear and lead many astray; 
 \V{12}  and, owing to the increase of wickedness, the love of most will grow cold. 
 \V{13}  Yet the man that endures to the end shall be saved. 
 \V{14}  And this Good News of the Kingdom shall be proclaimed throughout the world as a witness to all nations; and then will come the end. 
 \V{15}  As soon, then, as you see ‘the Foul Desecration,’ mentioned by the Prophet Daniel, standing in the Holy Place,”   (the reader must consider what this means) 
 \V{16}  “then those of you who are in Judea must take refuge in the mountains; 
 \V{17}  and a man on the housetop must not go down to get the things that are in his house; 
 \V{18}  nor must one who is on his farm turn back to get his cloak. 
 \V{19}  And alas for the women that are with child, and for those that are nursing infants in those days! 
 \V{20}  Pray, too, that your flight may not take place in winter, nor on a Sabbath; 
 \V{21}  for that will be ‘a time of great distress, the like of which has not occurred from the beginning of the world down to the present time’ — no, nor ever will again. 
 \V{22}  And had not those days been limited, not a single soul would escape; but for the sake of ‘God’s People’ a limit will be put to them. 
 \V{23}  And, at that time, if any one should say to you ‘Look! here is the Christ!’ or ‘Here he is!’, do not believe it; 
 \V{24}  For false Christs and false Prophets will arise, and will display great signs and marvels, so that, were it possible, even God’s People would be led astray. 
 \V{25}  Remember, I have told you beforehand. 
 \V{26}  Therefore, if people say to you ‘He is in the Wilderness!’, do not go out there; or ‘He is in an inner room!’, do not believe it; 
 \V{27}  For, just as lightning will start from the east and flash across to the west, so will it be with the Coming of the Son of Man. 
 \V{28}  Wherever a dead body lies, there will the vultures flock.’ 
 \V{29}  Immediately after the distress of those days, ‘the sun will be darkened, the moon will not give her light, the stars will fall from the heavens,’ and ‘the forces of the heavens will be convulsed.’ 
 \V{30}  Then will appear the sign of the Son of Man in the heavens; and all the peoples of the earth will mourn, when they see the Son of Man coming on the clouds of the heavens,’ with power and great glory; 
 \V{31}  And he will send his angels, with a great trumpet, and they will gather his People round him from the four winds, from one end of heaven to the other. \indenting[yes]\par 
 \V{32}  Learn the lesson taught by the fig tree. As soon as its branches are full of sap, and it is bursting into leaf, you know that summer is near. 
 \V{33}  And so may you, as soon as you see all these things, know that he is at your doors. 
 \V{34}  I tell you, even the present generation will not pass away, till all these things have taken place. 
 \V{35}  The heavens and the earth will pass away, but my words shall never pass away. 
 \V{36}  But about that Day and Hour, no one knows — not even the angels of Heaven, nor yet the Son — but only the Father himself. 
 \V{37}  For, just as in the days of Noah, so will it be at the Coming of the Son of Man. 
 \V{38}  In those days before the flood they went on eating and drinking, marrying and being married, up to the very day on which Noah entered the ark, 
 \V{39}  Taking not notice till the flood came and swept them one and all away; and so will it be at the Coming of the Son of Man. 
 \V{40}  At that time, of two men on a farm one will be taken and one left; 
 \V{41}  Of two women grinding with the hand-mill one will be taken and one left. 
 \V{42}  Therefore watch; for you cannot be sure on what day your Master is coming. 
 \V{43}  But this you do know, that, had the owner of the house known at what time of night the thief was coming, he would have been on the watch, and would not have allowed his house to be broken into. 
 \V{44}  Therefore, do you also prepare, since it is just when you are least expecting him that the Son of Man will come. 
 \V{45}  Who, then is that trustworthy, careful servant, who has been placed by his master over his household, to give them their food at the proper time? 
 \V{46}  Happy will that servant be whom his master, when he comes home, shall find doing this. 
 \V{47}  I tell you that his master will put him in charge of the whole of his property. 
 \V{48}  But, should he be a bad servant, and say to himself ‘My master is a long time in coming,’ 
 \V{49}  And begin to beat his fellow-servants, and eat and drink with drunkards, 
 \V{50}  That servant’s master will come on a day when he does not expect him, and at an hour of which he is unaware, 
 \V{51}  And will flog him severely, and assign him his place among the hypocrites, where there will be weeping and grinding of teeth. \indenting[yes]\par 
 \C{25}  Then the Kingdom of Heaven will be like ten bridesmaids who took their lamps and went out to meet the bridegroom. 
 \V{2}  Five of them were foolish, and five were prudent. 
 \V{3}  The foolish ones took their lamps, but took no oil with them; 
 \V{4}  While the prudent ones, besides taking their lamps, took oil in their jars. 
 \V{5}  As the bridegroom was late in coming, they all became drowsy, and slept. 
 \V{6}  But at midnight a shout was raised — ‘The Bridegroom is coming! Come out to meet him!’ 
 \V{7}  Then all the bridesmaids awoke and trimmed their lamps. 
 \V{8}  And the foolish said to the prudent ‘Give us some of your oil; our lamps are going out.’ 
 \V{9}  But the prudent ones answered ‘No, for fear that there will not be enough for you and for us. Go instead to those who sell it, and buy for yourselves.’ 
 \V{10}  But while they were on their way to buy it, the bridegroom came; and the bridesmaids who were ready went in with him to the banquet, and the door was shut. 
 \V{11}  Afterwards the other bridesmaids came. ‘Sir, Sir,’ they said, ‘open the door to us!’ 
 \V{12}  But the bridegroom answered ‘I tell you, I do not know you.’ 
 \V{13}  Therefore watch, since you know neither the Day nor the Hour. \indenting[yes]\par 
 \V{14}  For it is as though a man, going on his travels, called his servants, and gave his property into their charge. 
 \V{15}  He gave three thousand pounds to one, twelve hundred to another, and six hundred to a third, in proportion to the ability of each. Then he set out on his travels. 
 \V{16}  The man who had received the three thousand pounds went at once and traded with it, and made another three thousand. 
 \V{17}  So, too, the man who had received the twelve hundred pounds made another twelve hundred. 
 \V{18}  But the man who had received the six hundred went and dug a hole in the ground, and hid his master’s money. 
 \V{19}  After a long time the master of those servants returned, and settled accounts with them. 
 \V{20}  The man who had received the three thousand pounds came up and brought three thousand more. ‘Sir,’ he said, ‘you entrusted me with three thousand pounds; look, I have made another three thousand!’ 
 \V{21}  ‘Well done, good, trustworthy servant!’ said his master. ‘You have been trustworthy with a small sum; now I will place a large one in your hands; come and share your master’s joy!’ 
 \V{22}  Then the one who had received the twelve hundred pounds came up and said ‘Sir, you entrusted me with twelve hundred pounds; look, I have made another twelve hundred!’ 
 \V{23}  ‘Well done, good, trustworthy servant!’ said his master. ‘You have been trustworthy with a small sum; now I will place a large one in your hands; come and share your master’s joy!’ 
 \V{24}  The man who had received the six hundred pounds came up, too, and said ‘Sir, I knew that you were a hard man; you reap where you have not sown, and gather up where you have not winnowed; 
 \V{25}  And, in my fear, I went and hid your money in the ground; look, here is what belongs to you!’ 
 \V{26}  ‘You lazy, worthless servant!’ was his master’s reply. ‘You knew that I reap where I have not sown, and gather up where I have not winnowed? 
 \V{27}  Then you ought to have placed my money in the hands of bankers, and I, on my return, should have received my money, with interest. 
 \V{28}  ‘Therefore,’ he continued, ‘take away from him the six hundred pounds, and give it to the one who has the six thousand. 
 \V{29}  For, to him who has, more will be given, and he shall have abundance; but, as for him who has nothing, even what he has will be taken away from him. 
 \V{30}  As for the useless servant, ‘put him out into the darkness’ outside, where there will be weeping and grinding of teeth.’ \indenting[yes]\par 
 \V{31}  When the Son of Man has come in his glory and all the angels with him, then he ‘will take his seat on his throne of glory’; 
 \V{32}  And all the nations will be gathered before him, and he will separate the people — just as a shepherd separates sheep from goats- - 
 \V{33}  Placing the sheep on his right hand, and the goats on his left. 
 \V{34}  Then the King will say to those on his right ‘Come, you who are blessed by my Father, enter upon possession of the Kingdom prepared for you ever since the beginning of the world. 
 \V{35}  For, when I was hungry, you gave me food; when I was thirsty, you gave me drink; when I was a stranger, you took me to your homes; 
 \V{36}  When I was naked, you clothed me; when I fell ill, you visited me; and when I was in prison, you came to me.’ 
 \V{37}  Then the Righteous will answer ‘Lord, when did we see you hungry, and feed you? Or thirsty, and give you a drink? 
 \V{38}  When did we see you a stranger, and take you to our homes? Or naked, and clothe you? 
 \V{39}  When did we see you ill, or in prison, and come to you?’ 
 \V{40}  And the King will reply ‘I tell you, as often as you did it to one of these my Brothers, however lowly, you did it to me.’ 
 \V{41}  Then he will say to those on his left ‘Go from my presence, accursed, into the ‘aeonian fire which has been prepared for the Devil and his angels.’ 
 \V{42}  For, when I was hungry, you gave me no food; when I was thirsty, you gave me no drink; 
 \V{43}  When I was a stranger, you did not take me to your homes; when I was naked, you did not clothe me; and, when I was ill and in prison, you did not visit me.’ 
 \V{44}  Then they, in their turn, will answer ‘Lord, when did we see you hungry, or thirsty, or a stranger, or naked, or ill, or in prison, and did not supply your wants?’ 
 \V{45}  And then he will reply ‘I tell you, as often as you failed to do it to one of these, however lowly, you failed to do it to me.’ 
 \V{46}  And these last will go away ‘into aeonian punishment,’ but the righteous ‘into aeonian life.’”  \indenting[yes]\par 
 \C{26}  When Jesus had finished all this teaching, he said to his disciples: 
 \V{2}  “You know that in two days time the Festival of the Passover will be here; and that the Son of Man is to be given up to be crucified.”  
 \V{3}  Then the Chief Priests and the Councillors of the Nation met in the house of the High Priest, who was called Caiaphas, 
 \V{4}  And plotted together to arrest Jesus by stealth and put him to death; 
 \V{5}  But they said: “Not during the Festival, for fear of causing a riot.” \indenting[yes]\par 
 \V{6}  After Jesus had reached Bethany, and while he was in the house of Simon the leper, 
 \V{7}  a woman came up to him with an alabaster jar of very costly perfume, and poured the perfume upon his head as he was at table. 
 \V{8}  The disciples were indignant at seeing this. “What is this waste for?” they exclaimed. 
 \V{9}  “It could have been sold for a large sum, and the money given to poor people.” 
 \V{10}  “Why are you troubling the woman?”   Jesus said, when he noticed it.   “For this is a beautiful deed that she has done to me. 
 \V{11}  You always have the poor with you, but you will not always have me. 
 \V{12}  In pouring this perfume on my body, she has done it for my burying. 
 \V{13}  I tell you, wherever, in the whole world, this Good News is proclaimed, what this woman has done will be told in memory of her.”  \indenting[yes]\par 
 \V{14}  It was then that one of the Twelve, named Judas Iscariot, made his way to the Chief Priests, 
 \V{15}  And said “What are you willing to give me, if I betray Jesus to you?” The Priests ‘weighed him out thirty pieces of silver’ as payment. 
 \V{16}  So from that time Judas looked for an opportunity to betray Jesus. \indenting[yes]\par 
 \V{17}  On the first day of the Festival of the Unleavened Bread, the disciples came up to Jesus, and said: “Where do you wish us to make preparations for you to eat the Passover?” 
 \V{18}  “Go into the city to a certain man,”   he answered,   “and say to him ‘The Teacher says — My time is near. I will keep the Passover with my disciples at your house.’”  
 \V{19}  The disciples did as Jesus directed them, and prepared the Passover. 
 \V{20}  In the evening Jesus took his place with the twelve disciples, 
 \V{21}  And, while they were eating, he said:   “I tell you that one of you will betray me.”  
 \V{22}  In great grief they began to say to him, one by one: “Can it be I, Master?” 
 \V{23}  “The one who dipped his bread beside me in the dish,”   replied Jesus,   “is the one who will betray me. 
 \V{24}  True, the Son of Man must go, as Scripture says of him, yet alas for that man by whom the Son of Man is being betrayed! For that man ‘it would be better never to have been born!’”  
 \V{25}  And Judas, who was betraying him, turned to him and said: “Can it be I, Rabbi?”   “It is,”   answered Jesus. \indenting[yes]\par 
 \V{26}  While they were eating, Jesus took some bread, and, after saying the blessing, broke it and, as he gave it to his disciples, said:   “Take it and eat it; this is my body.”  
 \V{27}  Then he took a cup, and, after saying the thanksgiving, gave it to them, with the words:   “Drink from it, all of you; 
 \V{28}  For this is my Covenant blood, which is poured out for many for the forgiveness of sins. 
 \V{29}  And I tell you that I shall never, after this, drink of this juice of the grape, until that day when I shall drink it new with you in the Kingdom of my Father.”  \indenting[yes]\par 
 \V{30}  They then sang a hymn, and went out to the Mount of Olives. 
 \V{31}  Then Jesus said to them:   “Even you will all fall away from me to-night. Scripture says — ‘I will strike down the shepherd, and the sheep of the flock will be scattered.’ 
 \V{32}  But, after I have risen, I shall go before you into Galilee.”  
 \V{33}  “If every one else falls away from you,” Peter answered, “I shall never fall away!” 
 \V{34}  “I tell you,”   replied Jesus,   “that this very night, before the cock crows, you will disown me three times!”  
 \V{35}  “Even if I must die with you,” Peter exclaimed, “I shall never disown you!” All the disciples spoke in the same way. \indenting[yes]\par 
 \V{36}  Then Jesus came with them to a garden called Gethsemane, and he said to his disciples:   “Sit down here while I go and pray yonder.”  
 \V{37}  Taking with him Peter and the two sons of Zebediah, he began to show signs of sadness and deep distress of mind. 
 \V{38}  “I am sad at heart,”   he said,   “sad even to death; wait here and watch with me.”  
 \V{39}  Going on a little further, he threw himself on his face in prayer.   “My Father,”   he said,   “if it is possible, let me be spared this cup; only, not as I will, but as thou willest.”  
 \V{40}  Then he came to his disciples, and found them asleep.   “What!”   he said to Peter,   “could none of you watch with me for one hour? 
 \V{41}  Watch and pray, that you may not fall into temptation. True, the spirit is eager, but human nature is weak.”  
 \V{42}  Again, a second time, he went away, and prayed.   “My Father,”   he said,   “if I cannot be spared this cup, but must drink it, thy will be done!”  
 \V{43}  And coming back again he found them asleep, for their eyes were heavy. 
 \V{44}  So he left them, and went away again, and prayed a third time, again saying the same words. 
 \V{45}  Then he came to the disciples, and said:   “Sleep on now, and rest yourselves. Hark! my time is close at hand, and the Son of Man is being betrayed into the hands of wicked men. 
 \V{46}  Up, and let us be going. Look! my betrayer is close at hand.”  
 \V{47}  And, while he was still speaking, Judas, who was one of the Twelve, came in sight; and with him was a great crowd of people, with swords and clubs, sent from the Chief Priests and Councillors of the Nation. 
 \V{48}  Now the betrayer had arranged a signal with them. “The man whom I kiss,” he had said, “will be the one; arrest him.” 
 \V{49}  So he went up to Jesus at once, and exclaimed: “Welcome, Rabbi!” and kissed him; 
 \V{50}  On which Jesus said to him:   “Friend, do what you have come for.”   Thereupon the men went up, seized Jesus, and arrested him. 
 \V{51}  Suddenly one of those who were with Jesus stretched out his hand, and drew his sword, and striking the High Priest’s servant, cut off his ear. 
 \V{52}  “Sheathe your sword,”   Jesus said,   “for all who draw the sword will be put to the sword. 
 \V{53}  Do you think that I cannot ask my Father for help, when he would at once send to my aid more than twelve legions of angels? 
 \V{54}  But in that case how would the Scriptures be fulfilled, which say that this must be?”  
 \V{55}  Jesus at the same time said to the crowds:   “Have you come out, as if after a robber, with swords and clubs, to take me? I have sat teaching day after day in the Temple Courts, and yet you did not arrest me.”  
 \V{56}  The whole of this occurred in fulfillment of the Prophetic Scriptures. Then the disciples all forsook him and fled. \indenting[yes]\par 
 \V{57}  Those who had arrested Jesus took him to Caiaphas, the High Priest, where the Teachers of the Law and the Councillors had assembled. 
 \V{58}  Peter followed him at a distance as far as the courtyard of the offices, to see the end. 
 \V{59}  Meanwhile the Chief Priests and the whole of the High Council were trying to get such false evidence against Jesus, as would warrant putting him to death, 
 \V{60}  But they did not find any, although many came forward with false evidence. Later on, however, two men came forward and said: 
 \V{61}  “This man said ‘I am able to destroy the Temple of God, and to build it in three days.’” 
 \V{62}  Then the High Priest stood up, and said to Jesus: “Have you no answer? What is this evidence which these men are giving against you?” 
 \V{63}  But Jesus remained silent. On this the High Priest said to him: “I adjure you, by the Living God, to tell us whether you are the Christ, the Son of God.” 
 \V{64}  “It is true,”   Jesus answered;   “Moreover I tell you all that hereafter you shall ‘see the Son of Man sitting on the right hand of the Almighty, and coming on the clouds of the heavens.’”  
 \V{65}  Then the High Priest tore his robes. “This is blasphemy!” he exclaimed. “Why do we want any more witnesses? You have just heard his blasphemy! 
 \V{66}  What is your decision?” They answered: “He deserves death.” 
 \V{67}  Then they spat in his face, and struck him, while others dealt blows at him, saying as they did so: 
 \V{68}  “Now play the Prophet for us, you Christ! Who was it that struck you?” 
 \V{69}  Peter, meanwhile, was sitting outside in the courtyard; and a maidservant came up to him, and exclaimed: “Why, you were with Jesus the Galilean!” 
 \V{70}  But Peter denied it before them all. “I do not know what you mean,” he replied. 
 \V{71}  When he had gone out into the gateway, another maid saw him, and said to those who were there: “This man was with Jesus of Nazareth!” 
 \V{72}  Again he denied it with an oath: “I do not know the man!” 
 \V{73}  But soon afterwards those who were standing by came up and said to Peter: “You also are certainly one of them; why, your very way of speaking proves it!” 
 \V{74}  Then Peter began to swear, with most solemn imprecations: “I do not know the man.” At that moment a cock crowed; 
 \V{75}  and Peter remembered the words which Jesus had said —   ‘Before a cock has crowed, you will disown me three times’;  and he went outside, and wept bitterly. \indenting[yes]\par 
 \C{27}  At daybreak all the Chief Priests and the Councillors of the Nation consulted together against Jesus, to bring about his death. 
 \V{2}  They put him in chains and led him away, and gave him up to the Roman Governor, Pilate. 
 \V{3}  Then Judas, who betrayed him, seeing that Jesus was condemned, repented of what he had done, and returned the thirty pieces of silver to the Chief Priests and Councillors. 
 \V{4}  “I did wrong in betraying a good man to his death,” he said. “What has that to do with us?” they replied. “You must see to that yourself.” 
 \V{5}  Judas flung down the pieces of silver in the Temple, and left; and went away and hanged himself. 
 \V{6}  The Chief Priests took the pieces of silver, but they said: “We must not put them into the Temple treasury, because they are blood-money.” 
 \V{7}  So, after consultation, they bought with them the ‘Potter’s Field’ for a burial-ground for foreigners; 
 \V{8}  And that is why that field is called the ‘Field of Blood’ to this very day. 
 \V{9}  Then it was that these words spoken by the Prophet Jeremiah were fulfilled —  ‘They took the thirty pieces of silver, the price of him who was valued, whom some of the people of Israel valued, 
 \V{10}  And gave them for the Potter’s field, as the Lord commanded me.’ 
 \V{11}  Meanwhile Jesus was brought before the Roman Governor. “Are you the King of the Jews?” asked the Governor.   “It is true,”   answered Jesus. 
 \V{12}  While charges were being brought against him by the Chief Priests and Councillors, Jesus made no reply. 
 \V{13}  Then Pilate said to him: “Do not you hear how many accusations they are making against you?” 
 \V{14}  Yet Jesus made no reply — not even a single word; at which the Governor was greatly astonished. 
 \V{15}  Now, at the Feast, the Governor was accustomed to grant the people the release of any one prisoner whom they might choose. 
 \V{16}  At that time they had a notorious prisoner called Barabbas. 
 \V{17}  So, when the people had collected, Pilate said to them: “Which do you wish me to release for you? Barabbas? Or Jesus who is called ‘Christ’?” 
 \V{18}  For he knew that it was out of jealousy that they had given Jesus up to him. 
 \V{19}  While he was still on the Bench, his wife sent this message to him — “Do not have anything to do with that good man, for I have been very unhappy to-day in a dream on account of him.” 
 \V{20}  But the Chief Priests and the Councillors persuaded the crowds to ask for Barabbas, and to kill Jesus. 
 \V{21}  The Governor, however, said to them: “Which of these two do you wish me to release for you?” “Barabbas,” they answered. 
 \V{22}  “What then,” Pilate asked, “shall I do with Jesus who is called ‘Christ?’ “Let him be crucified,” they all replied. 
 \V{23}  “Why, what harm has he done?” he asked. But they kept shouting furiously: “Let him be crucified!” 
 \V{24}  When Pilate saw that his efforts were unavailing, but that, on the contrary, a riot was beginning, he took some water, and washed his hands in the sight of the crowd, saying as he did so: “I am not answerable for this bloodshed; you must see to it yourselves.” 
 \V{25}  And all the people answered: “His blood be on our heads and on our children’s!” 
 \V{26}  The Pilate released Barabbas to them; but Jesus he scourged, and gave him up to be crucified. \indenting[yes]\par 
 \V{27}  After that, the Governor’s soldiers took Jesus with them into the Government House, and gathered the whole garrison round him. 
 \V{28}  They stripped him, and put on him a red military cloak, 
 \V{29}  And having twisted some thorns into a crown, put it on his head, and a rod in his right hand, and then, going down on their knees before him, they mocked him. “Long life to you, King of the Jews!” they said. 
 \V{30}  They spat at him and, taking the rod, kept striking him on the head; 
 \V{31}  And, when they had left off mocking him, they took off the military cloak, and put his own clothes on him, and led him away to be crucified. \indenting[yes]\par 
 \V{32}  As they were on their way out, they came upon a man from Cyrene of the name of Simon; and they compelled him to go with them to carry the cross. 
 \V{33}  On reaching a place named Golgotha (a place named from its likeness to a skull), 
 \V{34}  they gave him some wine to drink which had been mixed with gall; but after tasting it, Jesus refused to drink it. 
 \V{35}  When they had crucified him, they divided his clothes among them by casting lots. 
 \V{36}  Then they sat down, and kept watch over him there. 
 \V{37}  Above his head they fixed the accusation against him written out — ‘THIS IS JESUS THE KING OF THE JEWS.’ 
 \V{38}  At the same time two robbers were crucified with him, one on the right, the other on the left. 
 \V{39}  The passers-by railed at him, shaking their heads as they said: 
 \V{40}  “You who ‘destroy the Temple and build one in three days,’ save yourself! If you are God’s Son, come down from the cross!” 
 \V{41}  In the same way the Chief Priests, with the Teaches of the Law and Councillors, said in mockery: 
 \V{42}  “He saved others, but he cannot save himself! He is the ‘King of Israel’! Let him come down from the cross now, and we will believe in him. 
 \V{43}  He has trusted in God; if God wants him, let him deliver him now; for he said ‘I am God’s Son.’” 
 \V{44}  Even the robbers, who were crucified with him, reviled him in the same way. 
 \V{45}  After mid-day a darkness came over all the country, lasting till three in the afternoon. 
 \V{46}  And about three Jesus called out loudly:   “Eloi, Eloi, lema sabacthani”   — that is to say,   ‘O my God, my God, why has thou forsaken me?’ 
 \V{47}  Some of those standing by heard this, and said: “The man is calling for Elijah!” 
 \V{48}  One of them immediately ran and took a sponge, and, filling it with common wine, put it on the end of a rod, and offered it to him to drink. 
 \V{49}  But the rest said: “Wait and let us see if Elijah is coming to save him.” However another man took a spear, and pierced his side; and water and blood flowed from it. 
 \V{50}  But Jesus, uttering another loud cry, gave up his spirit. 
 \V{51}  Suddenly the Temple curtain was torn in two from top to bottom, the earth shook, the rocks were torn asunder, 
 \V{52}  the tombs opened, and the bodies of many of God’s People who had fallen asleep rose, 
 \V{53}  And they, leaving their tombs, went, after the resurrection of Jesus, into the Holy City, and appeared to many people. 
 \V{54}  The Roman Captain, and the men with him who were watching Jesus, on seeing the earthquake and all that was happening, became greatly frightened and exclaimed: “This must indeed have been God’s Son!” 
 \V{55}  There were many women there, watching from a distance, who had accompanied Jesus from Galilee and had been attending on him. 
 \V{56}  Among them were Mary of Magdala, Mary the mother of James and Joseph, and the mother of Zebediah’s sons. \indenting[yes]\par 
 \V{57}  When evening had fallen, there came a rich man belonging to Ramah, named Joseph, who had himself become a disciple of Jesus. 
 \V{58}  He went to see Pilate, and asked for the body of Jesus; upon which Pilate ordered it to be given him. 
 \V{59}  So Joseph took the body, and wrapped it in a clean linen sheet, 
 \V{60}  And laid it in his newly-made tomb which he had cut in the rock; and, before he left, he rolled a great stone against the entrance of the tomb. 
 \V{61}  Mary of Magdala and the other Mary remained behind, sitting in front of the grave. \indenting[yes]\par 
 \V{62}  The next day — that is, the day following the Preparation-Day- -the Chief Priests and Pharisees came in a body to Pilate, and said: 
 \V{63}  “Sir, we remember that, during his lifetime, that impostor said   ‘I shall rise after three days.’ 
 \V{64}  So order the tomb to be made secure till the third day. Otherwise his disciples may come and steal him, and then say to the people ‘He has risen from the dead,’ when the latest imposture will be worse than the first.” 
 \V{65}  “You may have a guard,” was Pilate’s reply; “go and make the tomb as secure as you can.” 
 \V{66}  So they went made the tomb secure, by sealing the stone, in presence of the guard. \indenting[yes]\par 
 \C{28}  After the Sabbath, as the first day of the week began to dawn, Mary of Magdala and the other Mary had gone to look at the grave, 
 \V{2}  When suddenly a great earthquake occurred. For an angel of the Lord descended from Heaven, and came and rolled away the stone, and seated himself upon it. 
 \V{3}  His appearance was as dazzling as lightning, and his clothing was as white as snow; 
 \V{4}  And, in their terror of him, the men on guard trembled violently and became like dead men. 
 \V{5}  But the angel, addressing the women, said; “You need not be afraid. I now that it is Jesus, who was crucified, for whom you are looking. 
 \V{6}  He is not here; for he has risen, as he said he would. Come, and see the place where he was lying; 
 \V{7}  And then go quickly and say to him disciples ‘He has risen from the dead, and is going before you into Galilee; there you will see him.’ Remember, I have told you.” 
 \V{8}  On this they left the tomb quickly, in awe and great joy, and ran to tell the news to the disciples. 
 \V{9}  Suddenly Jesus met them.   “Welcome!”   he said. The women went up to him, and clasped his feet, bowing to the ground before him. Then Jesus said to them: 
 \V{10}  “Do not be afraid; go and tell my brothers to set out for Galilee, and they shall see me there.”  
 \V{11}  While they were still on their way, some of the guard came into the city, and reported to the Chief Priests everything that had happened. 
 \V{12}  So they and the Councillors met and, after holding a consultation, gave a large sum of money to the soldiers, 
 \V{13}  And told them to say that his disciples came in the night, and stole him while they were asleep; 
 \V{14}  “And should this matter come before the Governor,” they added, “we will satisfy him, and see that you have nothing to fear.” 
 \V{15}  So the soldiers took the money, and did as they were instructed. And this story has been current among the Jews from that day to this. \indenting[yes]\par 
 \V{16}  The eleven disciples went to Galilee, to the mountain where Jesus told them to meet him; 
 \V{17}  And, when they saw him, they bowed to the ground before him; although some felt doubtful. 
 \V{18}  Then Jesus came up, and spoke to them thus: \indenting[yes]\par  “All authority in heaven and on the earth has been given to me. 
 \V{19}  Therefore go and make disciples of all the nations, baptizing them into the Faith of the Father, the Son, and the Holy Spirit, 
 \V{20}  And teaching them to lay to heart all the commands that I have given you; and, remember, I myself am with you every day until the close of the age.”  \marking[RAChapter]{ } \marking[RABook]{ } \marking[RASection]{ }

        \stoptext

        