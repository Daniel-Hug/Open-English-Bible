\RAHeader{Preface and Frequently Asked Questions}

\section{Preface}

This is a working release of the \quotation{Open English Bible}. It comprises the base Twentieth Century New Testament text with formatting applied but has been built without the growing number of patches (which can be found on the GitHub repository).

Accordingly it covers the New Testament only, with a mostly modern language and sentence structure. However, there are a number of areas in which translation choices dating from the early twentieth century would give a misleading impression to modern readers:

\startitemize
\item
  like the New English Bible and the RSV, the TCNT deliberately retained archaic language for Old Testament quotes and direct addresses to God (which may mislead readers into believing that the distinction is present in the original text),
\item
  it used masculine gendered universals such as 'man' for 'humanity' and 'sons' when a mixed group or ungendered referent is intended, which may mislead readers into the believing that only men were intended to be referred to, and
\item
  it used bare references to 'the Jews' when Jewish authorities or specific groups of Jewish people were intended to be referred to, which may mislead readers as to the theological and practical relationship between early followers of Jesus and their neighbours.
\stopitemize

These issues will be remedied before an official edition of the OEB is released.

\section{Frequently Asked Questions}

\subsubject{Why another translation?}

The English language market is currently very well covered by
translations for all seasons, everything from the Message, through
the CEV, NRSV, AV, and so on through the acronym soup. The OEB aims
at the levels of accuracy and natural language achieved by these
bibles, but the purpose of the OEB is not to gift the world with a
more accurate or \quote{better} translation per se. The existing
commercial English translations are, for the large part, of a high
standard of accuracy and are the result of much work by
knowledgeable and well intended translators.

However, with the honourable exception of the World English Bible
(WEB) - and of course the Authorised Version - all of these bibles
are subject to copyright, and owned by a particular organisation or
publisher. Without wishing to criticise these organisations, their
translations naturally have limitations on what end users can do
with them.

The OEB will have no restrictions on what its readers and users can
do with it (for both good and bad). They may quote it, publish it
in part or full, on their blogs, in their churches, remix it,
reword it, correct its egregious translation mistakes or indeed add
their own.

\subsubject{Isn't this dangerous?}

This generation have a greater ability to confirm the accuracy of
their bible translations than any previous generation in history.
The underlying Greek and Hebrew, lexicons, learned discussions and
flamewars are all just a click away - as are a dozen other
translations. Attempting to use copyright law to create authority
is an exercise doomed to failure, as are all attempts to enforce
truths by the power of the state.

\subsubject{Why not just use the World English Bible?}

The
\useURL[1][http://ebible.org][][World English Bible (WEB)]\from[1]
is an impressive achievement, and of great importance. However its
language, based as it is on the ASV, remains bound to the Tyndale
tradition. As a free counterpart to the NRSV and ESV it is
invaluable; however there is still a need for a free bible as a
counterpart to the NEB/REB and TNIV/NIV translations. As well, the
World English BIble is based upon the Byzantine Majority Text, and
the OEB is based on the heavily Alexandrian focused text of Wescott
\& Hort.

\subsubject{What English language texts is the OEB based on?}

The New Testament of the OEB is being formed on the base of the
\quotation{Twentieth Century New Testament}, in particular the
revised edition published in 1904. There is currently no decision
on how to tackle the Old Testament.

\subsubject{Why the Twentieth Century New Testament?}

The TCNT was one of the earliest 20th century attempts at a
translation in clear modern language aimed at the ordinary reader
and based on a modern textual base (ie Westcott and Hort).
Predating the mid--20th century translations such as the New
English Bible and even Moffatt's groundbreaking attempt, it is out
of copyright worldwide. The TCNT also has a particular resonance
with the open source and free content communities of today - it was
created by a loose collaboration of volunteers rather than a
top-down hierachy. A worthwhile article on the making of the TCNT
can be found on Google Books. Given the requirements of (a) modern
language and (b) public domain status the TCNT was the best
contender.

\subsubject{How will the Twentieth Century New Testament be edited?}

The language of the 20CNT will be edited:

\startitemize
\item
  to reflect modern English usage (including the use of \quote{they}
  as a third person single pronoun) at a reading level corresponding
  roughly to the NEB/REB or NRSV
\item
  to reflect modern scholarship, including on the translation of
  terms such as \quote{the Jews} in John and terms referring to
  sexual practices (see TNIV and Dr Ann Nyland's version)
\stopitemize

This editing will be moderate, aiming for a scholarly defensible
mainstream translation usable within a religious community rather
than a translation focused on a readership completely unfamiliar
with the Bible or Christianity (as an example, the OEB will be
comfortable with the words \quote{Christ} and \quote{Messiah} and
will not replace them with \quote{Annointed One} or similar). A
freely licensed translation for the audiences of the Better Life
Bible or the CEV is a project for another day.

\subsubject{How do I know the OEB is accurate and unbiased?}

Although all translators have opinions on the truth and meaning of
the ancient documents which form the Bible, the intent of this
revision of the TCNT is not to push any particular theological line
but to provide a freely usable translation of an Alexandrian
text-type based critical text in modern English. The
\useURL[1][][][GitHub account]\from[1] for the OEB contains the
original text of the Twentieth Century New Testament in
\useURL[2][http://ubs-icap.org/usfm][][USFM]\from[2], (which can be
checked against
\useURL[3][http://www.archive.org/details/twentiethcentury00newyiala][][the pdf of the published edition on Archive.org]\from[3]);
it also contains a list of all of the changes made to that text. As
can be seen from the article on the making of the TCNT mentioned
above, the TCNT itself was the product of a wide range of
translators of differing backgrounds and was not itself intended to
push a particular theological point of view.

\subsubject{What is the underlying textual basis for the OEB?}

The TCNT was based on the Westcott \& Hort critical text and the
OEB will not change that - it will remain based on W\&H and will
not be conformed to the Majority Text or NA27. Since the
\useURL[4][http://groups.google.com/group/open-scriptures/browse_thread/thread/f57701ff851ae905][][NA27 is subject to a claim of copyright]\from[4],
W\&H remains the best available (ie public domain) text.

\subsubject{What licence is the OEB under?}

The OEB will be available under a
\useURL[5][http://creativecommons.org/licenses/by/3.0/us/][][CC Attribution licence]\from[5],
allowing the maximum reuse. There will be a request that altered
versions be distributed under a different name.

\subsubject{What variations will be available?}

The translation will be available in US and Commonwealth/UK
spelling variations.

\subsubject{What formats will the OEB be available in?}

The OEB will be generated in as many formats as possible, including
PDF, HTML etc and a version will be available for purchase from
Lulu or similar print-on-demand provider.

\subsubject{Who is involved in this?}

At the moment, this is a single person project. I am a lawyer with
strong technical skills, but without formal training in either
translation or in Biblical languages. Accordingly, the OEB is not
an exercise in translation so much as editing. I will make all
changes transparently, and will endeavour to ensure that no new
wording is added that is not attested to in at least one modern,
scholarly, mainstream translation.

\subsubject{Where can I find more information?}

http://openenglishbible.org

\marking[RAChapter]{ } \marking[RABook]{ } \marking[RASection]{ }